% !TEX root = ../../mat701_notes.tex
\newpage
\section{Repeated Integration}
\subsection{Fubini's Theorem}


Given $f(x,y)= \dfrac{1}{\sqrt{xy}}$ on $[0,1] \times [0,1]$, we already have a concept of $\int_{[0,1] \times [0,1]} f$. But how does this relate to the integral
	\[
	\int_0^1 \left( \int_0^1 \dfrac{1}{\sqrt{xy}} \; dy \right) \; dx?
	\]
We have the same issues with double series: $\sum_{i=1}^\infty \sum_{j=1}^\infty a_{ij}$.


\begin{ex}
Consider the infinite matrix with 1s on the main diagonal and $-1$ on the superdiagonal and zeros elsewhere. The sum of each row is 0 and the sum of column is 0 except the first and we obtain
	\[
	\begin{split}
	\sum_{i=1}^\infty \left( \sum_{j=1}^\infty a_{ij} \right)&= 0 \\
	\sum_{j=1}^\infty \left( \sum_{i=1}^\infty a_{ij} \right)&= 1
	\end{split}
	\]
This is the same issues with integrals. For example, take $f=1,-1,0$ and consider
	\[
	\begin{split}
	\int_0^\infty \left( \int_0^\infty f(x,y) \; dx \right) \; dy&= 0 \\
	\int_0^\infty \left( \int_0^\infty f(x,y) \; dy \right) \; dx&= 0
	\end{split}
	\]
Note that $f \notin L^1([0,\infty) \times[0,\infty))$. 
\end{ex}


% Give some extra background on Fubini.


\begin{thm}[Fubini's Theorem]
Let $I_1 \subseteq \R^n$, $I_2 \subseteq \R^m$ be intervals and $I= I_1 \times I_2$. Suppose that $f \in L^1(I)$. Then:
	\begin{enumerate}[(i)]
	\item For almost every $x \in I_1$, the `slice' function $y \mapsto f(x,y)$ is integrable on $I_2$.
	\item The `once-integrated' function $x \mapsto \int_{I_2} f(x,y) \; dy$ is integrable on $I_1$, and its value is $\int_I f$. 
	\end{enumerate}
\end{thm}






Why do we say for almost every $x$? A slice is defined on a set of measure zero. For the matrix example, one could choose a column and insert in a zero entry insert a nonmeasurable function or $\infty$. While the function becomes difficult, its integral is unchanged since its values were changed on a set of measure zero. 


Recall that $\int_I f= \int_{\R^{n \times m}} f \cdot \chi_I$, etc.. So we can reduce the proof to $I_1= \R^n$, $I_2= \R^m$, and $I= \R^{n+m}$ by replacing $f$ with $f \chi_I$. Let $\mathcal{F}:= \{ f \in L^1( \R^{n+m}) \colon \text{Fubini holds for } f \}$. We prove that $\mathcal{F}=L^1$. 

Properties of $\mathcal{F}$:
        \begin{itemize}
        \item $\mathcal{F}$ is a vector space: $f,g \in \mathcal{F}$, then $\alpha f + \beta g \in \mathcal{F}$.
        \item $\mathcal{F}$ is closed under monotone limits: if $f_k \in \mathcal{F}$ and $f_k \nearrow f$ or $f_k \searrow f$, where $f \in L^1(\R^{n \times m})$, then $f \in \mathcal{F}$.
        \item Squeeze Property: if $g \leq f \leq h$, where $g,h \in F$ and $g=h$ a.e., then $f \in \mathcal{F}$
        \item $\chi_I \in F$ for any closed bounded interval $I$. 
        \end{itemize}





To see why the second point holds: 

Proof 2 $f_k \nearrow f$ supplies for every $x \in I_1$, $x$-slice of $f_k \nearrow$ $x$-slice of $f$. For each $k$, there is $Z_k \subset I_1$ with $|Z_k|=0$ such that the $x$-slice of $f_k$ is integrable for $x \notin Z_k$. Let $Z= \cup_k Z_k$, then $|Z|=0$. So MCT, $\int f_k(x,y) \; dy \nearrow \int f(x,y) \; dy$ (though this may be infinite) for $x \notin Z$. Again by MCT, $\int (\int f_k) \; dx \nearrow \int (\int f dy) \; dx$. Then finally, we have $\int_I f_k \to \int_{I_1} f$ by MCT, noting the last integral is finite. But then $\int (\int f dy) \; dx$ is finite since they are equal. But then Fubini's Theorem holds for $f$ as well. 

We now prove the second. For almost all $x$, the slice of $g$ and $h$ are both integrable, and since the slice of $f$ is between, it is also integrable. 


(Proof of 3???) Once integrated functions for $g, f, h$ satisfy $\int g \; dy \leq \int f \; dy \leq \int h \; dy$ for almost all $x$. Integrate with respect to $x$
	\[
	\left( \int g \; dy \right),
	\]
but then 
	\[
	\left( \int g \; dy \right) \; dx \leq \left( \int f \; dy \right) \; dx  \leq \left( \int h \; dy \right) \; dx 
	\]
But the left/right integrals are equal to $\int_I g$ and $\int_I h$, respectively. But these are equal since $g=h$ a.e.. Therefore, $\left( \int f \; dy \right) \; dx= \int_I f$.


Proof of 4: Slice function of $\chi_I$ is either $\equiv 0$ or $\chi_{I_2}$, so $\int \chi_I= |I_2| \chi_{I_1}$. Then $\int \chi_I \;dy \; dx= |I_2| |I_1|= |I|= \int \chi_I$. \qed \\




% Fubini: f \in L^1(I_1 \times I_2)
% Tonelli: f \geq 0, measurable on I_1 \times I_2


Proof of Fubini:

We establish $\mathcal{F}= L^1$. We show this for $G_\delta$ sets and sets of measure 0, which will be enough. 

a. First, we show that $X_I \in \mathcal{F}$ for any closed bounded interval. This does not necessary worked for closed. Indeed, open sets are union of closed sets. 

% Picture: dotted square, the solid squares inside getting closer and closer to full square.

Consider the sequence of characteristic functions on the boxes in the figure, say $\chi_{I_k} \nearrow \chi_I$. Now for a set which is not open or closed, squeeze this between an open set containing and a closed set inside it.

b. Now $\chi_G \in \mathcal{F}$ if $G$ is open and $|G|<\infty$. Tile $G$ by half-open cubes $Q_j$, $\chi_G= \sum \chi_{Q_j}$. Monotone limit of partials sums of 1 and 2 bullets above. 

c. $\chi_{H} \in \mathcal{F}$ if $H$ is a $G_\delta$ set with $|H|<\infty$. Indeed, there exists $G$ open, $H \subset G$, and $|G|<|H|+\infty$ so if $H= \cap_{k=1}^\infty G_k$, $G_k$, with $H= (G \cap G_1) \cap (G\cap G_1 \cap G_2) \cap (G \cap G_1 \cap G_2 \cap G_3) \cap \cdots$ so it is a nested intersection of open sets of finite measure. Then use second bullet above. 

d. Now $\chi_Z \in \mathcal{F}$ if $|Z|=0$. There exists a $G_\delta$ set $H$ such that $Z \subset H$ and $|H|=0$. Now $0 \leq \chi_Z \leq \chi_H$ and $\chi_H=0$ a.e., then by squeeze (bullet 3) have the conclusion. 

e. We now show $\chi_X \in \mathcal{F}$ if $|E|<\infty$, any set of finite measure. Write $E= H \sm Z$. $H$ is $G_\delta$, $|Z|=0$. Then $\chi_E= \chi_H \sm \chi_Z$ and use linearity. 

f. integrable simple functions $\in \mathcal{F}$. Then nonnegative integrable in $\mathcal{F}$. Then integrable in $\mathcal{F}$ since $f= f^+ - f^-$. \qed \\





To use Fubini, we need $f \in L^1$. A lot of times, this is what we are trying to show! Fubini cannot be used to prove $f \in L^1(I_1 \times I_2)$. Take $f(x,y)= \dfrac{1}{\sqrt{xy}}$ on $I:=[0,1] \times [0,1]$. Cannot say $\int_I f = \int_0^1( \int_0^1 \dfrac{1}{\sqrt{xy}} \; dx)\;dy$ but need $f$ integrable, which is what you are trying to show. For these purposes, we use Tunelli's Theorem.

\begin{thm}
Same as Fubini Theorem, $f \geq 0$, measurable. Suppose $f \geq 0$, measurable on $I_1 \times I_2$. Then 

a): FOr a.e. $x \in I_1$, the slice function $y \mapsto f(x,y)$ is measurable on $I_2$.
b): the once-integrated function $x \mapsto \int_{I_2} f(x,y) \; dy$ is measurable on $I_1$ andits integral is $\int_I f$ (could be $\infty$). 
\end{thm}


In the example above, $\int_0^1 \dfrac{1}{\sqrt{xy}} \; dy= \dfrac{2}{\sqrt{x}}$ is measurable and $\int_0^1 \dfrac{2}{\sqrt{x}} \; dx = 4$ so $\int_I \dfrac{1}{\sqrt{xy}}= 4$. 

\pf Let $f_k= \min(f,k) \chi_{[-k,k]^n}$ integrable and $f_k \nearrow f$. Then Fubini applies. Spelled out. $\int_{I_1 \times I_2} f_k = \int_{I_1} \left( \int_{I_2} f_k(x,y) \; dy \right) \; dx$. The inner integral converegs a.e. y to $\int_{I_2} f(x,y) \; dy$ by MCT. THen left MCT $\int_I f$. And then MCT for outer integral. done. \qed \\


Application: 

Let $fL E \to [0,\infty]$ be measurable. $\omega_f(y)= |\{x \in E \colon f(x)>y\}|$. Then $\int_E f = \int_0^\infty \omega_f(y) \; dy$. 

\pf Recall $R(f,E)= \{(x,y) \in \R^{n+1} \colon x \in E, 0 \leq y \leq f(x)\}$. $\Gamma(f,E)= \{(x,y) \in \R^{n+1} \colon y=f(x)\}$. Now $|\Gamma|-0$, $g= \chi_{R \sm \Gamma}$ so $g(x,y)=1$ if $0 \leq y<f(x)$, 0 otherwise. $\int(\int g(x,y) \; dy)\;dx= \int(\int g(x,y) \; dx) \; dy)$. and $\int f(x) \chi_E \; dx = \int \omega_f(y) \; dy$ so $\int_E f= \int_0^\infty \omega_f(y) \; dy$. This last integral is even Riemann integrable since it is monotone. 



Generalizations: $f$ measurable, $\int_E |f|^p= p \int_0^\infty y^{p-1} \omega_{|f|}(y) \; dy$. $p>0$.

We know $\int_E |f|^p= \int_0^\infty \omega_{|f|^p}(y) \; dy$. Let $y=z^p$. Then this is $p \int_0^\infty z^{p-1} \omega_{|f|^p}(z^p) dz= p \int_0^\infty z^{p-1} \omega_{|f|}(z)\;dz$ and $\{ |f|^p>z^p\}= \{|f|>z\}$. \qed \\




Ex: $f(x,y)=\dfrac{\sin xy}{(x^2+y^2)^{3/2}} \in L^1(E)$. $E= \{(x,y) \colon x,y>0\}$. 

Idea: estimate above by nonnegative function $g$ and prove $g \in L^1(E)$ by Tonelli. i.e. $|f| \geq g$. $|f| \geq \dfrac{1}{(x^2+y^2)^{3/2}}$ for sure but problems around origin but good at infinity. Have $|f| \leq \dfrac{xy}{(x^2+y^2)^{3/2}}$ good near 0. So use min of the two functions. Better: Use first when $x^2+y^2 \leq 1$: $g_1=\dfrac{1}{(x^2+y^2)^{3/2}} \chi_{\{x^2+y^2 \geq 1\}}$ and $\omega_{g_1}(\alpha)= \pi/(4\alpha^{2/3}) - \pi/4$ if $\alpha<1$ and 0 otherwise. Note that $1/r^3 \cdot \chi_{r \geq 1}$ and $\alpha<1$ $1/r^3\alpha$ and $r= 1/\alpha^{1/3}$ and $1/4$ if circle roots $1/\alpha^{1/3}$. 



%%%%%%%
% Friday

Silicing a subset of $\R^{n+m}$. Given a measurable set $E \subset \R^{n+m}$, consider $E_x= \{ y \in \R^m \colon (x,y) \in E \}$. 

% Blob with an x-slice and axes

\begin{itemize}
\item $E_x$ is measurable for almost all $x$.
\item If $|E|<\infty$, then $|E_x|<\infty$ for almost every $x$.
\item If $|E|=0$, then $|E_x|=0$ for almost every $x$. 
\end{itemize}

All these things follow easily from applying Tonelli to $\chi_E$ since almost every slice function is measurable. Slice fo $\chi_E$ is $\chi_{E_x}$ and $\int_{\R^{n+m}} \chi_E< \infty$ then $\int_{\R^n} |E_x| \; dx <\infty$.


\begin{ex} % Measurability in \R^{n+m}$ is important.
Assuming the Continuum Hypothesis, there exists $E \subset [0,1]^2$ such that every vertical slice $E_x$ is countable and every horizontal slice $E_y$ is co-countable, i.e. $[0,1] \sm$ countable set. Using TOnelli
	\[
	\int( \int \chi_E \; dy ) \; dx = 0
	\]
since evert vertical slice is countable. The other way
	\[
	\int (\int \chi_E \; dx ) \; dy = 1
	\]
This means $E$ is not measurable in $\R^2$ (otherwise these would be equal). Then let $E= \{ (x,y) \colon y \prec x \}$.
\end{ex}




Products:
If $E_1 \subset \R^n$, $E_2 \subset \R^m$, are measurable, then $E_1 \times E_2$ is measurable in $\R^{n+m}$ and $|E_1 \times E_2|= |E_1| \, |E_2|$, taking $0 \cdot \infty= 0$. Assume $E_1 \times E_2$ is measurable, then
	\[
	|E_1 \times E_2|= \int (\int \chi_{E_1 \times E_2}(x,y) \; dy) \; dx= \int |E_2| \; \chi_{E_1}(x) \; dx.
	\]
It remains to prove that $E_1 \times E_2$ is measurable. It suffices to prove if when $|E_1|<\infty$ and $|E_2|<\infty$. In the general case, $E_1= \cup_{k=1}^\infty E_{1,k}$ and $E_2= \cup_{k=1}^\infty E_{2,k}$, then $E_1 \times E_2= \cup_{j,k} E_{1,k} \times E_{2,j}$. 

There exists $G_1 \supset E_1$, $G_2 \supset E_2$ such that $|G_1 \sm E_1|<\ep$, $|G_2 \sm E_2|<\ep$. Now $G_1 \times G_2$ is open and $E_1 \times E_2 \subset G_1 \times G_2$. Now $(G_1 \times G_2) \sm (E_1 \times E_2) \subset [(G_1 \sm E_1) \times G_2] \cup [G_1 \times (G_2 \sm E_2)]$. Both $G_1 \sm E_1$ and $G_2 \sm E_2$ have measure at most $\ep$. Now $G_1, G_2$ by cumes $I_k, Q_j$ with summed measure at most $\ep$. Then $|I_k \times Q_j|= |I_k||Q_j|$ so $|(G_1 \sm E_1) \times G_2| < \sum_{k,j} |I_k| |Q_j| < \ep |G_2|$. Then just choose $\ep$ sufficiently small. 




Convolution: $(f*g)(x)= \int_{\R^n} f(x-t)g(t) \;dt$, where $f,g$ are defined on $\R^n$ are measurable.  Now $f*g=g*f$ by change of variable $s=x-t$ so $\int f(x-t)g(t) \; dt = \int f(s) g(x-s) \; ds$, where $s,t \in \R^n$. In the general case, $h= h^+ - h^-$. 

Claim $\int h(x-t) \; dt = \int h(t) \; dt$. Proof: if $h \geq 0$, then are measures of sets $\{(t,y) \colon 0 \leq y \leq h(x-t)\}$ and $\{ (t,y) \colon 0 \leq y \leq h(t) \}$. THe first is obtained from the other by reflection $t \mapsto -t$ and translation (add $x$). These do not change measure or measurability. 


\begin{ex}
Ex: Convolution: Argument of $\chi_{[a,b]} *\cdot \chi_{[c,d]}$ is only nonzero on $\{x \colon a \leq x-t \leq b, c \leq t \leq d\}$ so $\{x \colon x-b \leq t \leq x-a, c \leq t \leq d\}$. so on $\max(x-b,c) \leq t \leq \min(x-a,d)$. 
	\[
	\chi_{[a,b]} * \chi_{[c,d]}= \int_\R \chi_{[a,b]}(x-t) \chi_{[c,d]}(t) \; dt =  (\min(x-a,d) - \max(x-b,c))^+
	\]
So $\chi_{[-1,1]} * \chi_{[-1,1]}= (\min(x+1,1) - \max(x-1,-1))^+=$....grossly complicated (or just wrong). So better way $|([-b,-a] + x) \cap [c,d]|$ so $|([-1,1]+x) \cap [-1,1]|$.Then get 
% Graph, triangle with height 2 at x=0, at x-axis at -2,2, and 0 elsewhere, is the plot. 
\end{ex}


\begin{thm}
If $f,g \in L^1(\R^n)$, then $f*g \in L^1(\R^n)$ and $\int |f*g| \leq \int |f| \int |g|$.
\end{thm}

\pf We know $|f*g| \leq |f| * |g|$ by the triangle inequality. We can work with $|f|$ and $|g|$ so assume $f,g \geq 0$. $\int (\int f(x-t) g(t) \; dt) \; dx$ by tonelli this is $\int (\int f(x-t) g(t) \; dx ) \; dt = \int g(t) (\int f(x-t) \; dx \; )dt$ where the inner integral is just $\int f$ by translation. But then this final integral is $\int f \cdot \int g$. 

But why $f(x-t)g(t)$ measurable on $\R^{2n}$? (need this for tonelli). Observe $f(x)$ and $g(t)$ are measurable on $\R^{2n}$ since the preimages of $>$ something is just $\{(x,t) \colon f>a\}= \{x \colon f(x)>a\} \times \R$ since it does not depend on $t$. Then this is measurable. Use the linear transformation $\phi: (x,t) \to (x-t,x+t)$ intvertible inverse linear, hence lipschitz. Use then HW 3.5: composition $f \circ \phi$ is measurable, $\phi^{-1}$ Lipschitz. Hence $f(x-t)$ is measurable on $\R^{2n}$. 





%%%%%%%%%%%%

If $f,g$ are probability density functions of $X$ and $Y$, then $f*g$ is the PDF of $X+Y$, assuming $X,Y$ are independent. 





\subsection{Marcinkiewitz Integrals}

In Analysis, one often has $f*g$, where one of the functions is not in $L^1$. These are so called ``singular integrals.'' The simplest example is the Hilbert transform
	\[
	\int_\R \dfrac{f(x-t)}{t} \; dt
	\]

% Look this up and give better description.

Let $K$ be a compact set in $\R^n$, and $\delta(y):= \dist(y,k)= \min\{|x-y| \colon x \in K\}$. 

% Picture?


\begin{thm}
If $B \supset K$ is a bounded open set, then for all $\lambda > 0$. 
	\[
	\int \dfrac{\delta(y)^\lambda}{|x-y|^{n+\lambda}} \; dy < \infty
	\]
for a.e. $x \in K$.
\end{thm}

\pf It suffices to show $\int_K \mu_\lambda(x) \; dx < \infty$.
	\[
	\int_K \mu_\lambda(x) \; dx = \int_K \int_{B \sm K} \dfrac{\delta(y)^\lambda}{|x-y|^{n+\lambda}} \; dy \; dx = \int_{B \sm K} \int_K \dfrac{\delta(y)^\lambda}{|x-y|^{n+\lambda}} \; dx \; dy.
	\]
But now observe $|x-y| \geq \delta(y)$. But then 
	\[
	\int_{B \sm K} \int_K \dfrac{\delta(y)^\lambda}{|x-y|^{n+\lambda}} \; dx \; dy \leq \int_{B \sm K} \left( \int_{|x-y| \geq \delta(y)} \dfrac{\delta(y)^\lambda}{|x-y|^{n+\lambda}} \right) \; dy \leq \int_{B \sm K} C \; dx = C |B \sm K|< \infty.
	\]
Using $\int_{|x-y| \geq \delta} \dfrac{1}{|x-y|^{n+\lambda}} \; dx \leq \dfrac{C}{\delta^\lambda}$. To prove this:

Recall $\int |f|^p= p \int_0^\infty \alpha^{p-1} |\{|f|>\alpha\}| \; d \alpha= *$. Use $P= n + \lambda$, $f=\dfrac{1}{|x-y|}$.
	\[
	| \{ |x-y| \geq \delta \colon \dfrac{1}{|x-y|}>\alpha\}| = |B(y,1/\alpha) \sm B(y,\delta)|= C((1/\alpha)^n - \delta^n)^+
	\]
which is $=0$ if $\alpha>1/\delta$. For *, note
	\[
	*= (n+\lambda)^{1/\delta} \alpha^{n+\lambda-1} C (1/\alpha^n - \delta^n) \; d\alpha \leq C' \int_0^{1/\delta^n} \alpha^{\lambda-1} \; d \alpha = C'' \dfrac{1}{\delta^\lambda}
	\]





$B$ is needed as the integral is $\sim \dfrac{1}{|y|^n}$ as $y \to \infty$. We could have integrate over $B \sm K$ since $\delta \equiv 0$ on $K$. Also, the $\delta$ is important for convergence. The almost always is also unavoidable: take $K=\{0\}$ in $\R$. Then $\delta(y)= |y|$. Then $\int_B \dfrac{|y|^\lambda}{|x-y|^{n+\lambda}} \; dy$ so at $x=0$, we get $\int_B \dfrac{1}{|y|^n}= \infty$. Generally, $\mu_\lambda(x)= \infty$ if $x$ is an isolated point of $K$. If $x$ is an interior point fo $K$, $\mu_\lambda(x)< \infty$ as $\delta=0$ in a neighborhood of $x$.




Question: in last part of proof, can we let $\lambda=0$? Specifically, is it true that $\int_{B \sm K} \dfrac{1}{|x-y|^n} \; dy< \infty$ for a.e. $x \in K$? Counterexample? $K$ is fat Cantor set?




%%%%%%%%%%%%%%%%%



8.1 $L^p$ Classes, Revisited

Want to build a theory of complete spaces based on integration. 


% Noncomplete: d(f,g)= (R) \int_0^1 |f-g| not complete. Take sequence convergeing to dirichlet. 

% F_1,\ldots,F_n,\ldots such that \bigcap_{k=1}^\infty F_k=F, Fat Cantor set. \chi_{F_k} Riemann integrable but \chi_F is not, \int |\chi_{F_k}- \chi_{F_j}|< \ep for k,j \geq M, lim \chi_F DNE wrt metric d.

First, measurability of functions $f: E \to \C$ (or $E \to \R$). Then $f=(f_1,\ldots,f_m)$ is measurable by definition if and only if $f_1,\ldots,f_m$ are measurable. Equivalently, $\{a_j<f_j<b_j\}= f^{-1}(\prod_{j=1}^m (a_j,b_j))$ are measurable for all $j$, $a_j,b_j \in \R$. Equivalently, every countable union of open boxes $U$ $f^{-1}(U)$ is measurable. 


Definition of $L^p(E)$: All measurable functions $f: E \to \C$ such that $\int_E |f|^p< \infty$. Here $0<p< \infty$, and if $f=g$ a.e., we consider them the same element of $L^p(E)$. 

Strictly speaking, $L^p=$ equivalence classes $[f]=$ all $g$ such that $g=f$ a.e.. 

Operations $c \cdot f$ ($c \in \C$) is also in $L^p(E)$. $c[f]=[cf]$. $f+g$ is in $L^p$ too:
	\[
	|f+g|^p \leq (|f|+|g|)^p \leq [2 \max(|f|,|g|)]^p = 2^p \max(|f|^p,|g|^p) \leq 2^p(|f|^p+|g|^p)
	\]
which is integrable. So $L^p$ is a vector space. 


\begin{dfn}[Essential Supremum]
$\esup f= \inf\{ \mu \colon f \leq \mu\}$ a.e. 
\end{dfn}


Some may call this essentially bounded. 


\begin{ex}
$\esup \chi_\Q=0$
$\esup \chi_F=1$, where $F$ fat cantor set. 
\end{ex}


$L^\infty(E)= \{meas f: E \to \C \colon \esup |f|<\infty\}$ (there exists $M \in \R$ such that $|f| \leq M$ a.e.). 

$\dfrac{1}{|x|}, x \in \R$ and $\infty$ $x=0$. Is this in $L^\infty$? This is finite a.e. but it is not bounded. 

\begin{ex}
$\dfrac{1}{\sqrt{x}}$ on $E=(0,1)$ is in $L^p$ if $\int_{(0,1)} \left(\dfrac{1}{\sqrt{x}}\right)^p<\infty$ But $\int_{(0,1)} \dfrac{1}{x^{p/2}}< \infty$ if and only if $p/2<1$ if and only $p<2$.

b) $\dfrac{1}{\sqrt{x}}$ on $E=[1,\infty)$ is in $L^p$ if and only if $\int_{[1,\infty)} \dfrac{1}{x^{p/2}} \; dx$ if and only if $p>2$ and $\infty$. ($p \in [2,\infty]$). 
\end{ex}


Why do we need $L^p$? Ranked from nicest to ugliest: 1) $L^2$, 2) $L^p$ $1<p<\infty$, $p \neq 2$,3) $L^1$, 4) $L^\infty$, $L^p$, $0<p<1$. 


Relations between $L^p$ spaces:

A) if $|E|<\infty$, and $f \in L^{p_1}$ then $f \in L^{p_2}$ for all $p_2<p_1$: $\{p \colon f \in L^p\}$ is an interval starting from 0.
b) if $f \in L^{p_1}$ and $f \in L^{p_2}$, then $f \in L^p$ for all $p_1<p<p_2$: $\{p \colon f\in L^p\}$ is an interval.


Proof is based on $\int |f|^p= \underbrace{\inf |f|^p}_{|f| \leq 1} + \underbrace{\int |f|^p}_{|f|>1}$, the first decreases as $p$ increases and the second increass as $p$ increases.

So for b we have
	\[
	\int_{|f| \leq 1} |f|^p \leq \int_{|f| \leq 1} |f|^{p_1} < \infty
	\]
	\[
	\int_{|f|>1} |f|^p \leq \int_{|f|>1} |f|^{p_2} < \infty
	\]
Proving $f \in L^p$ except $p_2=\infty$ is a special case. let $M= \esup |f|$ 
	\[
	\int_{|f|>1} \leq \int_{|f|>1} M^p \leq M^p |\{|f|>1\}| \leq M^p |\{|f|^{p_1}>1\}| \leq M^p \int |f|^{p_1} < \infty
	\]


Proof of A) $\int_{|f| \leq 1} |f|^p \leq \int_E 1 \leq |E|<\infty$ and for $\int_{|f|>1}$ use inequality as before. 




Are there any functions which are in none? $f \equiv 1$ on $\R$, $f \notin L^p$ for all $p$ except $p=\infty$. For none, use $f \equiv x$ on $\R$. Find an example of such on $(0,1)$ too. 




Fact: if $|E|<\infty$, $\esup_E |f|= \lim_{p \to \infty} \left( \int_E |f|^p\right)^{1/p}$. Notation: $\|f\|_{p,E}= \begin{cases} \left( \int_E |f|^p\right)^{1/p}, & 0<p<\infty \\ \esup_E |f|, 0<p<1 \end{cases}$. 



%%%%%%%%%%%%%
% 8.2
\subsection{H\"older and Minkowski Inequalities}


H\"older's inequality is about products and Minkowski is about sums. Recall $\|f\|_{p,E}= \left( \int_e |f|^p \right)^{1/p}$, $\|f\|_{\infty,E}= \esup_E |f|$. 


\begin{thm}[Minkowski's Inequality]
If $1 \leq p \leq \infty$, then $\|f+g\|_p \leq \|f\|_p + \|g\|_p$. 
\end{thm}

\pf Proof for $1<p<\infty$. $\int |f+g|^p= \int |f+g|^{p-1} |f+g| \leq \int |f| \cdot |f+g|^{p-1} + \int |g| \, |f+g|^{p-1}$. Using H\"older's inequality, we have $\leq \|f\|_p \left( \int (|f+g|^{p-1})^{p'} \right)^{1/p'} + \|g\|_p \left( \int (|f+g|^{p-1})^{p'}\right)^{1/p'} \leq (\|f\|_p + \|g\|_p) \left( \int |f+g|^p \right)^{1-1/p}$. Cancel (divide both by rt), then $\left( \int |f+g|^p \right)^{1/p} \leq \|f\|_p + \|g\|_p$. \qed \\


For infinite series: if $\sum_k \|f_k\|_p < \infty$, then $\sum f_k \in L^p$ and $\| \sum f_k \|_p \leq \sum \|f_k\|_p$ for $1 \leq p \leq \infty$. 

\pf Minkowski then $\| \sum_{k=1}^N |f_k|\|_p \leq \sum_k \|f_k\|_p$ then MCT allows $N \to \infty$ here. So $\sum |f_k| \in L^p$. Hence, $\sum f_k \in L^p$.




Axioms of a norm: 
$\|f\| \geq 0$, $\|f\|=0$ if and only if $f=0$: Positivitiy
$\|\lambda f\|= |\lambda| \, \|f\|$ for $\lambda \in \C$ or $\R$ homogeneity
$\|f+g\| \leq \|f\|+\|g\|$ subadditivity (triangle inequality)

So $\|f\|_p$ is a norm when $1 \leq p \leq \infty$ (and not if $0<p<1$). 

For $p=1$ and $p= \infty$, this is easy: $p=1$: $\int |f+g| \leq \int (|f|+|g|)= \int |f|+ \int |g|$. For $p=\infty$, $|f| \leq M$ a.e. $|g| \leq N$ a.e., then $|f+g| \leq M+N$ a.e.. For $1<p<\infty$, we need H\"older's inequality. 


Given $p \in [1,\infty]$, let $p' = \dfrac{p}{p-1}$; so $\frac{1}{p} + \frac{1}{p'}=1$, the conjugate exponent, where $p' \in [1,\infty]$. Clearly, 1 and $\infty$ are conjugate. Other pairs are $(1.1,11)$, $(1.5,3)$, and one self conjugate $(2,2)$. 

% Number line pairing them together

\begin{thm}[H\"older's Inequality]
$\int |fg| \leq \|f\|_p \|g\|_p$ for $1 \leq p \leq \infty$. 
\end{thm}

\pf Three steps: 
1. Pointwise (algebraic) inequality: $a,b \geq 0$, then $ab \leq a^p/p + b^{p'}/p'$: Young's Inequality. Key to it $x^{p-1}$ and $x^{p'-1}$ are inverses. $p'-1 = \frac{1}{p-1}$. 
2. Apply 1 to $|f|$ and $|g|$: $\int |fg| \leq 1/p \int |f|^p + 1/p' \int |g|^p$. 
3. Optimize using homogeneity. let $t>0$ then $\int |fg|= \int |tf \cdot 1/t g| \leq t^p/p \int |f|^p + 1/(p' t^{p'}) \int |g|^{p'}$. Differentiate wrt $t$. $t^{p-1} \int |f|^p- t^{-p'-1} \int |g|^{p'} = 0$ then $t^{p+p'}= (\int |g|^{p'})/(\int |f|^p)$, plut in this $t$ to the above. 

Alternatively, replace $f$ by $t/\|f\|_p$ and $g$ by $g/\|g\|_{p'}$, i.e. normalize them. 






Special cases: $p=1$: $\int |fg| \leq \esup |g| \cdot \int |f|$, easy.

$p=2$: $\int |fg| \leq \sqrt{\int |f|^2 \int |g|^2}$, Cauchy-Schwartz

Notice outer exponents must always add to 1. 


\begin{ex}
Give $f \in L^6([1,\infty])$. Want to estimate
	\[
	\int_1^\infty \dfrac{f(x)^2}{x} \; dx = \int f(x)^2 \cdot \dfrac{1}{x} \; dx  \leq \left(\int_1^\infty (f^2)^3 \right)^{1/3} \left( \int_1^\infty \dfrac{1}{x^{3/2}} \right)^{2/3}= \left(\left(\int_1^\infty f^6 \right)^{1/6} \right)^2 \cdot 2= 2^{2/3} \|f\|_6^2
	\]
Think what power do you need so that you can use the given information? So $P=3, p'= 3/2$. That is the idea of the example. 
\end{ex}












































