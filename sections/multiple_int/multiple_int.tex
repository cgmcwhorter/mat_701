% !TEX root = ../../mat701_notes.tex
\newpage
\section{Repeated Integration}
\subsection{Fubini's Theorem}


Given $f(x,y)= \dfrac{1}{\sqrt{xy}}$ on $[0,1] \times [0,1]$, we already have a concept of $\int_{[0,1] \times [0,1]} f$. But how does this relate to the integral
	\[
	\int_0^1 \left( \int_0^1 \dfrac{1}{\sqrt{xy}} \; dy \right) \; dx?
	\]
We have the same issues with double series: $\sum_{i=1}^\infty \sum_{j=1}^\infty a_{ij}$.


\begin{ex}
Consider the infinite matrix with 1s on the main diagonal and $-1$ on the superdiagonal and zeros elsewhere. The sum of each row is 0 and the sum of column is 0 except the first and we obtain
	\[
	\begin{split}
	\sum_{i=1}^\infty \left( \sum_{j=1}^\infty a_{ij} \right)&= 0 \\
	\sum_{j=1}^\infty \left( \sum_{i=1}^\infty a_{ij} \right)&= 1
	\end{split}
	\]
This is the same issues with integrals. For example, take $f=1,-1,0$ and consider
	\[
	\begin{split}
	\int_0^\infty \left( \int_0^\infty f(x,y) \; dx \right) \; dy&= 0 \\
	\int_0^\infty \left( \int_0^\infty f(x,y) \; dy \right) \; dx&= 0
	\end{split}
	\]
Note that $f \notin L^1([0,\infty) \times[0,\infty))$. 
\end{ex}


% Give some extra background on Fubini.


\begin{thm}[Fubini's Theorem]
Let $I_1 \subseteq \R^n$, $I_2 \subseteq \R^m$ be intervals and $I= I_1 \times I_2$. Suppose that $f \in L^1(I)$. Then:
	\begin{enumerate}[(i)]
	\item For almost every $x \in I_1$, the `slice' function $y \mapsto f(x,y)$ is integrable on $I_2$.
	\item The `once-integrated' function $x \mapsto \int_{I_2} f(x,y) \; dy$ is integrable on $I_1$, and its value is $\int_I f$. 
	\end{enumerate}
\end{thm}






Why do we say for almost every $x$? A slice is defined on a set of measure zero. For the matrix example, one could choose a column and insert in a zero entry insert a nonmeasurable function or $\infty$. While the function becomes difficult, its integral is unchanged since its values were changed on a set of measure zero. 


Recall that $\int_I f= \int_{\R^{n \times m}} f \cdot \chi_I$, etc.. So we can reduce the proof to $I_1= \R^n$, $I_2= \R^m$, and $I= \R^{n+m}$ by replacing $f$ with $f \chi_I$. Let $\mathcal{F}:= \{ f \in L^1( \R^{n+m}) \colon \text{Fubini holds for } f \}$. We prove that $\mathcal{F}=L^1$. 

Properties of $\mathcal{F}$:
        \begin{itemize}
        \item $\mathcal{F}$ is a vector space: $f,g \in \mathcal{F}$, then $\alpha f + \beta g \in \mathcal{F}$.
        \item $\mathcal{F}$ is closed under monotone limits: if $f_k \in \mathcal{F}$ and $f_k \nearrow f$ or $f_k \searrow f$, where $f \in L^1(\R^{n \times m})$, then $f \in \mathcal{F}$.
        \item Squeeze Property: if $g \leq f \leq h$, where $g,h \in F$ and $g=h$ a.e., then $f \in \mathcal{F}$
        \item $\chi_I \in F$ for any closed bounded interval $I$. 
        \end{itemize}





To see why the second point holds: 

Proof 2 $f_k \nearrow f$ supplies for every $x \in I_1$, $x$-slice of $f_k \nearrow$ $x$-slice of $f$. For each $k$, there is $Z_k \subset I_1$ with $|Z_k|=0$ such that the $x$-slice of $f_k$ is integrable for $x \notin Z_k$. Let $Z= \cup_k Z_k$, then $|Z|=0$. So MCT, $\int f_k(x,y) \; dy \nearrow \int f(x,y) \; dy$ (though this may be infinite) for $x \notin Z$. Again by MCT, $\int (\int f_k) \; dx \nearrow \int (\int f dy) \; dx$. Then finally, we have $\int_I f_k \to \int_{I_1} f$ by MCT, noting the last integral is finite. But then $\int (\int f dy) \; dx$ is finite since they are equal. But then Fubini's Theorem holds for $f$ as well. 

We now prove the second. For almost all $x$, the slice of $g$ and $h$ are both integrable, and since the slice of $f$ is between, it is also integrable. 


(Proof of 3???) Once integrated functions for $g, f, h$ satisfy $\int g \; dy \leq \int f \; dy \leq \int h \; dy$ for almost all $x$. Integrate with respect to $x$
	\[
	\left( \int g \; dy \right),
	\]
but then 
	\[
	\left( \int g \; dy \right) \; dx \leq \left( \int f \; dy \right) \; dx  \leq \left( \int h \; dy \right) \; dx 
	\]
But the left/right integrals are equal to $\int_I g$ and $\int_I h$, respectively. But these are equal since $g=h$ a.e.. Therefore, $\left( \int f \; dy \right) \; dx= \int_I f$.


Proof of 4: Slice function of $\chi_I$ is either $\equiv 0$ or $\chi_{I_2}$, so $\int \chi_I= |I_2| \chi_{I_1}$. Then $\int \chi_I \;dy \; dx= |I_2| |I_1|= |I|= \int \chi_I$. \qed \\




% Fubini: f \in L^1(I_1 \times I_2)
% Tonelli: f \geq 0, measurable on I_1 \times I_2


Proof of Fubini:

We establish $\mathcal{F}= L^1$. We show this for $G_\delta$ sets and sets of measure 0, which will be enough. 

a. First, we show that $X_I \in \mathcal{F}$ for any closed bounded interval. This does not necessary worked for closed. Indeed, open sets are union of closed sets. 

% Picture: dotted square, the solid squares inside getting closer and closer to full square.

Consider the sequence of characteristic functions on the boxes in the figure, say $\chi_{I_k} \nearrow \chi_I$. Now for a set which is not open or closed, squeeze this between an open set containing and a closed set inside it.

b. Now $\chi_G \in \mathcal{F}$ if $G$ is open and $|G|<\infty$. Tile $G$ by half-open cubes $Q_j$, $\chi_G= \sum \chi_{Q_j}$. Monotone limit of partials sums of 1 and 2 bullets above. 

c. $\chi_{H} \in \mathcal{F}$ if $H$ is a $G_\delta$ set with $|H|<\infty$. Indeed, there exists $G$ open, $H \subset G$, and $|G|<|H|+\infty$ so if $H= \cap_{k=1}^\infty G_k$, $G_k$, with $H= (G \cap G_1) \cap (G\cap G_1 \cap G_2) \cap (G \cap G_1 \cap G_2 \cap G_3) \cap \cdots$ so it is a nested intersection of open sets of finite measure. Then use second bullet above. 

d. Now $\chi_Z \in \mathcal{F}$ if $|Z|=0$. There exists a $G_\delta$ set $H$ such that $Z \subset H$ and $|H|=0$. Now $0 \leq \chi_Z \leq \chi_H$ and $\chi_H=0$ a.e., then by squeeze (bullet 3) have the conclusion. 

e. We now show $\chi_X \in \mathcal{F}$ if $|E|<\infty$, any set of finite measure. Write $E= H \sm Z$. $H$ is $G_\delta$, $|Z|=0$. Then $\chi_E= \chi_H \sm \chi_Z$ and use linearity. 

f. integrable simple functions $\in \mathcal{F}$. Then nonnegative integrable in $\mathcal{F}$. Then integrable in $\mathcal{F}$ since $f= f^+ - f^-$. \qed \\





To use Fubini, we need $f \in L^1$. A lot of times, this is what we are trying to show! Fubini cannot be used to prove $f \in L^1(I_1 \times I_2)$. Take $f(x,y)= \dfrac{1}{\sqrt{xy}}$ on $I:=[0,1] \times [0,1]$. Cannot say $\int_I f = \int_0^1( \int_0^1 \dfrac{1}{\sqrt{xy}} \; dx)\;dy$ but need $f$ integrable, which is what you are trying to show. For these purposes, we use Tunelli's Theorem.

\begin{thm}
Same as Fubini Theorem, $f \geq 0$, measurable. Suppose $f \geq 0$, measurable on $I_1 \times I_2$. Then 

a): FOr a.e. $x \in I_1$, the slice function $y \mapsto f(x,y)$ is measurable on $I_2$.
b): the once-integrated function $x \mapsto \int_{I_2} f(x,y) \; dy$ is measurable on $I_1$ andits integral is $\int_I f$ (could be $\infty$). 
\end{thm}


In the example above, $\int_0^1 \dfrac{1}{\sqrt{xy}} \; dy= \dfrac{2}{\sqrt{x}}$ is measurable and $\int_0^1 \dfrac{2}{\sqrt{x}} \; dx = 4$ so $\int_I \dfrac{1}{\sqrt{xy}}= 4$. 

\pf Let $f_k= \min(f,k) \chi_{[-k,k]^n}$ integrable and $f_k \nearrow f$. Then Fubini applies. Spelled out. $\int_{I_1 \times I_2} f_k = \int_{I_1} \left( \int_{I_2} f_k(x,y) \; dy \right) \; dx$. The inner integral converegs a.e. y to $\int_{I_2} f(x,y) \; dy$ by MCT. THen left MCT $\int_I f$. And then MCT for outer integral. done. \qed \\


Application: 

Let $fL E \to [0,\infty]$ be measurable. $\omega_f(y)= |\{x \in E \colon f(x)>y\}|$. Then $\int_E f = \int_0^\infty \omega_f(y) \; dy$. 

\pf Recall $R(f,E)= \{(x,y) \in \R^{n+1} \colon x \in E, 0 \leq y \leq f(x)\}$. $\Gamma(f,E)= \{(x,y) \in \R^{n+1} \colon y=f(x)\}$. Now $|\Gamma|-0$, $g= \chi_{R \sm \Gamma}$ so $g(x,y)=1$ if $0 \leq y<f(x)$, 0 otherwise. $\int(\int g(x,y) \; dy)\;dx= \int(\int g(x,y) \; dx) \; dy)$. and $\int f(x) \chi_E \; dx = \int \omega_f(y) \; dy$ so $\int_E f= \int_0^\infty \omega_f(y) \; dy$. This last integral is even Riemann integrable since it is monotone. 



Generalizations: $f$ measurable, $\int_E |f|^p= p \int_0^\infty y^{p-1} \omega_{|f|}(y) \; dy$. $p>0$.

We know $\int_E |f|^p= \int_0^\infty \omega_{|f|^p}(y) \; dy$. Let $y=z^p$. Then this is $p \int_0^\infty z^{p-1} \omega_{|f|^p}(z^p) dz= p \int_0^\infty z^{p-1} \omega_{|f|}(z)\;dz$ and $\{ |f|^p>z^p\}= \{|f|>z\}$. \qed \\




Ex: $f(x,y)=\dfrac{\sin xy}{(x^2+y^2)^{3/2}} \in L^1(E)$. $E= \{(x,y) \colon x,y>0\}$. 

Idea: estimate aboev by nonnegative function $g$ and prove $g \in L^1(E)$ by Tonelli. i.e. $|f| \geq g$. $|f| \geq \dfrac{1}{(x^2+y^2)^{3/2}}$ for sure but problems around origin but good at infinity. Have $|f| \leq \dfrac{xy}{(x^2+y^2)^{3/2}}$ good near 0. So use min of the two functions. Better: Use first when $x^2+y^2 \leq 1$: $g_1=\dfrac{1}{(x^2+y^2)^{3/2}} \chi_{\{x^2+y^2 \geq 1\}}$ and $\omega_{g_1}(\alpha)= \pi/(4\alpha^{2/3}) - \pi/4$ if $\alpha<1$ and 0 otherwise. Note that $1/r^3 \cdot \chi_{r \geq 1}$ and $\alpha<1$ $1/r^3\alpha$ and $r= 1/\alpha^{1/3}$ and $1/4$ if circle roots $1/\alpha^{1/3}$. 



























