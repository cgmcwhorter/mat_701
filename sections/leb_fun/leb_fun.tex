% !TEX root = ../../mat701_notes.tex
\newpage
\section{Lebesgue Measurable Functions}
\subsection{Measurable Functions}

It is still our goal to build a generalized theory of integration for functions. However, there are two clear potential problems for functions which we might integrate:
	\begin{itemize}
	\item The graph of $f$ could cover an infinite area and thus not be integrable in the traditional sense, e.g. $f(x)=\frac{1}{x}$ on the interval $(0,1)$. 
		\[
            	\begin{tikzpicture}
		\draw[thick,domain= 0.2:5,smooth,variable=\x] plot ({\x},{1/\x}); 
            
                	\coordinate (XAxisMin) at (-0.5,0);
                	\coordinate (XAxisMax) at (5,0);
                	\coordinate (YAxisMin) at (0,-0.5);
                	\coordinate (YAxisMax) at (0,5);
                	\draw [thin,-latex] (XAxisMin) -- (XAxisMax);
                	\draw [thin,-latex] (YAxisMin) -- (YAxisMax);	
            	\end{tikzpicture}
		\]
	\item The graph of $f$ could be so irregular that the `area underneath' is not well defined enough so that such a question even makes sense, e.g.  
		\[
		\chi_V(x)=
		\begin{cases}
		1, & x \in V \\
		0, & x \notin V
		\end{cases}
		\]
	where $V$ is the Vitale set. 
	\end{itemize}
As in the case of the Riemann integral, only certain classes of functions will be integrable in the Lebesgue sense. It will be the goal of this section to define `nice' classes of functions which address the issues above and which will later form a `nice' class of functions with which to form an integration theory around. 


\begin{dfn}[Measurable Function]
Let $E \subseteq \R^n$ and $\overline{\R}$ denote the extended real line. We say that a function $f: E \to \overline{\R}$ is (Lebesgue) measurable if for all $a \in \R$, the set $\{f>a\}:=\{ x \in E \colon f(x)>a\}$ is measurable. 
\end{dfn}


Note that we define $\{ f=\infty\}:= \bigcap_{k \in \Z} \{ f>k \}$ and $\{ f= -\infty\}:= \bigcap_{k \in \Z} \{ -f>k\}$. The reason for this choice of definition is that the sets $\{f > a\}$ describe the distribution for the values of $f$. In some sense, the `smoother' $f$ is, the smaller the variety of such sets one obtains from $f$. For example, when $f$ is continuous, the sets $\{f>a\}$ are open sets since $f^{-1}((a,\infty))$ is an open set. 


Furthermore, the measurability of a set should be linked in some way with the potential measurability of the functions on it and vice versa. If the set $E$ is `wild', less functions should be measurable on it. In fact, the measurability of $E$ is equivalent to that of the set $\{ f= -\infty\}$ since
	\[
	E= \{ f= -\infty\} \cup \left( \bigcup_{k=1}^\infty \{f > -k\} \right),
	\]
showing that if $f$ is measurable, then $E$ is measurable if and only if $\{ f= -\infty\}$ is measurable. 


Note that one could define Borel measurability similarly: $f: E \to \ov{\R}$ is Borel measurable, if for all $a \in \R$, the set $\{ f>a\}$ is Borel. Thus, every Borel function is also measurable. We can of course define measurability of functions more generally: if $M$ is a $\sigma$-algebra on $\R^n$, $f$ is $M$-measurable if $\{f>a\} \in M$ for all $a \in \R$. For us, $M$ will normally be the Lebesgue $\sigma$-algebra. Though what we do is in $\R^n$ with the Lebesgue $\sigma$-algebra, much of what we do holds, or has parallel results, in other spaces and $\sigma$-algebras. 





























\begin{ex}
If $M$ is the $\sigma$-algebra generated by $[0,1]$ and $[1,2]$. What functions $f$ are $M$-measurable?

$M$ is $\sigma$ algebra generated by $[0,1]$ and $[1,2]$. What $f$ are $M$-measurable?

$f$ over $[0,1]$, $[1,2]$ constant functions, over complements. Not $M$-measurable, $\chi_{\{3\}}$, $f(x)=x$ or $f(x)= \sin x$. 

% Give plot
\end{ex}


Before continuing, we make an important definition since the following abbreviation will serve as a constant shortening of the words needed to state, prove, and discuss results later, so we define it now.


\begin{dfn}[Almost Everywhere, a.e.]
We say that a property holds `almost everywhere' if it holds everywhere except on a set of exceptional values with measure zero, e.g. $f$ is finite almost everywhere signifies $|\{ |f|=\infty\}|=0$. We shall abbreviate this as `a.e.'.
\end{dfn}




% ------------------------
%  Properties of Measurable Functions
% ------------------------
\subsection{Properties of Measurable Functions}

1. Continuous then measurable as $\{f>a\}$ are open.

2. Enough to check that $\{f>a\}$ is measurable for a in some dense set $\mathcal{D} \subset \R$. Why? For all real $a$, for all $k$, find $d_k \in \mathcal{D}$, $a<d_k<a+1/k$. $\{f>a\} \cup_k \{f>d_k\}$.

3. Instead of $\{f>a\}$, it could be $\{f<a\}$ as $\{f \geq a\}$, $\{ f \leq a\}$, can use any of these 4 ways to check measurability.
	\[
	\begin{cases}
	\{f \geq a\}&= \bigcap_{k \in \N} \{f>a-1/k\} \\
	\{f>a\}&= \bigcup_{k \in \N} \{f \geq a+1/k\}
	\end{cases}
	\]

Note that 2,3 are independent of the $\sigma$-algebra. 


4. If $f$ is measurable and $f=g$ a.e., then $g$ is measurable (Borel fails this). Let $Z=\{f \neq g\}$, $|Z|=0$. $\{f>a\} \sm Z \subset \{g>a\} \subset \{f>a\} \cup Z$ but $\{f>a\} \sm Z$ and $\{f>a\} \cup Z$ have the same measure. 

Ex: 1/x is measurable on $\R$. Defined a.e. is okay. 


5. iff $f$, then $f^{-1}(G)$ is measurable for all open $G \subset \R$.

(5b) also for all borel sets $G \subset \R$.

\pf $G= \cup (a_k,b_k)$ so $f^{-1}(G)= \cup_k \{a_k<f<b_k\}$, measurable.

\pf (5b) a homework generalization. 

For all $\sigma$-algebra $M$, $\{E \colon f^{-1}(E) \in M\}$, where $M$ Lebesgue $\sigma$-algebra, like $M_f$. By 5, $M_f$ contains open sets, so it must contain Borel sets.

(5c) if $f$ is finite a.e. and $f^{-1}(G)$ is measurable for all open $G \subset \R$, $\{f>a\}= f^{-1}((a,\infty))$.


Warning: the composition of measurable functions is not always measurable. 


Recall $C$ cantor set, $F$ fat cantor set, $f(C)=F$, $f$ homeomorphism $\R \to \R$. Let $W \subset F$ be nonmeasurable. Write $\chi_W$ as a composition $\chi_{g(W)} \circ g$

% image

$\chi_{g(W)}$ is measurable, $g$ is continuous (so measurable).

4. is what we gain for going Borel -> Lebesgue and this is the price we pay.

Rem: if $f,g$ are Borel measurable, then $f \circ g$ is Borel measurable if for all $B \subset \R$ Borel we have $f^{-1}(B)$ is Borel, then $(f \circ g)^{-1}(B)= g^{-1}(f^{-1}(B))$, each composition on right Borel. 



%%%%%%%




	\begin{table}[h]
	\centering 
	\begin{tabular}{|c|c|c|c|} \hline
	\diagbox{is}{(*)} & Open & Borel & Measurable \\ \hline
	Open & Continuous & --- & --- \\ \hline
	Borel & Borel measurable & Borel measurable & --- \\ \hline
	Measurable & Measurable & Measurable & ? \\ \hline
	\end{tabular}
	\end{table}

% Where (*) is ``preimage of\dots''. 


When can we say that $f \circ g$ is measurable? What data about $f$ and $g$ would ensure this? 
We know that if $f$ is measurable and $g$ is continuous does not work from previous discussion. One possibility ensuring this would be $f$ is continuous and $g$ is measurable as $(f \circ g)^{-1}(U)= g^{-1}(f^{-1}(U))$, as $U$ is open $f^{-1}(U)$ is open and $g^{-1}$ of open is measurable (assuming $g$ takes values in $\R$ and $f$ is continuous on some open set containing the range of $g$). More generally, this will hold if $f$ is Borel measurable on some Borel set containing the range of $g$ and $g$ measurable and real-valued. 


\begin{ex}
If $f$ is measurable (and $\R$-valued) then $f^2$ is measurable, $f^2= \phi \circ f$, where $\phi(t)=t^2$. Similarly, $|f|$ is measurable since $|f|= \phi \circ f$, where $\phi(t)=|t|$. Also, $\sgn(f) = \begin{cases} 1, & f=0 \\ 0, & f=0 \\ -1, & f<0 \end{cases}$ is measurable since sign is Borel measurable (simple to see by checking preimages). So is floor(x), where floor is supremum of integers which are at most x. As is ceilin(x), defined mutatis mutandis. [They are both monotone,]

%Provide graphs
\end{ex}


\begin{thm}
If $f,g$ are measurable, then $f \pm g$, $fg$, and $f/g$ ($g \neq 0$ a.e.) are measurable.
\end{thm}

\pf $\{f+g>a\}= \bigcup_{b+c>a, b,c \in \Q} \left[ \{f>b\} \cap \{g>c\} \right]$ hence measurable set (countable union of measurable sets). Just need to prove equality. $\supset$ is obvious. Need only prove $\subset$. Suppose $f(x)+g(x)>a$. Let $\ep= f(x)+g(x)-a$. Use density of rationals to find $b \in \Q$ such that $f(x)- \ep/2<b<f(x)$ and $c \in \Q$ such that $g(x) - \ep/2<c<g(x)$. But then $b+c>f(x)+g(x)0\ep=a$. 

Now $fg= \dfrac{1}{4} \left[ (f+g)^2 - (f-g)^2 \right]$ is then measurable. For quotient only need reciprocal. We know $\frac{1}{g}$ is $\phi \circ g$, where $\phi(t)=\frac{1}{t}$, which is continuous on $\R \setminus \{0\}$. Now $g$ is nonzero almost everywhere, $\phi$ contains range of $g$ (ignoring set where range is $g=0$, since we can ignore for measurability). \qed \\


% f^g= e^{g\log f}$, so long as $f$ positive. Then measurability of this simple. 

Another way could combine is max of functions. Better to generalize to countable. So countable sup/inf of measurable functions is measurable.

\begin{thm}
The above statement, including infinite values. 
\end{thm}

\pf IF $f_k: E \to \overline{R}$ is measurable for all $k \in \N$, then $\sup_k f_k$ and $\inf_k$ are measurable. So $g(x)= \sup_{k \in \N} f_k(x)$ is measurable. Same for inf. 


	\[
	\begin{split}
	\{g>a\}&= \bigcup_{k \in \N} \{f_k>a\} \\
	\{g \geq a\}& \\
	\{g<a\}& \\
	\{g \leq a\}& \\
	\end{split}
	\]
First case done. Other follow similarly. HAND WAVE! For inf
	\[
	\{ \inf_k f_k<a\}= \bigcup_{k \in \N} \{f_k<a\}.
	\]
\qed \\





\begin{thm}
Suppose $f_k: E \to \overline{R}$ are measurable, then
%	\[
%	\begin{split}
%	\limsup_{k \to \infty} f_k&:= \inf_{m \in \N} \sup_{k \geq m} f_k \\
%	\liminf_{k \to \infty} f_k:= \sup_{m \in \N} \inf_{\k \geq m } f_k
%	\end{split}
%	\]
are measurable. 
\end{thm}

\pf Follows immediately from preceding result. [lim$=$ limsup$=$liminf when it exists.] \qed \\









Approximation: Given measurable $f$, find `simple' functions $f_k$ such that $f_k \to f$ pointwise. Here by simple we mean finite range. 

\begin{dfn}[Simple]
$g$ is simple if its range is finite subset of $\R$. [Need measurable?]
\end{dfn}


Need this to build notion of integration. 

\begin{thm}
For all measurable functions, there exists $f_k$ such that $f_k \to f$ pointwise. If $f \geq 0$, we can take $f_1 \leq f_2 \leq \cdots$.
\end{thm}

\pf $2^{-k} floor(2^k f)$. (floor to $\dfrac{integer}{2^k}$). This converges monotonically since increasing $k$ shortens the rounding distance, and then not as much difference so this converges to $f$ uniformly. Range not finite but countable. So just need to distribute the values among many functions. 
	\[
	f_k(x)=
	\begin{cases}
	\dfrac{j-1}{2^k}, & \dfrac{j-1}{2^k} \leq f(x) < \dfrac{j}{2^k} \\
	k, & f(x) \geq k
	\end{cases}
	\]
$f_k$ simple and $f_k \to f$. Now $f_k= \min(k,\max(-k,2^{-k}floor(2^kf)))$. $f \geq 0$, then $f_1 \leq f_2 \leq \cdots$. 
\qed \\



% Give some examples with plots, usually not continuous. 





\begin{dfn}
$f: E \to \R$ is continuous at $a \in E$ if for all $\ep>0$, there exists $\delta>0$ such that if $x \in E$ and $|x-a|<\delta$, then $|f(x)-f(a)|<\ep$.
\end{dfn}

Why better than normal definition with limit? Because $a$ can be a limit point not not defined there. Take $f(x)=1/x^2$, rather like allowing $f(0)=\infty$. 


Semincontinuity means rather than bounding difference from above below, means one of inequality $-\ep<f(x)-f(a)<\ep$ holds. 




\begin{dfn}[Upper Semicontinuous (usc)]
$f: E \to \overline{\R}$ is upper semicontinuous at $a \in E$ if for all $y>f(a)$, there exists $\delta>0$ such that if $|x-a|<\delta$, then $f(x)<y$. 
\end{dfn}

% red horizontal line, y, passing through line coming from solid point, below open circle then horizontal line down left, value a below holes.

% If a horzitaonl line above a point on graph, its above its neightborhood.

% Now same image with roles of lines switches and horizontal line passing through the middle. 

\begin{dfn}[Lower Semicontinuous (lsc)]
$f: E \to \overline{\R}$ is upper semicontinuous at $a \in E$ if for all $y>f(a)$, there exists $\delta>0$ such that if $|x-a|<\delta$, then $f(x)>y$. 
\end{dfn}

\begin{ex}
	\[
	f(x)=
	\begin{cases}
	\dfrac{1}{x^2}, & x \neq 0 \\
	\infty, & x=0
	\end{cases}
	\]
% give graph 
This is usc (automatically because of the $\infty$ value, why?) and lsc. 
\end{ex}

Both types of these functions are measurable. 



Remark: lsc -> measurable
usc-> measurable

\pf if $f$ usc on measurable $E$, then $\{f>a\}$ is open in $E$. So $\{f>a\}=E \cap U$. Hence $\{f>a\}$ is measurable.

For USC, use $\{f<a\}$.


% line, hole, point above it, USC. `USC kicking a field goal.' 

% Book: f usc/lsc/cts relative to E means f|_E is usc/lsc/cnt on E. Illstrate difference between ??

% Ex: f=\chi_\Q. Not continuous anywhere (at any point). But restricting to rationals is continuous. So book says relative to \Q. 



BV function

Given $f: [a,b] \to \R$, define $V(f; a,b)=\sup_p \sum_{i=1}^n |f(x_i)-f(x_{i-1})|$ sup over all partitions $P:a=x_0<x_1<\cdots<x_n=b$. $V(x^2; -1,1)=2$ (P=\{-1,0,1\} gives the maximum value. ) 

% Give plot.

\begin{dfn}
$f$ is BV on $[a,b]$ if $V(f; a,b)<\infty$
\end{dfn}

Interesting not called finite variation. 


Something which is not BV is 
	\[
	f(x)=
	\begin{cases}
	\sin(1/x), & x \in (0,1] \\
	0, & x=0
	\end{cases}
	\]
% give plot
% Choose partition where only +/- 1. So valuation at least +2+2+2+.... So clearly not BV. 


% Property: if $f$ is monotone, $V(f;a,b)= |f(a)-f(b)|$ (telescopes).
%2. V MONOTONE AND ADDITIVE WRT intervals: if $[a,b] \subset [c,d]$, then $V(f;a,b) \leq V(f; c,d)$. 
% if a \leq b \leq c, then V(f;a,c)= V(f; a,b) + V(f; b,c).  proof same as for riemann integral.

\begin{thm}
If $f$ is continuous on $[a,b]$ and differentiable on $(a,b)$ and $\int_a^b |f'| \; dx$ (riemann integral), then $V(f;a,b)= \int_a^b |f'| \; dx$.
\end{thm}

C.f. examples above. 


\pf Mean Value Theorem. 
	\[
	\sum_i |f(x_i) - f(x_{i-1})| = \sum_i |f'(\xi_i)| (x_i-x_{i-1})
	\]
$x_{i-1}< \xi_i < x_i$. the above converges to integral. \qed \\

3. Subadditive wrt $f$:
$V(f+g; a,b) \leq V(f; a,b) + V(g; a,b)$. 

\pf Triangle inequality.
	\[
	[f(x_i) + g(x_i)] - (f(x_{i-1}) + g(x_{i-1})) |
	\]
then use triangle splits into two and done. 

Why not exactly like integral, i.e. strict inequality. Take nonconstant function $f$ and then $g= -f$. Note $f+g$ identically zero. 




Sum of monotone functions are bounded variation. 



\begin{thm}[BV Decomposition]
If $f$ is BV on $[a,b]$, then there exists increasing $g$, $h$ on $[a,b]$ such that $f= g-h$. 
\end{thm}

\pf Let $g(x)= V(f; a,x)$, increasing (mononticity wrt interval). This forces $h(x):= g(x) - f(x)$. Why is $h(x) \leq h(y)$ for $x \leq y$? $V(f;a,x) - f(x) \leq V(f; a,y) - f(y)$. Why? $f(y)-f(x) \leq V(f; a,y) - V(f; a,x)= V(f;x,y)$, clearly this right side is $ \geq |f(x)-f(y)|$. \qed \\






























% Now near end of Ch. 4


% Reminder: unf conv: \sup|f_k -f| \to 0

% Ex: f_k(x)=x^2$ on $[0,1]$ f(x)=0 for all x. 
% here E(\ep)= [0,1-\ep] works. 
% sup_{E(\ep)} |f_k-f|= (1-\ep)^k \to 0
% |E|<\infty is important. Give picture with moving e^{-x^2}, aka moving mound. Converges to 0 but not uniformly. Only have uniform convergce on (-\infty,a]. 
% Need f defined on \R otherwise uniform makes no sense, i.e. no infty in domain. Take horizontal lines at y=n, i.e. f_k=k. Then makes no sense. 


\begin{thm}[Egorov's Theorem]
Suppose $\{f_k\}$ is a collection of measurable functions, $f_k: E \to \R$, $|E|<\infty$, and $f_k \to f$ a.e., where $f: E \to \R$. Then for all $\ep>0$, there exists a closed set $E(\ep) \subset E$ such that $f_k \to f$ uniformly on $E(\ep)$ and $|E \setminus E(\ep)|<\ep$. 
\end{thm}

\pf We need to find a set on which $\{f_k\}$ converges uniformly. It suffices to find, for each $j \in \N$, $E_j \subset E$ such that $|E \setminus E_j|< \ep/2^j$, and $\sup_{E_j} |f-f_k| \leq 1/j$ for sufficiently large $k$. Indeed, letting $E(\ep):= \bigcap_j E_j$, then $f_k \to f$ uniformly on $E(\ep)$, and $|E \setminus E(\ep)| \leq \sum_j \ep/2^j<\ep$. 

To find $E_j$, consider the set $G_m:=\{ x \colon |f_k(x)-f(x)| <1/j \text{ for all } k \geq m\}$. Since $|f_k-f| \to 0$ a.e., $\bigcup_m G_m$ is almost $E$, i.e. $\ms{\bigcup G_m}= |E|$. By continuity of measure for nested unions, $|G_m| \to |E|$. Therefore, there exists $m$ such that $|E \setminus G_m|< \ep/2^j$. Defining $E_j:= G_m$ completes the proof. \qed \\


% `Closed' comes for free: if we find any measurable E(\ep), then it will have a closed subset of almost the same measure. 

% Common tool is introducing sets such as the G_m's. See this in topology by using nested sequences of compact sets and concluding about intersections. 

% where is measure of finite? the |E\setminus G_m| < \ep/2^j., do not want to subtract infinities. 



\begin{thm}[Lusin's Theorem]
Let $E$ be a measurable set. A function $f: E \to \R$ is measurable if and only if for all $\ep>0$, there exists a closed set $E(\ep) \subset E$ such that $f\big|_{E(\ep)}$ is continuous, and $|E\setminus E(\ep)|<\ep$.
\end{thm}

\pf Suppose that $f$ is a simple function, i.e. $f= \sum_{k=1}^n a_k \chi_{E_k}$, where $E_1,\ldots,E_n$ are disjoint measurable sets. Choose a closed subset $E_k' \subset E_k$ such that $|E_k \setminus E_k'| < \ep/n$. Then $E(\ep):= \bigcup_{k=1}^n E_k'$ so that $|E\setminus E(\ep)|<\ep$. It only remains to check that $f\big|_{E(\ep)}$ is continuous, but this is immediate as the preimage of any set is closed.  

% Cross shape, each labeled E_1...E_4. Then small dotted around inside labeled E_k'

Now suppose only that $|E|<\infty$. There is a collection of simple functions $\{f_k\}$ such that $f_k \to f$ pointwise. Choose some $E(\ep/2^k)$ for $f_k$ as in the case above. By Egorov's Theorem, $f_k \to f$ uniformly on a set $E(\ep)$. Now $f$ is continuous on $E(\ep) \cap \left(\bigcap_k E_k(\ep/2^k)\right)$, which has compliment of measure at most $\ep+ \sum_k \ep/2^k=2\ep$. 

% Picture of nested shells. 

Now suppose that $|E|=\infty$. Write $E= \bigcup \left[ E \cap \{ |j| \leq |x| \leq j+1\}\right]$. In each shell, we have $E_j(\ep/2^j)$, closed set such that $f\big|_{E_j(\ep/2^j)}$ is continuous. Note that $f$ continuous on $\bigcup_j E_j(\ep/2^j)$. \qed \\




The converse direction of Lusin's theorem is simple since $f\big|_{E(1/j)}$ is continuous, then we have $f$ Borel on $\bigcup E(1/j)$ simply because $\{x \in \bigcup E(1/j) \colon f>a \}= \bigcup \{x \in E(1/j) \colon f>a\}$, countable union of Borel sets. and the rest has measure 0. 






% Ex:  This does not imply $f$ is continuous - does not have to be anywehre, take $f= \chi_\Q$. Let E(\ep)= \Q^C, then f\big|_{E(\ep)} is continuous \equiv 0.

% Ex:  if f= \chi_C, where C is the cantor set. Take E(\ep)= \R \setminus C

% Ex: f= \chi_F, F fat cantor set. If $\ep<|F|$, we must keep some $x \in F$. then must remove $\{x \colon |x-x_0|<\delta\} \setminus F, which has positive measure. 




% 09/24

% Quiz: Suppose $f_k: [0,1] \to \R$ are simple functions which converge uniformly to $f$. Is $f$ a simple function? 

% False: Take $f$ to be $y=x$ on $[0,1]$. Take step approximations to $f$. This converges uniformly but $f$ is not simple. 


% Section 4.4
% ------------------------
%  Convergence in Measure
% ------------------------
\subsection{Convergence in Measure}

Recall $\limsup E_k:= \bigcap_{m=1}^\infty \bigcup_{k=m}^\infty E_k= \{ x \colon x \in E_k, \text{ infinitely many }k\}$. We will not need $\liminf$ quite as regularly as $\limsup$.


% HW 1: If $\sum_k |E_k|<\infty$, then $|\limsup E_k|=0$. 

\begin{dfn}[Convergence in Measure]
Let $\{f_k\}$ be a collection of measurable functions, $f_k: E \to \R$. We say $f_k$ converges to $f$ in measure if for all $\ep>0$, $|\{ |f-f_k|>\ep\}| \to 0$. We denote this by $f_k \ma{m} f$. 
\end{dfn}


If we expand this definition out more, we have the following: $f_k \ma{m} f$ if for $\ep>0$ and for all $\ep_2>0$, there exists $N$ such that $|\{ |f-f_k|>\ep\}|<\ep_2$ for all $k \geq N$. If this is possible, one may as well replace the double epsilon with a single epsilon for a more concise version. So the definition could be taken as follows: $f_k \ma{m} f$ if for all $\ep>0$, there exists $N$ such that $|\{ |f_k-f|>\ep\}|<\ep$ for all $k \geq N$. 


\begin{ex}
$f_k(x)= \dfrac{1}{kx} \ma{m} 0$ on $\R$.

% Give plot of function. Horizontal lines at $\pm \ep$. 
	\[
	\begin{tikzpicture}[scale=1.5,every node/.style={scale=0.5}]
	\begin{axis}[
%	grid=both,
	axis lines=middle,
%	ticklabel style={fill=blue!5!white},
%	xmin= -7, xmax=7,
%	ymin= -6.5, ymax=6.5,
%	xtick={-6,-4,-2,0,2,4,6},
%	ytick={-6,-4,-2,0,2,4,6},
%	minor tick = {-5,-3,...,5},
	xlabel=\(x\),ylabel=\(\;y\)
	]

	\addplot[thick, domain= -6:-0.1,samples=100] {1/x};
	\addplot[thick, domain= 0.1:6,samples=100] {1/x};

	\end{axis}
	\end{tikzpicture}
	\]
We know that $\{ |f_k|>\ep\}= \{ |x|< \frac{1}{k \ep}\}$. This set has measure $2/(k\ep$, which tends to zero as $k$ tends to infinity. \xqed
\end{ex}


\begin{ex}
$f_k(x) = \frac{x}{k} \not\ma{m} 0$ on $\R$. $\{ |f_k|>\ep\}= \{ \dfrac{|x|}{k}>\ep\} = \{ |x|> k\ep\}$, which has infinite measure. \xqed
\end{ex}


\begin{ex}
Convergence in measure does not imply pointwise convergence. Let $f_k= \chi_{I_k}$, where $I_k$ are intervals.

[0,1] measure 1
[0,1/2], [1/2,1] measure 1/2
[0,1/3], [1/3,2/3], [2/3,1] measure 1/3

For all $\ep>0$, $\{ |f_k|>\ep\} < | I_k| \to 0$. 

So the sequence converges to 0 in measure. However, it clearly does not converge to 0 in measure.

There are infinitely many times point part of the interval. That is for all $x \in (0,1)$, there are infinitely many $k$ such that $x \in I_k$ so that $f_k(x) \not\to 0$. This can be expressed as saying $\limsup I_k= [0,1]$. 
\end{ex}


\begin{thm}
If $|E|<\infty$, then convergence a.e. implies convergence in measure.
\end{thm}

\pf Given $\ep>0$, Egorov's Theorem gives $E(\ep)$ such that $|E \setminus E(\ep)|<\ep$, and $f_k \to f$ uniformly on $E(\ep)$. Then there exists $N$ such that $|f_k-f|<\ep$ on $E(\ep)$ for all $k \geq N$. But then $|\{ |f_k-f|>\ep\}|<\ep$, which implies convergence in measure. \qed \\


\begin{thm}
If $f_k \ma{m} f$, then there is a subsequence $\{f_{k_j}\}$ converging to $f$ a.e.. 
\end{thm}

\pf Choose $\ep=1/2^j$. Find $N_j$ such that $|\{ |f_k-f|>1/2^j\}|<1/2^j$ for all $k \geq N$ (can make $N_{j+1}>N_j$). We have $\{f_{N_j}\}$. We claim that $f_{N_j} \to f$ a.e.. Let $E_j= \{ |f_{N_j}-f|>1/2^j\}$. Since $|E_j|<1/2^j$, we know $\sum |E_j|<\infty$. But then by HW 1, we know $|\limsup E_j|=0$. For all $x \notin \limsup E_j$, we have $x \in E_j$ only finitely many times. Then for sufficiently large $j$, $x \notin E_j$, say $x \notin E_j$ for $j \geq j_0$. But then $|f_{N_j}(x) - f(x)|<1/2^j \to 0$. \qed \\


\begin{rem}
``Triangle Inequality'' (quasi). Convergence in measure is not like convergence in metric space, though it is tempting to think of it in this way. We do have a replacement for a parallel triangle inequality. If $\{ |f-g|>\ep/2\}|<\ep/2$, and $|\{|g-h|>\ep/2\}|<\ep/2$, then $|\{ |f-h|>\ep\}|<\ep$. The proof is routine: $|f-h| \leq |f-g| + |g-h|$, so $|f-h|>\ep$ then one of the right side must be at least $\ep/2$. Then $\{|f-h|>\ep\} \subset \{|f-g|>\ep/2\} \cup \{|g-h|>\ep/2\}$. Their measures being at most $\ep/2$, the result then follows. 
\end{rem}


\begin{dfn}[Cauchy in Measure]
A collection of measurable functions $\{f_k\}$ is said to be Cauchy in measure if for all $\ep>0$, there exists $N$ such that $|\{ |f_j-f_k|>\ep\}|<\ep$ for all $k,j \geq N$. 
\end{dfn}


As it turns out, this is equivalent to convergence in measure.


\begin{thm}
Cauchy in measure is equivalent to convergence in measure. 
\end{thm}

\pfsk $\leftarrow$ is immediate by ``triangle inequality'' measure remark. Now the other direction. There are three standard steps that apply in these types of arguments. The strategy being the important piece here. Step 1. Enough to obtain a convergent subsequence: $f_1,f_2,f_3,\ldots$ cauchy sequence of functions with convergent subsequence. Use triangle inequality to show must converge to limit of conv subsequence. Conclude $f_k \to f$. Notice this holds for any type of convergence in any space. Step 2: Every Cauchy sequence has a ``geometric'' subsequence. Meaning $|\{ |f_{k_j} - f_{k_{j+1}}|>1/2^j\}|<1/2^j$. This is true for any metric space and any convergence. Use $\ep=1/2^j$, and get $k_j$. Step 3. Geometric trick implies convergent: given $\ep>0$, let $m$ be such that $1/2^{m-1}<\ep$. Let $\mathcal{A}= \bigcup_{j \geq m} \{ |f_{k_j} - f_{k_{j+1}}|>1/2^j\}$. Note $|\mathcal{A}|< \sum_{j=m}^\infty 1/2^j = 1/2^{m-1}<\ep$. On $E \setminus A$, we have $|f_{k_j} - f_{k_{j+1}}| \leq 1/2^j$ for all $j \geq m$. But this implies uniform convergence on the set $E \setminus A$. But then difference everywhere at most $\ep$ on $E \setminus A$ for sufficiently large $j$. Therefore, $|\{ |f_{k_j} - f|>\ep\}| \leq |A|<\ep$. \qed \\ 







% Section 5.1

% QUIZ
% Define $f_l: [0,\infty) \to \R$ by $f_k(x)= e^{-kx}$. We claim $f_k \ma{m} 0$ as $k \to \infty$. 
% True: Yes, find $x_0 = \log(1/\ep)/k$. Can also find areas etc etc. 

Two approaches to $\int_E f$, $f \geq 0$ on $E$.

First, $\int_E f= \sup \{ \int_E g \colon g \text{ simple } g \leq f\}$, $g = \sum_{k=1}^m a_k \chi_{E_k}$, where $E_k$ are disjoint, measurable sets. Here $\int_E g = \sum a_k |E_k|$.

Second, $\int_E = |R(f,E)|$, where $R(f,E)= \{ (x,y) \in \R^{n+1} \colon x \in E, 0 \leq y \leq f(x)\}$. 

First approach is common in many texts, we shall take the second. 

So $\int_E f$ exists if and only if $R(f,E)$ is measurable. 


% plot, curve line, shaded underneath to x-axis, horizontal shaded x-axis part labeled E and shaded region labeled R(f,E).

Some advantages of the second approach are monotonicity and countable additivity over $E$: so $0 \leq f \leq g$ and $\int_E f$, $\int_E g$ exists, then $\int_E f \leq \int_E g$ because $R(f,E) < R(g,E)$. 


Also  if $E= \bigcup_{k \in \N} E_k$, disjoint measurable sets, and $\int_{E_k} f$ exists, then $\int_E f$ exists and $\int_E f= \sum_{k \in \N} \int_{E_k} f$ because $R(f,E)= \bigcup R(f,E_k)$, the union taken over the disjoint, measurable sets. Can show in other definition but takes work due to $\sup$, whereas here we get it immediately. But we have the issue of existence and measurability. 

The following main theorem. 


\begin{thm}
If $fL E \to [0,\infty]$ is measurable, then $\int_E f$ exists (possibility $\infty$). 
\end{thm}



Idea of proof is from 4.1, $f= \lim f_k$, $f_1 \leq f_2 \leq \cdots$ are simple. So we will show $R(f_,E)$ are measurable and take their union. Need Lemma 1, $R(f_k,E)$ is measurable. Lemma 2 $\Gamma(f,E)$, which needs to be defined (simply the graph $\{(x,f(x) \colon x \in E\}$) has measure 0. This is boundary top of graph of shaded area of graph, we care because $\bigcup R(f_k,E)$ need not generally cover $R(f,E)$. It will surely cover $R(f,E) \setminus \Gamma(f,E)$, then $y<f(x)$, so there exists $k$ such that $y<f_k(x)$ so that $(x,y) \in R(f_k,E)$. 




Both lemmas rely on 

\begin{lem}
If $E \subseteq \R^n$ is measurable, then $E_A:= E \times [a,b]$ is measurable in $\R^{n+1}$ with measure $|E_a|= |E| \cdot a$, noting that $0 \cdot \infty=0$ in measure theory. 
\end{lem}

\pf If $E$ is an interval, follows by volume formula. If $E$ is general, then tile it by intervals, half open (so intervals disjoint). 

If $E$ is $G_\delta$ of finite measure, write $E= \bigcap G_k$, $G_1 \supset \cdots$ are open. 
	\[
	E_a = \bigcap_k (G_k \times [0,a]) \text{ so } |E_a|= \lim |G_k \times [0,a]|= a \lim |G_k| = a \cdot |E|.
	\]
What about general set? $|E|<\infty$, $E= H \setminus Z$, $M$ is $G_\delta$, $|M|=|E|$, $|H_a|=|H| \cdot a = |E| \cdot a$. So $|E_a| \leq |E| \cdot a$. 


Also for all $\ep>0$, there exists closed $F \subset E$ (hence $G_\delta$), $|F|>|E|-\ep$. 
	\[
	|F_a|= a |F|> a|E|- a\ep
	\]
So $E_a$ contains $F_a$ closed of measure $>|E_a|_e - a\ep$. Hence $E_a$ is measurable. $|E_a|= a|E|$. 

Still assumes $a<\infty$, otherwise $a=k$ and let $k \to \infty$ \qed? \\





\begin{lem} \label{lem:1}
$R(f_k,E)= \bigcup_j (E_j \times [0,a_j])$, $E_j$ disjoint measurable sets, where $f_k = \sum_j a_j \chi_{E_j}$. 
\end{lem}



Note that this shows that if $f$ takes countably many values, $f=a_k$ on $E_k$, then $\int_E f = \sum a_k |E_k|$. 




\begin{lem} \label{lem:2}
$|\Gamma(f,E)|=0$. Let $g= 2^{-j} floor(2^jf)$, then $g \leq f \leq g+ 2^{-j}$. Then $R(g,E) \subset R(f,E) \subset $. ???? Then $\Gamma(f,E) \subset R(g+2^{-j}, E) \setminus R(g-2^{-j},E)$. This right side has measure at most $2\cdot 2^{-j} |E| \to 0$, hence $|\Gamma(f,E)|=0$. 
\end{lem}






% Missing Lecture
% Friday 09/28


% Quiz $\int_E \chi_F= |E \cap F|$, where $E,F$ are measurable. 
% True: Evalauate over simple functions

Last time: $\int_E f= |R(f,E)|$, $R(f,E)= \{ (x,y) \in \R^{n+1} \colon 0 \leq y \leq f(x), x \in E\}$. $\Gamma(f,E)= \{ (x,f(x)) \in \R^{n+1} \colon x \in E\}$. 

$\int_E f= \sup \{ \int g \colon g \leq f, g \text{ simple}\}$, $g= \sum a_k \chi_{E_k} \to \sum g= \sum a_k |E_k|$, then $R(f,E)= (E \cap F) \times [0,1]$. 

Recall $R(f,E)$ is approximated by $R(f_k,E)$:

$\cup_k R(f_k,E) \subset R(f,E) \subset \left[ \cup_k R(f_k,E) \right] \cup \Gamma(f,E)$.

% Give curve with rectangles fit, Riemann integral 

Lemma: $|\Gamma(f,E)|=0$ for all measurable $f$.

\pf Write $E= \cup E_m$, $|E_m|<\infty$

Suffices $|\Gamma(f,E_m)|=0$ for all $|\ep>0$. Let $\cA_k=\{ x \in E_m \colon \ep k \leq f(x)< \ep(k+1)\}$, $k=0,1,2,\ldots$. Then $\Gamma(f,E_m) \subset \cup_k \cA_k + [\ep_k,\ep(k+1)]$, then $|\Gamma(f,E_m)| \leq \sum_k |\cA_k| \ep = \ep |E_m|$, then $|\Gamma(f,E_m)|=0$. 



\begin{thm}[Monotone Convergence Theorem]
If $f_k \geq 0$ are measurable on $E$ and $f_1 \leq f_2 \leq \cdots$, then $\int_E \lim f_k = \lim \int_E f_k$.
\end{thm}

\pf $f= \lim f_k$, then $R(f,E)= [\cup R(f_k,E)] \cup \Gamma(f,E)$ and $\cup R(f_k,E)$ are nested. So $|R(f,E)|= \lim |R(f_k,E)|$ by continuity of measure. \qed \\


\begin{ex}
$\Q=\{q_1,q_2,\ldots\}$. Let $f_k= \chi_{\{q_1,\ldots,q_k\}}$. Then $f_k \nearrow f= \chi_\Q$. Monotone convergence theorem fails for Riemann integral, so $\int_{\chi_\Q}= \lim \int f_k=0$. 
\end{ex}



\begin{thm}
The two definitions of $\int_E f$ agree:
	\[
	\int_E f= \sup\{ \sum a_k |E_k| \colon g \leq f, g= \sum a_k \chi_{E_k}\}
	\]
\end{thm}

\pf Note $\sum a_k |E_k|= \int_E g$ by measure of product. Since $f \geq g$, we get $\int f \geq \int g$ so $\geq$ holds. Reverse: pick $g_k \nearrow f$ pointwise. Then $\int g_k \to \int f$ so $\sup \int g_k \geq \int f$. \qed \\



\begin{ex}
$f_1 \geq f_2 \geq \cdots$. Counterexample $f_k= \chi_{[k,\infty)}$. $\lim \int f_k= \infty$ but $\int \lim f_k=0$. Similarly $\inf\{ \int g \colon g \geq f, g \text{ simple}\} \neq \int f$ in general. if $g \geq f$ then $g>0$ but being simple $g \geq c>0$ then $\int g= \infty$. 
\end{ex}


Chebshev's Inequality: Suppose $f$ is nonnegative measurable on $E$. For all $\alpha>0$ $|\{ x \in E \colon f(x)>\alpha \}| \leq \frac{1}{\alpha} \int_E f$.

\pf Let $A=\{f>\alpha\}$. Then $f \geq \alpha \chi_A \to \int f \geq \int \alpha \chi_A= \alpha |A| \to |A| \leq \frac{1}{\alpha} \int f$.


\begin{ex}
$f(x)= \dfrac{1}{\sqrt{x}}$ on $[0,1)$. Let $c= \int_{(0,1)} f \to$ Chevyshev $|\{f>\alpha\}| \leq \dfrac{c}{\alpha}$. In fact, $\{f>\alpha\}= (0,1/\alpha^2) \to$ measure is $1/\alpha^2$, if $\alpha \geq 1$. 
\end{ex}



Additivity over functions: $\int_E (f+g)= \int_E f + \int_E g$, $f,g \geq 0$ measurable. 

\pf Use simple fucntions, $f_k \nearrow f$, $g_k \nearrow g$. So it suffices to prove that $\int (f_k+g_k)= \int f_k + \int g_k$. $f_k= \sum a_j \chi_{E_j}$, $g_k = \sum b_k \chi_{F_k}$. Let $H= E_i \cap F_j$ note $\cup E_i= \cup F_i = E$. Then $f_k= \sum_{i,j} a_i \chi_{H_{i,j}}$, $g_k= \sum b_j \chi_{H_{ij}}$. Then $f_k+g_k= \sum (a_i+b_j) \chi_{H_{ij}}$ then $\int (f_k+g_k) = \sum (a_i+b_j) |H_{ij}|= \sum a_i |H_{ij}| + \sum b_j |H_{ij}|$. 

Countable subadditivity:

$\int \sum_{k=1}^\infty f_k = \sum_k \int_E f_k$, $f_k \geq 0$ measurable.

True for $\sum_{k=1}^N$ then $N \to \infty$ using MCT


\begin{ex}
Let $\Q \cap [0,1]= \{q_1,\ldots\}$

$f_k= \dfrac{1}{\sqrt{|x-q_k|}}$ $\int_{[0,1]} f \leq c < \infty$

$f= \sum_{k=1}^\infty \dfrac{1}{k^2} \; \dfrac{1}{\sqrt{|x-q_k|}} \ma{\lim_{x \to q} f(x)=\infty \text{ for all } q \in \Q} \int f \leq c \sum \dfrac{1}{k^2}< \infty$ so $|\{f>\alpha\}| \leq \dfrac{1}{\alpha}\, C'$, what does this say about $\Q$?
\end{ex}




%%%%%%%%%%%%%%%%%%%%%%%%%%%%%%%%%%%%%





% Last Time: $f \geq 0$ measurable: $\int_E f= |R(f,E)|$ or $\int_E f= \sup \{ \sum a_k |E_k| \colon \sum a_k \chi_{E_k} \leq f\}$. 


% Quiz: Suppose $f: [0,\infty) \to [0,\infty)$ is measurable. Claim $\int_{[0,\infty)} e^{-x/k} f(x) \; dx \to \int_{[0,\infty)} f(x) \; dx$ as $k \to \infty$. 

% True: $x/k \searrow 0$, $e^{-x/k} \nearrow 1$, $e^{-x/k} \geq 0$, converges increasing then Monotone Convergence Theorem.






We have not yet proved $\int_E cf = c \int_E f$, where $c \geq 0$ and $f \geq 0$.

\begin{prop}
\[ \int_E cf = c \int_E f \]
\end{prop}

\pf This is true for simple functions: $\int c (\sum a_k \chi_{E_k})= \int \sum c a_k \chi_{E_k} = \sum c a_k |E_k| = c \sum a_k |E_k| = c \int \sum a_k \chi_{E_k}$. Then for general functions use the Monotone Convergence Theorem. For general $f \geq 0$, take simple $f_k \geq 0$, $f_k \nearrow f$, then $c f_k \nearrow cf$, by MCT $\int cf= \lim cf_k = c \lim \int f_k = c \int f$. \qed \\


What if the convergence is not $f_k \nearrow f$? 

\begin{ex}
$\chi_{[k,\infty)} \to 0$ pointwise. But $\int \chi_{[k,\infty)}$ are all infinite and certainly do not converge to $0$. Note that $\chi_{[k,\infty)} \searrow 0$. 
\end{ex}


\begin{ex}
$k \chi_{(0,1/k)}$. But $\int k \chi_{(0,1/k)} =1 \not\to 0$. 

% Give shaded rectangle height k and width 1/k.
\end{ex}


\begin{ex}
$\frac{1}{k} \chi_{[0,k]} \to 0$ uniformly but $\int \frac{1}{k} \chi_{[0,k]}= 1 \to 0$. 

% Give similar image as above. 
\end{ex}


What is the similarity of these three examples? The functions vanish but their integrals do not???? The general principle is that the area under the graph can disappear or `escape' to infinity but does not appear from nowhere.......... A precise statement is Fatou's Lemma. As a reminder,


\begin{dfn}[limsup,liminf]
Given a sequence $\{f_k\}$,
	\[
	\begin{split}
	\limsup_{k \to \infty} f_k = \lim_{n \to \infty} \sup_{k \geq n} f_k
	\liminf_{k \to \infty} f_k = \lim_{n \to \infty} \int_{k \geq n} f_k.
	\end{split}
	\]
\end{dfn}


\begin{ex}
\begin{enumerate}[(i)]
\item $\lim f_k(x)$ exists if and only if $\limsup f_k(x)= \liminf f_k(x)$. 
\item The Piano Key Sequence (press one key going left to right): $\chi_{[0,1]}$, then $\chi_{[0,1/2]}, \chi_{[1/2,1]}$, then $\chi_{[0,1/3]}, \chi_{[1/3,2/3]}, \chi_{[2/3,1]}$. [Go in pascal triangle order]
	\[
	\begin{split}
	\limsup f_k&= \chi_{[0,1]} \\
	\liminf f_k&= 0
	\end{split}
	\]
\end{enumerate}
\end{ex}


\begin{rem}
Relation to sequences to sets $\{E_k\}$: $\limsup E_k= \bigcap_{m=1}^\infty \bigcup_{k \geq m} E_k$, $\liminf E_k= \bigcup_{m=1}^\infty \bigcap_{k \geq m} E_k$. $\limsup \chi_{E_k}= \chi_{\limsup E_k}$ and same for liminf. and sup like union and inf like intersection. 
\end{rem}


\begin{lem}[Fatou]
$f_k \geq 0$ measurable.
\[ \int_E \liminf f_k \leq \liminf \int_E f_k \]
\end{lem}

\pf Let $g_m= \inf_{k \geq m} f_k$. Then $g_m \nearrow \liminf f_k$. By MCT, $\int g_m \to \int \liminf f_k$. But $f_k \geq g_k$ so that $\int f_k \geq \int g_k$, hence $\liminf \int f_k \geq \lim \int g_k$. [ Rem: if $a_k \geq b_k$ for all $k$, then $\liminf a_k \geq \liminf b_k$, $\limsup a_k \geq \limsup b_k$.] \qed \\


\begin{thm}[Lebesgue Dominated Convergence (DCT)]
Suppose $f_k \geq 0$ are measurable and there is $\phi \geq 0$ measurable such that $f_k \geq \phi$ ($\phi$ dominates) and $\int_E \phi< \infty$. If $f_k$ converges to $f$ a.e., then $\int_E f_k \to \int_E f$. 
\end{thm}

\pf Fatou's Lemma gives $\int f \leq \liminf \int f_k$. Also, $\phi - f_k \geq 0$. Then Fatou's Lemma applies again: $\int \phi - f \leq \liminf \int (\phi - f_k)$. Which is $\int f \geq \limsup \int f_k$. Then $\int \phi - \int f \leq \liminf (\int \phi - \int f_k)$ cancel constant $-\int f \leq \liminf (-\int f_k)$ then $\int f \geq \limsup f_k$. Comparing very first and last equations, we have $\int f= \lim_{k \to \infty} \int f_k$. \qed \\





\begin{ex}
$\frac{1}{k} \chi_{[0,k]} \to 0$ uniformly. 

% Give plot: lots of rectangles growing to right, overlay them. plot with $1/x$.

Try to cover with one function get something like $1/x$. This is not good enough for DCT since $\int \frac{!}{x}= \infty$. If $f_k \leq \frac{1}{x^2}$ on $[1,\infty)$, this exmaples does not work. `Best' you can do is $\frac{1}{k^2} \chi_{[0,k]} \to 0= \int \lim f_k$. 
\end{ex}


\begin{ex}
$\lim \int_{[1,\infty)} \dfrac{\sin^{2k} x}{x^2} \; dx$. Now $\dfrac{\sin^{2k} x}{x^2} \to 0$ a.e. and is at most $\frac{1}{x^2}$ in absolute value, so must be 0. Now if $\lim \int_{[0,\infty)} \dfrac{\sin^{2k} x}{x^2} \; dx$, $\dfrac{\sin^{2k} x}{x^2} \leq \dfrac{\sin^2 x}{x^2}$ and $\int_{[0,1]} \dfrac{\sin^2 x}{x^2} \leq \int_{[0,1]} 1 \leq 1$. so $\int_{[0,\infty)} \dfrac{\sin^2 x}{x^2}< \infty$.
\end{ex}











%%%%%%%%%%%%%%%%%%%%%%%%%%%%%%%%%%%%%%%%%




% Not all, however, need such advanced techniques. 

% $\int_0^\infty \dfrac{\sin x}{x} \; dx$
% \int_1^\infty \sin(x^2) \; dx; u=x^2 \; du=2x \; dx

% Want to define integral genreally, 

Suppose that $f: E \to \ov{\R}$ is measurable. Define functions $f^+= \max(f,0)$ and $f^-=\max(-f,0)$. These are both nonnegative functions with $f^+ - f^-=f$ and $f^+ + f^-= |f|$. 


\begin{dfn}[Integrable Function]
Suppose $f: E \to \ov{\R}$ is a measurable function, not necessarily nonnegative. We define $\int_E f$ to be
	\[
	\int_E f:= \int_E f^+ - \int_E f^-,
	\]
provided at least one of the integrals on the right is finite. We say that $f$ is integrable on $E$ if $\int_E f$ is finite, and we write $f \in L^1(E)$, 
\end{dfn}


\begin{rem}
Note that $\int f$ is finite if and only if $\int f^+$ and $\int f^-$ are finite if and only if $\int |f|$ is finite. Therefore, all Lebesgue integrable functions with finite integral are automatically absolutely integrable. This means there are integrals which are (Improper) Riemann integrable which are not Lebesgue integrable. Take Example~\ref{}, $\int_{[0,\infty)} \dfrac{\sin x}{x} \; dx$ does not exist as a Lebesgue integrable, i.e. $\dfrac{\sin x}{x} \notin L^1(0,\infty)$, while it is Riemann integrable on $(0,\infty)$. 
\end{rem}


We next examine a few properties of this integral. Not surprisingly, many of these properties resemble those of their primordial Riemann integral. 

\begin{itemize}
\item Triangle Inequality: $\left| \int_E f \right| \leq \int_E |f|$ as $\left| \int f^+ - \int f^- \right| \leq \int f^+ + \int f^-$.

\item Additivity of Domain: If $\{E_k\}_{k=1}^N$ is a collection of disjoint, measurable sets, then $\int_{\cup E_k} f = \sum_{k=1}^m \int_{E_k} f$ as $\int f^+$, $\inf f^-$ split using the properties of the integral for nonnegative functions. 

However, care is needed even in the countable additivity case. Consider $\int_0^\infty \sin x$, defining the $E_k$ to be as below:

% Give picture. Alternate E_k the positive negative parts of sin x.

We know that $\int_{E_k} f= 0$ for all $k$ but that $\int_{\cup E_k} f$ does not exist. 
\end{itemize}


\begin{lem}
Suppose that $\int_E f$ exists. Let $E= \bigcup_{k=1}^\infty E_k$ be a union of disjoint, measurable sets. Then $\int_{E_k} f$ exists for all $k$, and 
	\[
	\int_E f= \sum_{k=1}^\infty \int_{E_k} f
	\]
\end{lem}

\pf By countable additivity for integrals of nonnegative functions, we know that $\int_E f^+= \sum_k \int_{E_k} f^+$ and $\int_E f^-= \sum_k \int_{E_k} f^-$. Now observe
	\[
	\int_E f = \sum_k \left( \int_{E_k} f^+ - \int_{E_k} f^- \right).
	\]
Since $\int_E f$ exists, at least one of $\int_E f^+$, $\int_E f^-$ is finite. 

% Go through cases and explain why sum exists. 

\qed \\




Additivity over $f$:

Suppose we have $\int_E (f+g)= \int_E f + \int_E g$, assuming that both integrals on the right are finite. Let $h:= f+g$. There are eight possibilities for the signs of $(f,g,h)$. This splits $E$ into eight different parts. On each part, write a between $f,g,h$ as addition of nonnegative functions, e.g. $f \geq 0$, $g<0$, $h<0$, then $f+g=h$ becomes $f+(-h)=(-g)$, all nonnegative here. Then
	\[
	\int f + \int (-h) = \int (-g)
	\]
Note that we are making use of the fact that the functions are finite a.e.. 



Remark: if $f \in L^1(E)$, then $|f|<\infty$ a.e., since $\int |f|<\infty$. 




Convergence: 

There is little to say about convergence since everything follows from convergence of nonnegative functions. 


For example, how does MCT apply? The answer is that it does not really apply. We need an integrable `baseline', $\phi \in L^1(E)$, i.e. a function to play the role of the $x$-axis. 

MCT for nonnegative functions: 

% Picture: Volume of `piles', have small growing arcs along a x-axis.
% Picture: Same idea except have \phi curvy line. Then there are f's with almost same shape growing closer to it. Could also go from below, have these labeled g's. 

If $\phi \leq f_1 \leq f_2 \leq \cdots$ or $\phi \geq f_1 \geq f_2 \geq \cdots$, then $\int f_k \to \int \lim f_k$. 

\pf Use MCT for $f_k - \phi$ or $\phi - f_k$, nonnegative, increasing. Then $\int (f_k - \phi) \to \int (f-\phi)$; hence, $\int f_k \to \int f$. \qed \\



However, there is one useful result here:

\begin{thm}[Dominated Convergence Theorem]
If $f_k \to f$ a.e. on $E$, and are measurable, and there exists $\phi \in L^1(E)$ such that $|f_k| \leq \phi$ for all $k$, then
	\[
	\int_E f_k \longrightarrow \int_E f.
	\]
\end{thm}

% Note these integrals are finite 

\pf Since $0 \leq \underbrace{f_k+\phi}_{\geq 0} \leq 2 \phi$, DCT for nonnegative functions applies. $\int (f_k+\phi) \to \int (f+\phi)$. \qed \\


\begin{ex}
Consider $\sum_{k=1}^\infty B_k \sin(kx)$ on the interval $(0,\pi)$. Assume that $C:=\sum |B_k|<\infty$. What do we need to assume about $B_k$ to obtain
	\[
	\int_{(0,\pi)} f(x) = \sum_k \int_{(0,\pi)} B_k \sin(kx)?
	\] 
Observe that $C$ is a domination function. $f_k(x) = \sum_{j=1}^k B_j \sin(jx)$ satisfies $|f_k| \leq \phi$. Also, $f_k \to f$ pointwise. 

% Similarly, we can use $B_k = \dfrac{1}{2} \int_{(0,\pi)} f(x) \sin(kx) \; dx$, a straightforward computation gives 
\end{ex}












% Last time: $\int_E (f+g) = \int_E f + \int_E g; f,g measurable and \int_E f and \int_E g finite

Shorter proof: let $h= f+g$:
$h^+ - h^-= f^+ - f^- + g^+ - g^-$
$h^+ + f^- + g^-= f^+ + g^+ + h^-$
Integrate, apply additivitivty for nonnegative.
$\int h^+ + \int f^- + \int g^-= \int f^+ + \int g^+ + \int h^-$
all integrals finite then 
$\int h^+ - \int h^- = \int f^+ - \int f^- + \int g^+ - \int g^-$. 



DCT: Works for continuous parameters.

If we have $f_t$ ($t$ real) and $|f_t| \leq \phi$, $\phi \in L^1(E)$ and $f_t$ measurable and $f_t \to f$ a.e. as $t \to t_0$. Then $\lim_{t \to t_0} \int_E f_t= \int f$. 

\pf Sequential characterization of limits: Take any sequence $t_k \to t_0$ and apply DCT to $f_{t_k}$. Get $\int f_{t_k} \to \int f$. \qed \\


\begin{ex}
Suppose $|E|<\infty$, $f \in L^1(E)$, $f>0$ on $E$.
a) $|int_E f^p \to |E|$ as $p \to 0^+$
b) if also $\log f \in l^1(E)$, then $\int_E \dfrac{f^p-1}{p} \to \int_E \log f$ as $p \to 0^+$.


apf) $f^p \to 1$ as $p \to 0^+$. Need dominating function. Consider only $0<p \leq 1$. Then $f^p \leq f$? if $f \geq 1$ but notice $f^p \leq 1$ if $0<f<1$. So $f^p \leq \max(f,1) \leq f+1$. This is simpler than the less mysterious max. So $f+1 \in L^1(E)$ dominating function. Hence, $\int_E f^p \to \int_E 1= |E|$. 

bpf) derivative of $p \mapsto e^{p \log f}$ at $p=0$ which is $\log f$. $\dfrac{f^p-1}{p} \to \log f$. Slope of secant 

% Plot axes e^x but label f^p, given secant from (0,f(0)) to (p,f(p))

Slope inc: $\dfrac{f^p-1}{p} \leq \dfrac{f^1-1}{1}= f-1$

$f \leq 1$: $\left\| \dfrac{f^p-1}{p} \right| \leq |\log f|$.

% Same plot but e^{-x}$.


% Continued in next class:
If $f<0$ is measurable and $f \in L^1(E)$, then
	\[
	\int_E \dfrac{f^p-1}{p} \to \int_E \log f \text{ as } p \to 0^+.
	\]
We need integrability to make this possible. 


MCT applies: 
if $\phi \in L^1$ is a `baseline' and $\phi \geq f_1 \geq f_2 \geq \cdots$ or 
$\phi \leq f_1 \leq f_2 \leq \cdots$, then $\int \lim f_k = \lim \int f_k$. 

For all $p_k \searrow 0$, we want to show $f_k= \dfrac{f^{p_k} - 1}{p_k}$ is a decreasing sequence, $f_1 \geq f_2 \geq f_3 \geq \cdots$. The reason why is because of the shape of these things. 

% Give same $f^p$ graph. Secant line graph. 

Now taking $p_k$ which is smaller, so $p_{k+1}$ is to left so the slope of the secant line is less because it is a convex function. The conclusion is the same for the decreasing version because this is also convex. `Baseline' is then $f-1$. 
\end{ex}









% Monday 10/08/2018
% Section 5.4 - 5.5
\subsection{Relation between the Lebesgue and Riemann-Stieltjes Integrals}


Recall most important HW (5.1\#1): if $f \geq 0$ measurable, then $f \in L^1$ if and only if $\sum_{j= -\infty}^\infty 2^j |\{ f>2^j\}| < \infty$.  Let $\omega_f= |\{ x \in E \colon f(x)>\alpha\}|$. Notice this is just notation for something we have already discussed at length. With this notation, we can say that
	\[
	f \in L^1(E) \Longleftrightarrow \sum_{j= -\infty}^\infty 2^j \omega_{|f|}(2^j)< \infty,
	\]
where $f$ is measurable but not necessarily nonnegative. What is the significance of the `2' here? The proof of the exercise only makes use of the fact that multiplication by 2 preserves the inequality: $g= 2^j \chi_{\{2^j<f<2^{j+1}\}}$ and then $g \leq f \leq 2g$. This is like the Cauchy Condensation Test. However, this works for any real number $\lambda>1$. This allows us to then characterize the above as 
	\[
	\begin{split}
	f \in L^1(E) &\Longleftrightarrow \sum_{j= -\infty}^\infty 2^j \omega_{|f|}(2^j)< \infty \\
	&\Longleftrightarrow \sum_{j= -\infty}^\infty \lambda^j \omega_{|f|}(\lambda^j)< \infty 
	\end{split}
	\]
for any $\lambda>1$. The generalization of this is as follows:

\begin{dfn}[$L^p$]
 $f: E \to \ov{\R}$ belongs to $L^p(E)$ if $\int |f|^p<\infty$, where $0<p<\infty$. 
\end{dfn}


By the discussion above (or \#1 of HW 5.1), we know that $f \in L^p$ if and only if $\sum \lambda^j |\{ |f|^p>\lambda^j \}|<\infty$, i.e. $\sum \lambda^j |\{ |f|> \lambda^{j/p} \}|<\infty$. If we choose $\lambda= 2^p$, then
	\[
	f^p \in L^p \Longleftrightarrow \sum_{j= -\infty}^\infty 2^{pj} |\{ f>2^j \}| < \infty. 
	\]

% Notice larger p, makes life easier/harder for convergence based on power and when positive or negative. and other similar observations. 


\begin{thm}
If $|E|<\infty$ and $f: (a,b]$ is measurable, where $a,b \in \R$, then 
	\[
	\int_E f = \int_a^b \alpha d(-\omega_f(\alpha)).
	\]
\end{thm}

\pf Consider $\mathcal{P}$: $a=x_0<x_1<\cdots<x_n=b$. Let $E_i= \{x_{i-1}< f \leq x_i \}$. Note $E_i$ is a partition of $E$.
	\[
	\sum_i x_{i-1} \chi_{E_i} \leq f \leq \sum_i x_i\chi_{E_i}.
	\]
But then 
	\[
	\sum_{i=1}^n x_{i-1} |E_i| \leq \int_E f \leq \sum_{i=1}n x_i |E_i| (*)
	\]
Here 
	\[
	|E_i|= -(\underbrace{\omega(x_i)}_{f>x_i} - \underbrace{\omega(x_{i-1})}_{f>x_{i-1}}
	\]
But then the left sum is $L(\alpha,P,-\omega)$ and the right one is $U(\alpha,P,-\omega)$. The sup of $L$ and inf of $U$ both go to $\int_a^b \alpha d(-\omega)$. Thus, $\int_E f = \int_a^b f \; d(-\omega)$. \qed \\


% Remind of definition of Riemann Stieltjes integral. Put at beginning? Notice that difference of \alpha function is a measure of (x_{i-1},x_i]. So we see the connection. 







\subsection{Relation between the Lebesgue and Riemann Integral}

We want to compare $\int_{[a,b]} f$ and $R\!\!\int_a^b f$, where $a,b \in \R$ and $f$ is bounded. Following shows we have generalized.


\begin{thm}
If $R\!\!\int_a^b f$ exists, then $\int_{[a,b]} f$ exists and they are equal.
\end{thm}

\pf 
Riemann then Leb: If $R$ integral exists, then there exists sequence of partitions $\{P_k\}$ such that $U(f,P_k)$ and $L(f,P_k)$ converge to $\int_a^b f$. Arrange $P_1 \subset P_2 \subset P_3 \subset \cdots$. Let $g_k, h_k$ be step functions for these partitions. $g_k \leq f \leq h_k$ and $L(f,P_k)= \int g_k$, $U(f,P_k)= \int h_k$. 

% Give plot of small curve with g_k, h_k upper/lower rectangles. Shade colors. 

By MCT, $g_1 \leq g_2 \leq \cdots$ and $h_1 \geq h_2 \geq \cdots$. Then $\int g_k \to \int g$, where $g= \lim g_k$ and $\int h_k \to \int h$, where $h= \lim h_k$. So $\int g = \int h = R\!\!\int_a^b f$ since $U$ and $L$ converge to $\int_a^b f$. But we know that $g \leq f \leq h$. This means that we must have equality a.e.. But this completes the proof. \qed \\






The following characterization of Riemann integrals.

\begin{thm}
Riemann integral exists if and only if $|\mathcal{D}|=0$, where $\mathcal{D}$ is the set of discontinuities of $f$. 
\end{thm}

\pf See text. 























