% !TEX root = ../../mat701_notes.tex
\newpage
\section{Abstract Integration}
\subsection{Measure Spaces}

So far, we only have had Lebesgue measure on $\R^n$. But there are other measures: length of curves in $\R^2$, surface area on surfaces, probability on probability space $\Omega$ $P(\text{event})=$ measure of outcomes. 

% Give weird closed curve
% Give picture of torus
% Give bulls eye in circle. 

\begin{dfn}[Measure Space]
A measure space is a triple, $(X,\Sigma,\mu)$, where $X$ is a set, $\Sigma$ is a $\sigma$-algebra on subsets of $X$, and $\mu: \Sigma \to [0,\infty]$ is a countably additive function on disjoint sets. 
\end{dfn}


\begin{rem}
One often also requires $\mu(\emptyset)=0$. However, this does almost follow from our definition since $\mu \geq 0$ and if $\emptyset \neq A \subseteq X$, then 
	\[
	\mu(A)=\mu(A \cup \emptyset)= \mu(A) + \mu(\emptyset)
	\]
so that $\mu(\emptyset)=0$ unless $\mu(\emptyset)=\infty$.  
\end{rem}

\begin{ex} \hfill
\begin{enumerate}[(i)]
\item $(\R^n,\mathcal{L},\mu)$, $\mathcal{L}$ is the the set of Lebesgue measurable set. $\mu(E)=|E|$.

\item $(X,2^X,\mu)$, $X$ is any set, $\mu(E)$ is the cardinality of the set. 
\item Like (ii) but
	\[
	\mu(E)=
	\begin{cases}
	0, & E \text{ countable} \\
	\infty, & E \text{ uncountable}
	\end{cases}
	\]

\item $(\R^n,L,\mu_w)$, where
	\[
	\mu_w(E)= \int_E w
	\]
where $w \geq 0$ is measurable. One may say that $w$ is the `weight' or `density.'  % Gaussian curve
\end{enumerate}
\end{ex}


\begin{nex}
like (ii) above but
	\[
	\mu(E)=
	\begin{cases}
	0, & |E|<\infty \\
	\infty, & |E| \text{ infinite}
	\end{cases}
	\]
this is \emph{not} a measure space since it is not countably additive. 
\end{nex}


\begin{dfn}[Additive Set Function (ASF)]
We say a function $\phi: \Sigma \to \R$, where $\Sigma$ is a $\sigma$-algebra, is countably additive. 
\end{dfn}


The difference between $\phi$ and a measure is that it can be negative and it cannot be infinite. This is also called a \emph{finite signed measure}. 


\begin{ex}
$\phi(E)= \int_E f$, where $f \in L^1$. 
\end{ex}

An advantage of ASF can pass to complement: $\phi(E^C)= \phi(X) - \phi(E)$ and get results for $\cap$ from $\cup$.


\begin{thm}
ASF is continuous on monotone sequences of sets, i.e. $E_1 \subset E_2 \subset \cdots$ then $\phi(\cup E_k)= \lim \phi(E_k)$ or $E_1 \supset E_2 \supset \cdots$ then $\phi(\cap E_k)= \lim \phi(E_k)$
\end{thm}

\pf The second follows from the first by taking complements. Now
	\[
	\bigcup_{k=1}^\infty E_k= E_1 \cup (E_2 \sm E_1) \cup (E_3 \sm E_2 \sm E_1) \cup \cdots
	\]
disjoint union so
	\[
	\phi(\cup E_k) = \phi(E_1) + (\phi(E_2) - \phi(E_1)) + \cdots
	\]
which telescopes so we have $\lim_{k \to \infty} \phi(E_k)$. \qed \\


For measures $\mu$. the first works but in the second statement above, we require $\mu(E_1)< \infty$.


\begin{thm}[`Fatou's Lemma']
if $\phi \geq 0$ is an ASF (i.e. finite measure), then $\phi(\liminf E_k) \leq \liminf \phi(E_k)$. Hence by considering $E^C_k$, $\phi(\limsup E_k) \geq \limsup \phi(E_k)$. 
\end{thm}

\pf Let $M_k= \bigcap_{j \geq k} E_j$, then $\liminf E_k= \bigcup_{k=1}^\infty H_k$.
	\[
	\phi \left( \bigcup H_k \right) = \lim_{k \to \infty} \phi(H_k)= \liminf_{k \to \infty} \phi(E_k),
	\]
using $H_k \subset E_k$, so $\phi(E_k) - \phi(H_k)= \phi(E_k \sm H_k) \geq 0$. 




Jordan Decomposition for ASF 

Every ASF is the difference of two nonnegative ASFs. 

Application: Write ASF $\phi = \phi_1 - \phi_2$, both nonnegative. Conclude if $E= \limsup E_k= \liminf E_k$, then $\phi(E)= \lim \phi(E_k)$. 


\begin{dfn}[Upper Variation]
Given an ASF $\phi$, upper variation $\overline{V}(E)= \sup_{A \subset E} \phi(A)$. 
\end{dfn}

\begin{dfn}[Lower Variation]
Given an ASF $\phi$, lower variation $\underline{V}(E)= - \inf_{A \subset E} \phi(A)$. 
\end{dfn}

\begin{dfn}[Total Variation]
Given an ASF $\phi$, total variation: $V(E)= \overline{V}(E) + \underline{V}(E)$.
\end{dfn}

How does this relate to Jordan decomposition? Idea of proof is to show $\phi(E)= \overline{V})(E) - \underline{V}(E)$. 

where (needs to be proved) $\overline{V}$, $\underline{V}$ are ASFs. Why this was
	\[
	\int_{A \subset E, A \in \Sigma} \phi(A) = \inf_{B \subset E, B \in \Sigma} (\phi(E) - \phi(B)) = \phi(E) - \sup_{B \subset E, B \in \Sigma} \phi(B) = \phi(E) - \overline{V}(E)
	\]
Take $A= E \sm B$. Hence, $\phi(E)= \overline{V}(E) - \underline{V}(E)$. 


Clear total variation is ASF if the upper/lower are. So sketch proof that they are:

1) Countable subadditivity. 
$A \subset \cup E_k$, write as $A= (A \cap E_1) \cup (A \cap E_2 \sm E_1) \cup \cdots$. Get $A= \cup A_k$, $A_k \subset E_k$. $\phi(A_k) \leq \overline{V}(E_k)$ hence $\phi(A) \leq \sum \overline{V} (E_k)$. The upper follows the same.

2) $V< \infty$, which is done by contradiction. 
$\overline{V}(E) \geq \sum \overline{V}(E_k)$ just pick $A_k \subset E_k$, $\phi(A_k) \geq \overline{V}(E_k) - \ep/2^k$, use countable subadditivity of $\phi$. 




%%%%%%%%%%%%%%%%%%%%%%


\subsection{Measurable Functions and Integrals}

Repeat all previous work for abstract measure spaces. Recall $(X, \Sigma, \mu)$ is a measure space. 

\begin{dfn}
A function $f: X \to \overline{\R}$ is measurable if $\{f>a\} \in \Sigma$ for all $a \in \R$. 
\end{dfn}

Fur functions $f: X \to \R$, equivalent to $f^{-1}(\text{Borel}) \in \Sigma$. 

\begin{ex}
Let $X= \N$, $\Sigma = 2^X$, then all functors are measurable.
\end{ex}

Basic properties: $cf$, $f+g$, $\sup_k f_k$, $\inf_k f_k$, $\lim_k f_k$ are measurable. 

$\sum_{\text{finite}} a_k \chi_{E_k}$, disjoint, $a_k$ unique is measurable if and only if $E_k \in \Sigma$. 

Every nonnegative measurable function $f$ is the limit of $f_k$ (simple), $f_k \nearrow f$, $f_k = \min(k, 2^{-k} \text{ floor}(2^k f)$, second piece round down to $j/2^k$: $\{f_k = j/2^k\} = \{ j/2^k \leq f< (j+1)/2^k\}$ or $\emptyset$. 

What does not \emph{not} generalize.

``continuous'' then measurable. Continuity is not even defined on $X$.

Lusin's Theorem does not apply.

If $f$ is measurable and $f=g$ a.e. wrt $\mu$ then $g$ not necessarily measurable. (because subsets of measure zero no longer necessarily measurable). example: $(\R, \mathcal{B}, |\cdot|)$, Borel with Lebesgue measure. in Cantor set $C$, there is a non-Borel subset $A$. $\chi_A \equiv 0$ a.e. but $\chi_A$ is not measurable. 


$X= \bigcup_{k=1}^\infty E_k$, $E_k$ has finite measure. Example: $\R$ with counting measure. ie $\infty$ on infinite sets and finite otherwise. 


\begin{dfn}
If $f \geq 0$ is measurable. $\int_E f \; d\mu = \sup\{ \sum_j a_j \mu(E_j) \}$ over simple functions $\sum a_j \chi_{E_j} \leq f \}$. 
\end{dfn}

Can choose $a_j = \inf_{E_j} f$, so $\int f \; d\mu= \sup_p L(f,P)$, where $P$ is partition of $E$ into $E_1, \ldots, E_m$, $L(f,P)= \sum_j \int_{E_j} f \mu(E_j)$. Choosing finer partition $P$, get larger $L(f,P)$. 


\begin{dfn}
If $f$ measurable and one of $\int_E f^+ \; d\mu$, $\int_E f^- \; d\mu$ is finite, define $\int_E f \; d\mu= \int_E f^+ \; d\mu - \int_E f^- \; d\mu$. if both finite, $f \in L^1(E,\mu)$. 
\end{dfn}


\begin{ex}
$(\N,2^\N,\text{counting})$, $\int f \;d\mu= \sum_{k=1}^\infty f(k)$, if abs convergent. $L(f,P)= \sum_{k=1}^N f(k)$ So integrals are series! Notice integral finite iff abs convergent. 
\end{ex}


\begin{ex}
$(\R, 2^\R, \delta_{x_0})$. $\delta_{x_0}$ the point mass at $x_0$, i.e. dirac delta 
	\[
	\delta_{x_0}= 
	\begin{cases}
	1, & x_0 \in A \\
	0, & x_0 \notin A
	\end{cases}
	\]
$\int_\R f \; d\delta_{x_0}= f(x_0)$ \pf Suppose $f$ nonnegative (otherwise, use $f^+$ and $-f^-$ and then combine later). Let $P=\{E_1,\ldots,E_m\}$ be a partition. One $E_j$ contains $x_0$, then $L(f,P)= \inf_{E_j} f \leq f(x_0)$. Hence, $\int f \leq f(x_0)$. Also, $P = \{ \{x_0\}, \R \sm \{x_0\} \}$ gives $L(f,P)= f(x_0)$. 
\end{ex}


We need another proof of MCT, since before used ``volume definition'' of $\int f$ and this does not generalize. 


\begin{lem}
if $E= A \cup B$, disjoint measurable, then $\int_E f= \int_A f + \int_B f$ for nonnegative measurable $f$. 
\end{lem}

\pf Any partition $P$ of $E$ can be refined, $(E_j$ replaced by $E_j \cap A$, $E_j \cap B$) so we get $P= P_A \cup P_B$. Then $L(f,P)= L(f_1,P_A) + L(f,P_B)$. Hence, $\int f \leq \int_A f + \int_B f$. But $P_A,P_B$ approx $\int_A f, \int_B f$ part them into $P$, conclude $\int_E f \geq \int_A f + \int_B f$. \qed \\


%%%%%%%%%%%%%%%%%%%%%%%%%%%

% Continuing 10.2

\begin{ex}
$X= \R^2$, $\Sigma=$ Borel sets, $\mu(E)= | \{x \in \R \colon (x,0) \in E\}|$. $\int f \; d\mu= \int_\R f(x,0) \; dx$. Important example of integrating along a curve. 
\end{ex}


Our goal is now to prove the MCT (if $f_k \nearrow f$, $f_k \geq 0$ measurable, then $\int f_k \to \int f$). We will need some preparation. 

Subadditivity over $f$: $\int (f+g) = \int f + \int g$, where $f,g \geq 0$ measurable. How do we prove this? For simple, easy: $f= \sum a_k \chi_{E_k}$, $g= \sum b_l \chi_{F_l}$, rewrite as $\sum_{k,l} a_k \chi_{E_k \cap F_l}$ and $\sum_{k,l} b_l \chi_{E_k \cap F_l}$. The first is integral $\sum a_k \mu(E_k \cap F_l)$ and second is integral $\sum b_l \mu(E_k \cap F_l)$. But then for the sum: $\sum (a_k+b_l) \chi_{E_k \cap F_l}$ converges to integral $\sum (a_k+b_l) \mu(E_k \cap F_l)$.


\begin{lem}
Suppose $f_k, g \geq 0$ are simple, $f_1 \leq f_2 \leq \cdots \lim f_k \geq g$. Then $\lim \int f_k \geq \int g$.
\end{lem}

\pf Partition set $E$ to make $g$ constant (on each piece) $g \equiv c$. Given $c' < c$. Let $A_k = \{f_k >c \}$. Then $A_1 \subset A_2 \subset \cdots$ and $\bigcup A_k = E$, so $\mu(A_k) \to \mu(E)$. But $\int_E f_k \geq \int_{A_k} f_k \geq c' \mu(A_k)$. So $\lim \int f_k \geq c' \mu(E)$. So $\lim \int f_k \geq c \mu(E)= \int g$. \qed \\


\begin{lem}
If $f_k \nearrow f$, $f_k$ simple nonnegative, then $\int f_k \to \int f$. 
\end{lem}

\pf For all simple $g \leq f$, we have $\lim \int f_k \geq \int g$ by the previous lemma. Take $\sup$ over $g$, get $\lim_l \int f_k \geq \int f$. But reverse is immediate since $f_k \nearrow f$ so $f_k \leq f$. \qed \\


Proof of $\int (f+g) = \int f + \int g$.

Take simple $f_k \nearrow f$, $g_k \nearrow g$. Pass to limit in $\int (f_k + g_k) = \int f_k + \int g_k$. \qed \\

Then follows for general function by writing $f^+$ and $f^-$. 


Monotone Convergence Theorem (Two forms):

Sequence Form: if $f_k$ are nonnegative, measurable, and $f_k \nearrow f$, then $\int f_k \to \int f$.

Series Form: If $g_k \geq 0$ measurable, then $\in \sum g_k = \sum \int g_k$.


These are equivalent. [The duality of sequences and series in a sense.]

\pf (Series Form since easier to prove) $\int g \geq \sum_{k=1}^\infty \int g_k$ by considering $g \geq \sum_{k=1}^N g_k$. $\int g \geq \sum_{k=1}^N \int g_k$. Need $\leq$. Introduce simple $g_{k,j} \nearrow g_k$ as $j \to \infty$.
	\[
	\begin{split}
	g_{11} \quad g_{12} \quad g_{13} \quad \cdots \to g_1 \\
	g_{21} \quad g_{22} \quad g_{23} \quad \cdots \to g_2 \\
	g_{31} \quad g_{32} \quad g_{33} \quad \cdots \to g_3 \\
	\end{split}
	\]
% Give box around upper left corner, then top to on second column, top three on third columnm, etc.

We use an ``upper triangular sum''. $s_j = \sum_{k=1}^j g_{k,j}$ which converges monotonically to $g$, i.e. $s_j \nearrow g$. By second lemma above, $\int s_j \to \int g$. Also, $\sum_{k=1}^j g_k \geq s_j$, hence $\sum_1^\infty \int g_l \geq \int g$. \qed \\



MCT implies Fatou which implies DCT.

H\"older, Minkowski: $L^p(E,\mu)$ is a Banach space. 


Now we move on to 10.3 


Given a measure space $(X, \Sigma, \mu)$ and an ASF $\phi$, we say

\begin{dfn}[Absolutely Continuous]
$\phi$ is absolutely continuous with respect to $\mu$ if $\mu(A)=0$, then $\phi(A)=0$. 
\end{dfn}


\begin{dfn}[Singular]
$\phi$ is singular with respect to $\mu$ if there exists $Z$ such that $\mu(Z)=0$ and $\phi(A)= 0$ for all $A \subset Z^C$. 
\end{dfn}


Equivanetly, for total variation $V$

AC: $\mu(A)=0$ then $V(A)=0$

S: $\mu(Z)=0$ and $V(Z^C)=0$.

Main results: every ASF is the sum of AC and S. AC ASF has density $\phi(A)= \int_A \omega$.

Examples:

a) $\mu=$ Lebesgue measure on $\R$ and $\phi(A)= \int_A f$, $f \in L^1(\R)$. THis is AC

b) same $\mu$, $\phi(A)$ is 1 if $x_0 \in A$ and $0$ if $x_0 \notin A= \delta_{x_0}$. 

Singular: let $Z= \{x_0\}$ then $\mu(Z)=0$, $V(Z^C)=0$. 


On $[0,1]$, fix $w \geq 0$ integrable. 
fix $w \geq 0$, $w \in L$, $\mu_w(A) = \int_A w$.

Is lebesuge measure on $[0,1]$ AC wrt $\mu_w$? No if $|\{w=0\}|>0$ 

Yes if $w>0$ a.e.

% Plot, axes, function w, along x-axis then linearly increasing for a bit. thick black lines, tick mark along x-axis where the linear function stops. 































