% !TEX root = ../../mat701_notes.tex
\newpage
\section{Quiz Solutions}


























% Q15
\begin{quizsol}
Suppose that $f: [0,\infty) \to [0,\infty)$ is measurable and $\int_{[0,\infty)} f < \infty$. Is it true that $\int_{[k,\infty)} f(x) \; dx \to 0$ as $k \to \infty$? 
\end{quizsol}

\pf Let $f_k:= f \chi_{[k,\infty)}$. Then goes to $0$ as $k \to \infty$ and $f$ dominates $f_k$ and $f_k \to 0$. 


Another idea: $f \chi_{(0,k)} \nearrow f$ by MCT, $\int \chi_{(0,k)} f \to \int f$. 


Or 

$R(f,E)$ has finite measure, have nested subsets from $k$ to infinite areas, nested, contained, measure sets tend to measusre of intersection which is 0? 













% Quiz16
\begin{quizsol}
Suppose that $f: \R \to \R$ is measurable and the function $g(x,y)= f(x)f(y)^2$ is integrable on $\R^2$. Is $f$ integrable on $\R^2$, i.e. is $f \in L^1(\R)$?
\end{quizsol}

\pf We do indeed have $f \in L^1(\R)$. The function given by $x \mapsto f(x)f(y)^2$ is integrable a.e. on $y$. There are two cases: $f(y)=0$ for almost all $y$ or there exists $y$ such that $f$ is integrable and $f(y) \neq 0$. In either case, $f$ is integrable. \qed \\

% This needs to be cleaned up.





% Quiz 17
\begin{quizsol}
Let $f(x)=2$ and $g(x)=3$ for all $x \in \R$. Is the convolution $f*g$ equal to 6 everywhere on $\R$?
\end{quizsol}

\pf It is the case that $f*g=6$ as
	\[
	(f*g)(x)= \int_\R f(x-t)g(t) \; dt= \int_\R 6 \; dt = \infty.
	\]


% Quiz 18
\begin{quizsol}
Suppose that $f,g \in L^\infty(\R^n)$. Is it the case that $fg \in L^\infty(\R^n)$? 
\end{quizsol}

\pf We have $|fg|=|f|\,|g| \leq |f| \, \|g\|_\infty$ for almost all $x \in \R^n$. But then
	\[
	\int |fg| \leq \int |f| \|g\|_\infty = \|g\|_\infty \int |f|= \|f\|_\infty \|g\|_\infty. 
	\]
Alternatively, $f,g \in L^\infty(\R^n)$ implies that $f,g$ are bounded almost everywhere. Suppose $|f| \leq M$ and $|g| \leq N$ for almost all $x \in \R^n$. But then $|fg| \leq MN$ for almost all $x \in \R^n$. Therefore, $fg \in L^\infty(\R^n)$. Note that it is not generally true that if $f,g \in L^p$, for some $p<\infty$, that $fg \in L^p$. For example, $\frac{1}{x^r} \in L^p((0,1))$ if and only if $pr<1$ if and only if $p<1/r$. Then generally, $(1/x^r)^2$ is not in the same $L^p$. \qed \\


%% Quiz 19 warmup
%\begin{quizsol}
%What are the possible $p$ such that for $f \in L^p([0,1])$, we have $\int_0^1 \dfrac{|f(x)|}{\sqrt{x}} \; dx < \infty$? 
%\end{quizsol}
%
%\pf Take product $f$ and $\dfrac{1}{\sqrt{x}}$. We need power $p'<2$ for $1/\sqrt{x}$ but then by conjugate powers we need $p>2$. This uses H\"older's Inequality. \qed \\


% Quiz 19
\begin{quizsol}
Suppose that $f,g: \R \to \C$ are measurable functions such that $|f|^4$ and $|g|^{4/3}$ are integrable. Prove or disprove: $fg$ is integrable.
\end{quizsol}

\pf We know $\int |fg| \leq \|f\|_4 \|g\|_{4/3}$ and $\dfrac{1}{4} + \dfrac{3}{4} = 1$. This fails if $f,g$ are not measurable. Take $g=1$ and $f$ a $(-1)^n$ type of deal on a nonmeasurable set. \qed \\


% Quiz 20
\begin{quizsol}
Suppose that $0< p \leq \infty$ and that the sequence $\{a_k\}$ belongs to $\ell^p$. Is it the case that every subsequence $\{a_{k_j}\}$ belongs to $\ell^p$?
\end{quizsol}

\pf $p<\infty$, $\sum |a_{k_j}|^p \leq \sum |a_k|^p$ and $p=\infty$, $a_k \leq M$ then $a_{k_j} \leq M$.


% Quiz 21
\begin{quizsol}
For $k \in \N$, let $f_k= \sqrt{k} \chi_{[0,1/k]}$. Is it the case that $f_k \to 0$ in $L^2(\R)$?
\end{quizsol}

\pf This is not the case as 
	\[
	\| \sqrt{k} \chi_{[0,1/k]} \|_2= \left( \int k \chi_{[0,1/k]} \right)^{1/2}= 1.
	\]
However, this does converge for $p<2$ as
	\[
	\int_0^{1/k} k^{p/2}= k^{p/2-1} \to 0
	\]
if $p<2$. \qed \\

















