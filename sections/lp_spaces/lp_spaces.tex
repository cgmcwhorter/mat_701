% !TEX root = ../../mat701_notes.tex
\newpage
\section{$L^p$ Spaces}
\subsection{Introduction}

Recall that a metric spaces is called complete if every Cauchy sequence converges. While most metric spaces are not complete, every metric space can be embedded into a complete space; that is, given a metric space $X$, there exists a space $\widetilde{X}$ such that there is an injection $\iota: X \hookrightarrow \widetilde{X}$. The space $\widetilde{X}$ can be taken to be the set of equivalence classes of Cauchy sequences in $X$. Then $X$ naturally embeds into $\tilde{X}$ as the constant sequences. In fact using this construction, $X$ is a dense subset of $\widetilde{X}$. 


\begin{ex} \hfill
\begin{enumerate}[(i)]
\item The metric spaces $\R^n$ and $\C^n$ are complete metric spaces. This essentially follows from the Bolzano-Weierstrass Theorem. 
\item The space $C[a,b]$ of continuous real-valued functions on $[a,b]$ is a Banach space, and hence is complete.
\item $\Q$ is not a complete metric space. For example, it is routine to verify the sequence $\{x_n\}$, where $x_0=1$ and $x_{n+1}= \dfrac{x_n^2+2}{2x_n}$, is Cauchy and that (using the Monotone Convergence Theorem) $\lim_{x \to \infty} x_n= \sqrt{2} \notin \Q$. 
\item The space of continuous functions on $[0,1]$ with metric $d(f,g)= \int_0^1 |f-g| \; dx$ is not complete. [Here the integral should be taken as the Riemann integral.] To see this, take any sequence of functions converging to the Dirichlet function. More concretely, it is routine to verify that the sequence of functions $\{f_n\}$, where
	\[
	f_n(x):= 
	\begin{cases}
	1, &  x=0 \\
	1-nx, & 0<x< \dfrac{1}{n} \\
	0, & \dfrac{1}{n} \leq x
	\end{cases}
	\]
is Cauchy. However, $\lim_{n \to \infty} f_n(x)$ is the zero function $[0,1]$ except at $x=0$, where the function is 1, which is not continuous. One can extend this example by considering the space of Riemann integrable functions on $[0,1]$ with the same metric $d$. Consider the sequence of compact sets $\{F_k\}$ used in constructing the Fat Cantor Set $F$, c.f. Lemma~\ref{lem:rcontfun}. We know that $\chi_{F_k}$ is Riemann integrable for all $k$ and that $\int |\chi_{F_k} - \chi_{F_j}|<\ep$ for all $k,j \geq N$, where $N$ is taken sufficiently large. However, the function $\chi_F$ is not Riemann integrable. Therefore, $\lim \chi_{F_k}$ does not exist with respect to the given metric $d$. 
\end{enumerate}
\end{ex} \xqed 


In any introductory course in Functional Analysis, a great deal of time is spent analyzing Banach spaces, which are a type of topological vector space.\footnote{To be precise, a Banach space is a complete normed vector space.} These play roles in Analysis and the rest of Pure Mathematics, Statistics, Probability, Physics, Finance, Engineering, etc.. We want to build a theory of complete spaces based on integration. The spaces we will build, the ``Lebesgue spaces'' $L^p$, in doing so will serve as important examples in Functional Analysis. 



% ------------------------
%  L^p spaces
% ------------------------
\subsection{Definition of $L^p$ and Examples}

Suppose that $E \subseteq \C^n$ or $E \subseteq \R^n$ and let $f: E \to \C$ or $f: E \to \R$. Then $f=(f_1,\ldots,f_n)$ is measurable if and only if each of the functions $f_1,\ldots,f_n$ are measurable. Equivalently, the sets $\{ a_j < f_j < b_j \} = f^{-1}(\prod_{j=1}^m (a_j,b_j))$ are measurable for all $j$ and $a_j,b_j \in \R$ if and only if for every countable union of open boxes $U$, $f^{-1}(U)$ is measurable. 


\begin{dfn}[$L^p, 0<p<\infty$]
Let $E \subseteq \R^n$ be measurable, and $0<p<\infty$. Then we define 
	\[
	L^p(E):= \left\{ f \colon \int_E |f|^p < \infty \right\}
	\]
\end{dfn}


Again, we allow $f$ to be complex-valued. In this case, if we write $f= f_1 + if_2$ for measurable real-valued functions $f_1,f_2$, we have $|f|^2= f_1^2 + f_2^2$. But then $|f_1|,|f_2| \leq |f| \leq |f_1| + |f_2|$. It follows then that $f \in L^p(E)$ if and only if $f_1,f_2 \in L^p(E)$. We write
	\[
	\nm{f}{p,E}:= \left( \int_E |f|^p \right)^{1/p}, \quad 0<p<\infty.
	\]
Therefore, $L^p(E)$ is the set of equivalence classes of measurable functions $f$ for which $\nm{f}{p,E}<\infty$. Finally, whenever the set $E$ is clear from context, we will write $L^p$ instead of $L^p(E)$ and $\nm{f}{p}$ instead of $\nm{f}{p,E}$. This this notation, we have $L^1=L$. 


\begin{rem}
Strictly speaking, $L^p$ is a set of equivalence classes of functions $f$ such that $\int_E |f|^p< \infty$; that is, we say two functions $f,g$ define the same equivalence class in $L^p(E)$ if and only if $f=g$ a.e. on $E$. 
\end{rem}


For $c \in \R$ or $c \in |C$ and $f \in L^p(E)$, it is clear that $c \cdot f \in L^p(E)$; that is, $c[f]= [cf]$. Furthermore, if $f,g \in L^p(E)$, then 
	\[
	|f+g|^p \leq (|f|+|g|)^p \leq [2 \max(|f|,|g|)]^p = 2^p \max(|f|^p,|g|^p) \leq 2^p(|f|^p+|g|^p)
	\]
so that $f+g \in L^p(E)$. This shows that $L^p$ is an $\R$, respectively $\C$, vector space. To define $L^p(E)$ in the case where $p=\infty$, we will first need a definition.


\begin{dfn}[Essential Supremum]
Let $f$ be a real or complex valued function on a measurable set $E$ of positive measure. Then we define $\essup{E} f$ as follows: if $|\{ x \in E \colon f(x)>\alpha\}|>0$ for all real $\alpha$, then we define $\essup{E} f= \infty$ and otherwise define $\essup{E} f:= \inf\{ \alpha \colon |\{x \in E \colon f(x)>\alpha\}|=0 \}$. 
\end{dfn}


The essential supremum is the generalization of the maximums to measurable functions; the difference being the fact that values of a function on sets os measure zero do not affect the essential supremum. As such, we say that $f$ is essentially bounded (or simply bounded) if $\essup{E} |f|<\infty$. 


\begin{ex} \hfill
\begin{enumerate}[(i)]
\item Take $E=\R$ and consider $f= \chi_\Q$. Then $\essup{\R} f=0$ since $|\Q|=0$ and $f \equiv 0$ on $\R \sm \Q$. 
\item Take  $E:= [0,1]$, where $F$ is the fat Cantor set, c.f. Lemma~\ref{lem:rcontfun}, and $f= \chi_F$. Then $\essup{E} \chi_F=1$ by similar reasoning to (i). 
\item Let $E= \R$ and define
	\[
	f(x)=
	\begin{cases}
	-10, & x=-1 \\
	10, & x=1 \\
	\pi, & \text{otherwise}
	\end{cases}
	\]
Then $\essup{E} f= \pi$. Notice that the largest value that $f$ takes is 10 but it takes this value on a set of measure zero.
\item Let $E=\R$ and define
	\[
	f(x)= 
	\begin{cases}
	x^3+x+1, & x \in \Q \\
	\arctan x, & x \in \R\sm\Q
	\end{cases}
	\]
As in (i), since $|\Q|=0$, we have $\essup{E}= \frac{\pi}{2}$. 
\end{enumerate}
\end{ex} \xqed


\begin{dfn}[$L^\infty(E)$]
Let $E \subseteq \R^n$ be a measurable set of positive measure. Then we define
	\[
	L^\infty(E)= \{ \text{meas } f: E \to \C \colon \essup{E} |f| < \infty \}
	\]
\end{dfn}

First, note we write $\nm{f}{\infty}= \nm{f}{\infty,E}= \essup{E} |f|$. The fact that $f \in L^\infty(E)$ is equivalent to the fact that there exists $M \in \R$ such that $|f| \leq M$ a.e.. Thus, $L^\infty(E)$ is the set of functions which are essentially bounded on $E$. 


\begin{ex} \hfill
\begin{enumerate}[(i)]
\item Let $E=\R$ and consider the function $f= \dfrac{1}{|x|}$. The function $f$ is a.e. finite on $E$ but it is not bounded so that $f \notin L^\infty$. 

\item Consider the function $f(x)= \dfrac{1}{\sqrt{x}}$ on the set $E=(0,1)$. Then $f \in L^p$ if
	\[
	\int_{(0,1)} \left( \dfrac{1}{\sqrt{x}} \right)^p < \infty.
	\]
But this occurs if and only if $p/2<1$. Therefore, $f \in L^p$ if and only if $p<2$. 

\item Let $E= [1,\infty)$ and define $f= \dfrac{1}{\sqrt{x}}$. Then $f \in L^p$ if and only if
	\[
	\int_{[1,\infty)} \dfrac{1}{x^{p/2}} \; dx.
	\]
But this happens if and only if $p>2$ or $p= \infty$. Therefore, $f \in L^p$ if and only if $p \in [2,\infty]$. 
\end{enumerate}
\end{ex} \xqed



% ------------------------
%  Relation between L^p
% ------------------------
\subsection{Relations between $L^p$ spaces}

Not all the $L^p$ spaces are `created equally'---there are `nicer' and `uglier' $L^p$ spaces. Ranked, the $L^p$ spaces are as follows:
	\begin{enumerate}
	\item[1.] (Nicest) $L^2$
	\item[2.] $L^p$, $1<p<\infty$, $p \neq 2$
	\item[3.] $L^1$
	\item[4.] $L^\infty$
	\item[5.] (Worst) $L^p$, $0<p<1$. 
	\end{enumerate}
The reasoning for placing $L^2$ first is essentially because it is probably the most familiar to the  reader. The reasoning for this ordering will make itself apparent in this section. The simplest relationship is between $L^\infty$ and $L^p$ for $p<\infty$.


\begin{prop}
If $|E|<\infty$, then $\nm{f}{\infty}= \ds\lim_{p \to \infty} \nm{f}{p}$. 
\end{prop}

\pf Let $M= \nm{f}{\infty}$. If $M' < M$, then the set $A= \{ x \in E \colon |f(x)|>M'\|$ must have positive measure. Furthermore,
	\[
	\nm{f}{p} \geq \left( \int_a |f|^p \right)^{1/p} \geq M' |A|^{1/p}.
	\]
However, $|A|^{1/p} \to 1$ as $p \to \infty$. This proves that $\liminf_{p \to \infty} \nm{f}{p} \geq M'$. Therefore, $\liminf_{p \to \infty} \nm{f}{p} \geq M$. But we also have
	\[
	\nm{f}{p} \leq \left( \int_E M^p \right)^{!/p}= M |E|^{1/p}.
	\]
Therefore, $\limsup_{p \to \infty} \nm{f}{p} \leq M$. \qed \\


Note that the requirement that $|E|<\infty$ is necessary: the constant function $f(x)=c$, $c \neq 0$, is in $L^\infty$ but is not in $L^p$ for any $0<p<\infty$. Now we show that the $L^p$ spaces are nested, and in a sense, form an interval. 


\begin{prop} \hfill
If $|E|<\infty$, and $f \in L^{p_2}$ then $f \in L^{p_1}$ for all $0<p_1<p_2$. Furthermore, if $f \in L^{p_1}$ and $f \in L^{p_2}$, then $f \in L^p$ for all $p_1<p<p_2$. That is, $\{p \colon f \in L^p\}$ is an interval starting from 0.
\end{prop}

\pf Let $f \in L^{p_2}$. Write $E= E_1 \cup E_2$, where $E_1:=\{ x \colon |f| \leq 1\}$ and $E_2:= \{ x \colon |f|>1\}$. Then we have
	\[
	\int_E |f|^p = \int_{E_1} |f|^p + \int_{E_2} |f|^p, \quad 0<p<\infty. 
	\]
Now we have $\int_{E_1} |f|^p \leq |E_1|$. The integral $\int_{E_2} |f|^p$ is an increasing function in $p$. As $f \in L^{p_2}$ and $p_1<\infty$, then $f \in L^{p_1}$. If $p_2= \infty$, then $f$ is a bounded function on a set of finite measure so that $f \in L^{p_1}$. For the second statement, suppose that $p_2<\infty$. We have
	\[
	\begin{split}
	\int_{E_1}& |f|^p \leq \int_{E_1} |f|^{p_1} < \infty \\
	\int_{E_2}& |f|^p \leq \int_{E_2} |f|^{p_2} < \infty
	\end{split}
	\]
The result then follows. Now suppose that $p_2= \infty$, and let $M:= \essup{E} |f|$. Proving $f \in L^p$ except $p_2=\infty$ is a special case. let $M= \essup |f|$. The integral $\int_{E_1} |f|^p=0$. It suffices to show the integral $\int_{E_2} |f|^p$ is finite. We have
	\[
	\int_{E_2} |f|^p \leq \int_{E_2} M^p \leq M^p |E_2| \leq M^p |\{ |f|^{p_2}>1\}| \leq M^p \int_{|f|^{p_1}>1} |f|^{p_1} < \infty 
	\] \qed \\


Again, the requirement that $|E|<\infty$ is necessary. For example, $f(x)= x^{-1/p_1} \in L^{p_2}([1,\infty))$ for $p_1<p_2$ but $f(x) \notin L^{p_1}([1,\infty))$. Moreover, any nonzero constant is in $L^\infty$ but not in $L^{p_1}(E)$ if $|E|=\infty$ and $p_1<\infty$. A function may belong to $L^{p_1}$ for $p_1<p_2$ and not belong to $L^{p_2}$. For example, if $p_2<\infty$ then $f(x)= x^{-1/p_2} \in L^{p_1}((0,1))$, where $p_1<p_2$, but $f(x) \notin L^{p_2}((0,1))$. Furthermore, $f(x)= -\log x \in L^{p_1}((0,1))$ for $p_1<\infty$ but $f(x) \notin L^\infty((0,1))$. For any set $|E|$ (even with infinite measure), if $f \in L^{p_1}(E)$, then $f \in L^{p_2}(E)$, where $p_1<p_2$. Finally, there are functions which are not in $L^p$ for any $p$. For example, if $f \equiv 1$ on $\R$, then $f \notin L^p$ for all $p$ except for $p= \infty$. If $f(x)=x$ on $\R$, then $f \notin L^p$ for all $p \in (0,\infty]$. One can find examples on finite intervals also. As a final remark for this section, if $|E|< \infty$, we have
	\[
	\essup{E} |f|= \lim_{p \to \infty} \left( \int_E |f|^p \right)^{1/p}
	\]
We are then able to write 
	\[
	\|f\|_{p,E}=
	\begin{cases} 
	\ds\left( \int_E |f|^p\right)^{1/p} \!\!\!\!, & 0<p<\infty \\ 
	\essup{E} |f|, & 0<p<1 
	\end{cases}
	\]



% ------------------------
%  Holder and Minkowski
% ------------------------
\subsection{H\"older and Minkowski Inequalities}

In this section, we prove two important and powerful inequalities: H\"older's and Minkowski's Inequality. H\"older's inequality is about controlling products while Minkowski is about controlling sums. These are among the most useful tools in Analysis. We begin with a more basic inequality. 

\begin{thm}[Young's Inequality]
Let $y= \phi(x)$ be continuous, real-valued, and strictly increasing for $x \geq 0$ with $\phi(0)=0$. If $x= \psi(y)$ is the inverse of $\phi$, then for $a,b>0$,
	\[
	ab \leq \int_0^a \phi(x) \; dx + \int_0^b \psi(y) \; dy.
	\]
with equality holds if and only if $b= \phi(a)$. 
\end{thm}

\pf The proof becomes immediate if we interpret each term as an area, and that the graph of $\phi$ may also serve as the graph of $\psi$ by interchange of the $x$ and $y$ axes. Equality holds if and only if $(a,b)$ lies on the graph of $\phi$. 
	\[
	\begin{tikzpicture}[scale=1.5,every node/.style={scale=1}]
	\node at (3.5,2.2) {$y=\phi(x)$}; 
	\node at (3.6, -0.2) {$x$};
	\node at (-0.2, 3.1) {$y$};
	\node at (2,-0.2) {$a$};
	\node at (-0.2,2) {$b$};

	\draw[thin,-latex] (-0.5,0) -- (3.5,0);
	\draw[thin,-latex] (0,-0.5) -- (0,3);
	\draw[thick,domain= 0:3,smooth,variable=\x] plot ({\x},{1.5^\x-1});
	\draw[thick, domain= 0:2,smooth,variable=\x] plot ({2},{\x});
	\draw[thick,domain= 0:2.70951,smooth,variable=\x] plot ({\x},{2});
	\end{tikzpicture}
	\] \qed \\


Now taking $\phi(x)= x^\alpha$ for $\alpha>0$, the $\psi(y)= y^{1/\alpha}$, and Young's Inequality is 
	\[
	ab \leq \dfrac{a^{1+\alpha}}{1+\alpha} + \dfrac{b^{1+1/\alpha}}{1+1/\alpha},
	\]
Setting $p= \alpha+1$ and $p'= 1+1/\alpha$, we have
	\[
	ab \leq \dfrac{a^p}{p} + \dfrac{b^{p'}}{p'}, 
	\]
where $a,b \geq 0$, $1<p<\infty$, and $\dfrac{1}{p} + \dfrac{1}{p'} = 1$.


\begin{dfn}[Conjugate Exponents]
Numbers $p,p'>0$ with $1/p+1/p'=1$ are called conjugate exponents. We make the convention that $1/\infty=0$ here.
\end{dfn}


With this terminology, the numbers 1 and $\infty$ are conjugate. Other examples are the pairs $\{1.1,11\}$, $\{1.5,3\}$, and $\{2,2\}$, which is the only number which is self-conjugate. 












\begin{thm}[Minkowski's Inequality]
If $1 \leq p \leq \infty$, then $\|f+g\|_p \leq \|f\|_p + \|g\|_p$. 
\end{thm}

\pf Proof for $1<p<\infty$. $\int |f+g|^p= \int |f+g|^{p-1} |f+g| \leq \int |f| \cdot |f+g|^{p-1} + \int |g| \, |f+g|^{p-1}$. Using H\"older's inequality, we have $\leq \|f\|_p \left( \int (|f+g|^{p-1})^{p'} \right)^{1/p'} + \|g\|_p \left( \int (|f+g|^{p-1})^{p'}\right)^{1/p'} \leq (\|f\|_p + \|g\|_p) \left( \int |f+g|^p \right)^{1-1/p}$. Cancel (divide both by rt), then $\left( \int |f+g|^p \right)^{1/p} \leq \|f\|_p + \|g\|_p$. \qed \\


For infinite series: if $\sum_k \|f_k\|_p < \infty$, then $\sum f_k \in L^p$ and $\| \sum f_k \|_p \leq \sum \|f_k\|_p$ for $1 \leq p \leq \infty$. 

\pf Minkowski then $\| \sum_{k=1}^N |f_k|\|_p \leq \sum_k \|f_k\|_p$ then MCT allows $N \to \infty$ here. So $\sum |f_k| \in L^p$. Hence, $\sum f_k \in L^p$.




Axioms of a norm: 
$\|f\| \geq 0$, $\|f\|=0$ if and only if $f=0$: Positivitiy
$\|\lambda f\|= |\lambda| \, \|f\|$ for $\lambda \in \C$ or $\R$ homogeneity
$\|f+g\| \leq \|f\|+\|g\|$ subadditivity (triangle inequality)

So $\|f\|_p$ is a norm when $1 \leq p \leq \infty$ (and not if $0<p<1$). 

For $p=1$ and $p= \infty$, this is easy: $p=1$: $\int |f+g| \leq \int (|f|+|g|)= \int |f|+ \int |g|$. For $p=\infty$, $|f| \leq M$ a.e. $|g| \leq N$ a.e., then $|f+g| \leq M+N$ a.e.. For $1<p<\infty$, we need H\"older's inequality. 







\begin{thm}[H\"older's Inequality]
If $1 \leq p \leq \infty$ and $1/p+1/p'=1$, then $\|fg\|_1 \leq \|f\|_p \|g\|_p$, i.e.
	\[
	\begin{split}
	\int_E |fg| &\leq \left( \int_E |f|^p \right)^{1/p} \left( \int_E |g|^{p'} \right)^{1/p'}, \quad 1 < p < \infty \\
	\int_E |fg| &\leq (\essup{E} |f| ) \int_E |g|. 
	\end{split}
	\]
\end{thm}

\pf The last inequality is the case when $p= \infty$, and is routine. Suppose then that $1<p<\infty$. If $\|f\|_p= \|g\|_{p'}=1$, then
	\[
	\int_E |fg| \leq \int_E \left( \dfrac{|f|^p}{p} + \dfrac{|g|^{p'}}{p'} \right)= \dfrac{\|f\|_p^p}{p} + \dfrac{\|g\|^{p'}_{p'}}{p'}= \dfrac{1}{p} + \dfrac{1}{p'}= 1 = \|f\|_p \, \|g\|_{p'}.
	\]
If $\|f\|_p$ or $\|g\|_{p'}$ is zero, then $fg$ is zero a.e. in $E$ and the result follows. We may then assume that neither $\|f\|_p$ nor $\|g\|_{p'}$ is zero. Setting 
	\[
	\widetilde{f}= \dfrac{f}{\|f\|_p} \quad \text{ and } \quad \widetilde{g}= \dfrac{g}{\|g\|_{p'}}
	\]
we have $\|\widetilde{f}\|_p=\|\widetilde{g}\|_{p'}=1$. But then in this case, the result follows by the work above. \qed \\


Notice outer exponents must always add to 1 in H\"older's Inequality. The case where $p=2$ is a classical result:

\begin{cor}[Cauchy-Schwarz Inequality]
	\[ \int_E |fg| \leq \left( \int_E |f|^2 \right)^{1/2} \left( \int_E |g|^2 \right)^{1/2} \]
\end{cor}


\begin{ex}
Suppose $f \in L^6([1,\infty])$. Want to estimate the integral $\ds \int_1^\infty \dfrac{f(x)^2}{x} \; dx$. Writing 
	\[
	\int_1^\infty \dfrac{f(x)^2}{x} \; dx= \int_!^\infty f(x)^2 \cdot \dfrac{1}{x} \; dx
	\]
We want to find a power $p'$ such that the integral $1/x^{p'}$ converges. However, we still want to make use of the given information. We have $f^2$ and $f \in L^6$. Writing $f^6=(f^2)^3$, we see that $p'= 3/2$ will work. So we have
	\[
	\begin{split}
	\int_1^\infty \dfrac{f(x)^2}{x} \; dx= \int f(x)^2 \cdot \dfrac{1}{x} \; dx  &\leq \left( \int_1^\infty (f^2)^3 \right)^{1/3} \left( \int_1^\infty \dfrac{1}{x^{3/2}} \right)^{2/3} \\
	&= \left( \left( \int_1^\infty f^6 \right)^{1/6} \right)^2 \cdot 2^{2/3} \\
	&= 2^{2/3} \|f\|_6^2
	\end{split}
	\]
\end{ex} \xqed


Suppose that $f \in L^p$ and $g \in L^q$, we have $\ds\int_\R |fg| \leq C \|f\|_p \|g\|_q$ for some $C$, then $p^{-1} + q^{-1}= 1$. Why? Rescaling to $f(\lambda x)$ and $g(\lambda x)$, we have
	\[
	\int_\R f(\lambda x) g(\lambda x) \; dx = \dfrac{1}{\lambda} \int_\R fg.
	\]
Similarly, 
	\[
	\left(\int_\R |f(\lambda x)|^p \right)^{1/p}= \left(\dfrac{1}{\lambda}\right)^{1/p} \, \|f\|_p.
	\]
But then we have
	\[
	\dfrac{1}{\lambda} \int |fg| \leq C \left( \dfrac{1}{\lambda} \right)^{1/p+1/q} \, \|f\|_p \|g\|_q.
	\]
This does not hold for all $\lambda>0$ unless $p^{-1} + q^{-1}= 1$. Note also the special case in H\"older's Inequality when $g \equiv 1$:
	\[
	\int_E |f \cdot 1| \leq \left( \int_E \|f\|^p \right)^{1/p} \left( \int_E 1^{p'} \right)^{1/p'}.
	\]
 But then 
 	\[
	\int_E |f| \leq |E|^{1/p'} \|f\|_p.
	\]
 This is vacuous if $|E|= \infty$. 


\begin{ex}
Suppose $f: \R \to \R$ is $C^1$ and $f' \in L^p$, where $1< p \leq \infty$. Using H\"older's Inequality with $1/p'$, there exists $C$ such that $|f(a)-f(b)| \leq C |a-b|^{1/p'}$. We then have
	\[
	| f(a) - f(b)| \leq \int_a^b |f'(x)| \; dx \leq (b-a)^{1/p'} \|f'\|_p.
	\]
Notice that Lipschitz continuity arises in H\"older's Inequality in the case where $p=\infty$, so that $p'=1$. 
\end{ex} \xqed 


There is a sort of converse to H\"older's Inequality which gives a dual form of the $L^p$ norm.


\begin{thm}
Let $1 \leq p \leq \infty$ and $1/p+1/p'=1$. If $f$ is real valued and measurable on $E$, then 
	\[
	\|f\|_p = \sup \left\{ \int fg \colon \|g\|_{p'}, g \text{ real} \right\}= \sup \left\{ \dfrac{\int fg}{\|g\|_{p'}} \colon g \in L^{p'}, g \text{ real} \right\}.
	\]
\end{thm}

\pf The fact that the left side is larger than the right follows from H\"older's Inequality. It remains to prove the reverse inequality. If $\|f\|_p=0$, then $f \equiv 0$ a.e. on $E$, and the result follows. If $0<\|f\|_p<\infty$, by dividing by $\|f\|_p$, we may assume that $\|f\|_p=1$. One routinely verifies that $\|g\|_{p'}= 1$ and $\ds \int_E fg= 1$, which completes this case. 

Now suppose that $\|f\|_p= \infty$. 


% p130


choose $A$ such that $0<|A|<\infty$ and for which $|f|>\|f\|_\infty - \ep$. Then let $g = \chi_A \dfrac{|f|}{f}$ and compute $(\int fg)/ \|g\|_1= (\int_A |f|)/|A|> (\|f\|_\infty - \ep)|A|/|A|> \|f\|_\infty - \ep$. If $1<p<\infty$. If $\|f\|_p < \infty$, let $g= |f|^p/f$. so $fg= |f|^p$ and $|g|^{p'}= |f|^{(p-1)p'}= |f|^p$ since $p'= p/(p-1)$. Then $(\int fg) / \|g\|_{p'}= (\int |f|^p)/ (\int |f|^p)^{1/p'}= (\int |f|^p)^{1-1/p'}= (\int |f|^p)^{1/p}$. So we can create for finite $p$ a function for which equality holds in Holders. In case $\|f\|_p= \infty$, use MCT $f_k = \text{sign }f \cdot \min(|f|,k) \chi_{[-k,k]^n}$. But then $|f_k| \nearrow |f|$. The complete proof of this is left as an exercise. 



\qed \\




































% ------------------------
%  l^p
% ------------------------
\subsection{Sequence Classes $\ell^p$}

$\ell^p= \{$ sequences $\{a_k\}$ such that $\|a\|_p<\infty\}$. $\|a\|_p=$ $(\sum_k |a_k|^p)^{1/p}$ and $\sup |a_k|$ if $p=\infty$. Claim: $\|a\|_p \leq \|a\|_q$ if $p \geq q$.  

\pf We can rescale so that $\|a\|_1$. THen $|a_k| \leq 1$ for all $k$. Hence, $\sum |a_k|^p \leq \sum |a_k|^q \leq 1$. So $\|a\|_q \leq 1$. If $p<\infty$. The case $p=\infty$ is easy. \qed \\

A sequence $\{a_k\}$ can be represented by $f= \sum_{k=1}^\infty a_k \chi_{[k,k+1)}$. Then $\|f\|_p= \|a\|_p$ as $\int |f|^p= \sum_{k=1}^\infty \int_k^{k+1} |f|^p = \sum |a_k|^p$. Given sequence $a,b$, can apply Holder/mink to corresponding functions and get 
$\sum |a_kb_k| \leq \|a\|_p \|b\|_{p'}$ Holder
$\|a+b\|_p \leq \|a\|_p + \|b\|_p$ Mink

Ex: $a_k=1/k$ is in $\ell^p \iff p>1$. More generally, $a_k= 1/k^r \in \ell^p \iff p>1/r$, where $r>0$.

For geometric sequences, $a_k= 1/2^k \in \ell^p \iff $, well for all $p \in (0,\infty]$. 

$a_k= 1/(k^r \log^{2r}(k+1)) \iff p \geq 1/r$. 


%%%%%%%%%%%%%%%%%%%%%%

Norm: $\|x\| \geq 0$, $\|x\|=0$ iff $x=0$
$\| \lambda x\|= |\lambda| \, \|x\|$
$\|x+y\| \leq \|x\|+\|y\|$
Normed vector space: $(V, \|\cdot\|)$. 

$1 \leq p \leq \infty$
$\ell^2$ norm$L^p$ 
$L^p$ has norm $\|f\|_p= \left(\int |f|^p \right)^{1/p}$
$\ell^p$ has norm $\left( \sum |a_k|^p \right)^{1/p}$

$p=\infty$, special.


Norms give metric which then have convergence, sequences, open sets, etc etc (all regular topology plus more, compact sets, close open convergence continuous functions, etc etc. So once have norm, do not need to define these terms as they come immediately from the metric immediately from the norm. 

Now the main thing for today is the following: are these metric spaces complete?

\begin{dfn}[Banach Space]
A Banach Space is a complete normed linear space.
\end{dfn}

Complete means every Cauchy sequence converges. 



Convergence in $L^p$: $f_k \to f$ means $\int |f_k-f|^p \to 0$ if $p<\infty$ or essential sup $|f_k-f| \to 0$ if $p= \infty$. 


\begin{ex}
%Piano Keys: $\chi_{[0,1]} \to \chi_{[0,1/2]}, \chi_{[1/2,1]}  \to \cdots$. We know $f_k \to 0$ in $L^p$ for $1 \leq p < \infty$ but not for $L^\infty$. Why? $\|f_k-0\|_{L^p}= \left(\int |\chi_{I_k}|^p)^{1/p}= |I_k|^{1/p} \to 0$. and $\|f_k\|_\infty=1$ for all $k$. Note $f_k(x)$ do not converge for any $x$. 
%
Conversely, $k^{-1} \chi_{[0,k]} \to 0$ pointwise but not in $L^1$. $k \chi_{[0,1/k]} \to 0$ a.e. but not in $L^1$. The first converges in $L^p$ for $1<p \leq \infty$ while the second does not converge in any $L^p$ for $1 \leq p \leq \infty$. 

\end{ex}

So convergence pointwise and convergence in $L^p$ are not related for all $p$.


\begin{lem}
If $f_k \to f$ in $L^p$, then $f_k \ma{m} f$ (convergence in measure). 
\end{lem}

\pf $1 \leq p <\infty$: $|\{|f_k-f|>\ep\}|= |\{|f_k-f|^p>\ep^p\}| \leq 1/\ep^p \int |f_k-f|^p$ by Chebyshev and right integral $\to 0$ as $k \to \infty$. $p=\infty$: $f_k \to f$ in measure means $f_k \to f$ uniformly on $E\sm Z$, $|Z|=0$ hence in measure. \qed \\


The converse false as previous example they converged in measure.


\begin{thm} 
$L^p$ and $\ell^p$, $1 \leq p \leq \infty$ are Banach spaces.
\end{thm}

%\pf Only completeness remains. $p=\infty$: $\{f_k\}$ is Cauchy iff there exists $Z$ with $|Z|=0$ such that for all $\ep>0$, there exists $N$ such that $\sup_{Z^C| |f_k-f_j|<\ep$ for all $k,j \geq N$. So for all $x \in Z^C$, $\{f_k(x)\}$ is Cauchy, hence there exists limit $f(x)= \lim_{k \to \infty} f_k(x)$. Let $j \to \infty$ in $\sup_{Z^C| |f_k-f_j|<\ep$, get $|f_k(x)-f(x)|= \lim_{j \to \infty} |f_k(x)-f_j(x)| \leq \ep$ for all $x \in Z^C$. This means $\|f_k-f\|_\infty \leq \ep$ so $f_k \to f$ in $L^\infty$. 

$1 \leq p<\infty$: Suppose $\{f_k\}$ is Cauchy in $L^p$. Then $\{f_k\}$ is Cauchy in measure. $|\{ |f_k-f_j|>\ep\}| \to 0$ as $k,j \to \infty$. By Lemma, From 4.4, we have there exists $f$ such that $f_k \ma{m} f$. there exists subsequence $f_{k_j} \to f$ a.e. for a fixed $i$.
	\[
	f_{k_i} - f_{k_j} \ma{a.e.} f_{k_i} - f \quad j \to \infty
	\]
By Fatou's Lemma: 
	\[
	\int |f_{k_i}-f|^p \leq \liminf_{j \to \infty} \int |f_{k_i} - f_{k_j}|^p
	\]
By cauchy. Conclude $\int |f_{k_i} - f|^p \leq \ep$ for sufficiently large $i$. So $f_{k_i} \to f$ in $L^p$. Hence $f_k \to f$. General fact: Cauchy sequence with convergent subsequence converges. \qed \\

%Proof that $\ell^p$ is complete: we ahve $T: \epp^p \to L^p([1,\infty))$ $Ta= \sum a_k \chi_{[k,k+1)}$ if $\{a^{(i)}\}$ is Cauchy in $\epp^p$, then $\{Ta^{(i)}\}$ is Cauchy in $L^p$. $Ta^{(i)} \to f \in L^p$. Note: $f$ is of the form $\sum b_k \chi_{[k,k+1)}$ a.e. because $Ta^{(i_j)} \ma{a.e.] f$. Then $a^{(i)} \to b$ in $\ell^p$ \qed \\


Prove separability: separable if has countable dense subset. 

\begin{thm}
$L^p$, $\ell^p$ are separable if $p<\infty$ not if $p=\infty$.

In $L^p(\R^n)$: $\{ \sum_{fin} (a_k+ib_k) \chi_{Q_k}: a_k,b_k \in \Q, Q_k \text{ dyadic cube}\}$. is dense in $L^p$.

In $L^p(E)$, just restrict to $E$ (for $p<\infty$.

In $\ell^p$ $\{(a_1,a_2,\ldots,a_m,0,0,\ldots)$ then real an imag. parts rational, $m \in \N$ is dense for $p<\infty$. 


$p=\infty$ Find coountably many elements $f_\alpha$ such that $\|f_\alpha-f_\beta\|_\infty=1$ for all $\alpha \neq \beta$ in $L^\infty(\R^n)$: $\chi_{[0,a]^n}$ $a>0$ in $\ell^\infty$ all $\{0,1\}$ sequences. $L^\infty(E)$ in $L^\infty(E)$: $|E|>0$ consider $\phi(a)= |E \cap [-a,a]^n|$, $a>0$ is continuous. 
	\[
	\phi(b)-\phi(a) \leq |[-b,b]^n \sm [-a,a]^n| = (lb)^n - (la)^n \to 0
	\] 
%as $b \to a$. $\phi(0)=0$. $\phi(a) \to |E|$ as $a \to \infty$. So IVT has all values in between 0 and $infinity $|E|>0$, so uncountably many values. and $\|\chi_{E \cap [-a,a]^n} - \chi_{E \cap [-b,b]^n}\|_\infty =1$ if $\phi(a) \neq \phi(b)$. 
\end{thm}






















