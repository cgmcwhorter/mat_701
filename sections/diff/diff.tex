% !TEX root = ../../mat701_notes.tex
\newpage
\section{Differentiation}

% Sec 7.1: Defines ASF, abs continuity, and `indefinite integral' of f \in L^1(\R^n) is F(E)= \int_E f, an absolutely continuous ASF. Just insert review definitions and statement of the theorem. 



Our goal is to `reverse engineer' $f$ knowing $F(E)$. 


\subsection{Lebesgue Differentiation (LDT)}


% Give blob with point x in it. Give descending boxes to the point x, labeled Q's. The blob labeled R^n. The idea is given 'large' set about x, probably do not have info about region near x. But taking 'closer' sets to x should reveal information. Do this with an average.

The idea is to consider $\frac{1}{|Q|} F(Q)$, the average of $f$ over $Q$, 
	\[
	\lim_{Q \searrow x} \dfrac{F(Q)}{|Q|},
	\]
where $Q$ is a cube centered at $x$ and $|Q| \to 0$. 


\begin{thm}[Lebesgue Differentiation Theorem (LDT)]
If $f \in L^1(\R^n)$, then for a.e. $x \in \R^n$, 
	\[
	f(x)=\lim_{Q \searrow x} \dfrac{F(Q)}{|Q|}.
	\]
\end{thm}


\begin{ex}
$f= \chi_{[0,1]}$. 
% $Q= \lfloor x-h, x+h \rfloor$
% At x=0,1 does not exist: \int_{0-h}^{0+h} f= h$, divide by $|Q|=2h$. Get 1/2.
% For x<0 or x>1, clealry 0.
% For 0<x<1, clearly 1. 
% This clearly also shows the importance of continuity.
% More generally, if $f$ is continuous at x, then $F(Q)/|Q| \to f(x)$ as $Q \searrow x$.
\end{ex}


Can we use Lusin's Theorem to help prove this? That is, use: for all $\ep>0$, there exists $E$ such that $f \big|_E$ is continuous and $|E^C|<\ep$. The answer is no as one cannot neglect $(\int_{Q \sm E} f)/|Q|$ as this can be large. 


Can the limit not exist at all?

\begin{ex}
$f= \sum_{n=1}^\infty \chi_{[4^{-n}, 2 \cdot 4^{-n}]}$.

Intervals disjoint. And sum clearly converges (Why?). Now let $x=0$: then 
	\[
	\dfrac{1}{2 \cdot 4^{-k}} \int_{[-4^{-k},4^{-k}]} f = \dfrac{1}{2 \cdot 4^{-k}} \sum_{n=k+1}^\infty 4^{-n}= \dfrac{4^{-k-1}/(1-1/4)}{2 \cdot 4^{-k}}= \dfrac{1}{6}.
	\]
but
	\[
	\dfrac{1}{4 \cdot 4^{-k}} \int_{[-2 \cdot 4^{-k}, 2 \cdot 4^{-k}]}= \dfrac{1}{4 \cdot 4^{-k}} \sum_{n=k}^\infty 4^{-n}= \dfrac{4^{-k}(1-1/4)}{4 \cdot 4^{-k}}= \dfrac{1}{3}. 
	\]
\end{ex}


How to approximate $L^1$ function by continuous functions.


\begin{thm}
Th eset $C_c(\R^n)$ (sometimes called $C_0$) (continuous functions with compact support) is dense in $L^1(\R^n)$. 
\end{thm}

\pf Let $\mathcal{M}= \overline{C_c(\R^n)}$. Also a linear subspace of $L^1$. Recall lienar space of $\{ \chi_I \colon I \text{ interval} \}$ is dense in $L^1$. If we can show $\chi_I \in \mathcal{M}$, we are done. ($\mathcal{M}$ is dense and closed, hence $\mathcal{M}= L^1$). Given $I$ (an interval) let $f_k = [1- k \dist(x,I)]^+$ (positive so once negative call it zero,

% Give picture horizontal line, trapezoid sitting above small interval along the x-axis, line with slope k and -k

% Definitely subset. Continuous on compact, bounded, then can integrate. Moreover, clearly linear subspace which is dense. 

	\[
	\| f_k - \chi_I \| \leq |\{ x \colon 0<\dist(x,I)<1/k \}| \to 0.
	\]

% Notice, did not use I interval more than it was closed set. So any closed set will do. 

But then $\chi_I \in \mathcal{M}$. \qed \\


Let us review the situation. Given $f \in L^1$, we have $f_n \in C_c$ such that $\| f_k - f \|_1 \to 0$ and $\frac{1}{|Q|} \int f_k \to f_k(x)$ as $Q \searrow x$ by continuity. We want to know $\lim_{Q \searrow x} \frac{1}{|Q|} \int_Q f$. From what we know, we have
	\[
	\lim_{Q \searrow x} \dfrac{1}{|Q|} \int_Q f= \lim_{Q \searrow x} \lim_{k \to \infty} \dfrac{1}{|Q|} \int_Q f_k. 
	\]
What we want to be able to do is interchange the two limits, as then using the continuity of $f_k$, we would have 
	\[
	\lim_{Q \searrow x} \lim_{k \to \infty} \dfrac{1}{|Q|} \int_Q f_k\stackrel{?}{=} \lim_{k \to \infty} \lim_{Q \searrow x} \dfrac{1}{|Q|} \int_Q f_k= \lim_{k \to \infty}(x)= f(x).
	\]
a.e. after possibly passing to a subsequence. To make the interchange, need one of them to be uniform wrt the other. Uniformity means $\frac{1}{|Q|} | \int_Q (f-f_k)|$ is small either for large $k$ and all $Q$ (independent of $k$) or for small $Q$ and all $k$ (independent of $Q$). The second is hopeless, as need uniform convergence for this. So we try the first approach. This leads to the idea of a maximal function. 


\begin{dfn}[(Hardy-Littlewood) Maximal Function]
Give $f \in L^1(\R^n)$. Let 
	\[
	f^*(x)= \sup_{Q \text{ cent. at } x} \dfrac{1}{|Q|} \int_Q |f|,
	\]
the Hardy-Littlewood maximal function.
\end{dfn}


We will use this for $f - f_k$. Since $\| f - f_k \|_1 < \ep$, we will obtain $(f-f_k)^*<\ep$ at `most points.' 

\begin{ex}
$f= \chi_{[0,1]}$
	\[
	f^*= \begin{cases}
	1, & x \in (0,1) \\
	\frac{1}{2}, & x=0,1 \\
	\dfrac{1}{2x}, & x>1 \\
	\frac{1}{2(1-x)}, &  x<0
	\end{cases}
	\]
Note $f^* \notin L^1(\R)$.

% Give plot
\end{ex}



% f= 1/sqrt(x) 0<x<1 and 0 otherwise then f^*>f.


























