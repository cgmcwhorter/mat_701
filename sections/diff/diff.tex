% !TEX root = ../../mat701_notes.tex
\newpage
\section{Differentiation}

% Sec 7.1: Defines ASF, abs continuity, and `indefinite integral' of f \in L^1(\R^n) is F(E)= \int_E f, an absolutely continuous ASF. Just insert review definitions and statement of the theorem. 



Our goal is to `reverse engineer' $f$ knowing $F(E)$. 


\subsection{Lebesgue Differentiation (LDT)}


% Give blob with point x in it. Give descending boxes to the point x, labeled Q's. The blob labeled R^n. The idea is given 'large' set about x, probably do not have info about region near x. But taking 'closer' sets to x should reveal information. Do this with an average.

The idea is to consider $\frac{1}{|Q|} F(Q)$, the average of $f$ over $Q$, 
	\[
	\lim_{Q \searrow x} \dfrac{F(Q)}{|Q|},
	\]
where $Q$ is a cube centered at $x$ and $|Q| \to 0$. 


\begin{thm}[Lebesgue Differentiation Theorem (LDT)]
If $f \in L^1(\R^n)$, then for a.e. $x \in \R^n$, 
	\[
	f(x)=\lim_{Q \searrow x} \dfrac{F(Q)}{|Q|}.
	\]
\end{thm}


\begin{ex}
$f= \chi_{[0,1]}$. 
% $Q= \lfloor x-h, x+h \rfloor$
% At x=0,1 does not exist: \int_{0-h}^{0+h} f= h$, divide by $|Q|=2h$. Get 1/2.
% For x<0 or x>1, clealry 0.
% For 0<x<1, clearly 1. 
% This clearly also shows the importance of continuity.
% More generally, if $f$ is continuous at x, then $F(Q)/|Q| \to f(x)$ as $Q \searrow x$.
\end{ex}


Can we use Lusin's Theorem to help prove this? That is, use: for all $\ep>0$, there exists $E$ such that $f \big|_E$ is continuous and $|E^C|<\ep$. The answer is no as one cannot neglect $(\int_{Q \sm E} f)/|Q|$ as this can be large. 


Can the limit not exist at all?

\begin{ex}
$f= \sum_{n=1}^\infty \chi_{[4^{-n}, 2 \cdot 4^{-n}]}$.

Intervals disjoint. And sum clearly converges (Why?). Now let $x=0$: then 
	\[
	\dfrac{1}{2 \cdot 4^{-k}} \int_{[-4^{-k},4^{-k}]} f = \dfrac{1}{2 \cdot 4^{-k}} \sum_{n=k+1}^\infty 4^{-n}= \dfrac{4^{-k-1}/(1-1/4)}{2 \cdot 4^{-k}}= \dfrac{1}{6}.
	\]
but
	\[
	\dfrac{1}{4 \cdot 4^{-k}} \int_{[-2 \cdot 4^{-k}, 2 \cdot 4^{-k}]}= \dfrac{1}{4 \cdot 4^{-k}} \sum_{n=k}^\infty 4^{-n}= \dfrac{4^{-k}(1-1/4)}{4 \cdot 4^{-k}}= \dfrac{1}{3}. 
	\]
\end{ex}


How to approximate $L^1$ function by continuous functions.


\begin{thm}
Th eset $C_c(\R^n)$ (sometimes called $C_0$) (continuous functions with compact support) is dense in $L^1(\R^n)$. 
\end{thm}

\pf Let $\mathcal{M}= \overline{C_c(\R^n)}$. Also a linear subspace of $L^1$. Recall lienar space of $\{ \chi_I \colon I \text{ interval} \}$ is dense in $L^1$. If we can show $\chi_I \in \mathcal{M}$, we are done. ($\mathcal{M}$ is dense and closed, hence $\mathcal{M}= L^1$). Given $I$ (an interval) let $f_k = [1- k \dist(x,I)]^+$ (positive so once negative call it zero,

% Give picture horizontal line, trapezoid sitting above small interval along the x-axis, line with slope k and -k

% Definitely subset. Continuous on compact, bounded, then can integrate. Moreover, clearly linear subspace which is dense. 

	\[
	\| f_k - \chi_I \| \leq |\{ x \colon 0<\dist(x,I)<1/k \}| \to 0.
	\]

% Notice, did not use I interval more than it was closed set. So any closed set will do. 

But then $\chi_I \in \mathcal{M}$. \qed \\


Let us review the situation. Given $f \in L^1$, we have $f_n \in C_c$ such that $\| f_k - f \|_1 \to 0$ and $\frac{1}{|Q|} \int f_k \to f_k(x)$ as $Q \searrow x$ by continuity. We want to know $\lim_{Q \searrow x} \frac{1}{|Q|} \int_Q f$. From what we know, we have
	\[
	\lim_{Q \searrow x} \dfrac{1}{|Q|} \int_Q f= \lim_{Q \searrow x} \lim_{k \to \infty} \dfrac{1}{|Q|} \int_Q f_k. 
	\]
What we want to be able to do is interchange the two limits, as then using the continuity of $f_k$, we would have 
	\[
	\lim_{Q \searrow x} \lim_{k \to \infty} \dfrac{1}{|Q|} \int_Q f_k\stackrel{?}{=} \lim_{k \to \infty} \lim_{Q \searrow x} \dfrac{1}{|Q|} \int_Q f_k= \lim_{k \to \infty}(x)= f(x).
	\]
a.e. after possibly passing to a subsequence. To make the interchange, need one of them to be uniform wrt the other. Uniformity means $\frac{1}{|Q|} | \int_Q (f-f_k)|$ is small either for large $k$ and all $Q$ (independent of $k$) or for small $Q$ and all $k$ (independent of $Q$). The second is hopeless, as need uniform convergence for this. So we try the first approach. This leads to the idea of a maximal function. 


\begin{dfn}[(Hardy-Littlewood) Maximal Function]
Give $f \in L^1(\R^n)$. Let 
	\[
	f^*(x)= \sup_{Q \text{ cent. at } x} \dfrac{1}{|Q|} \int_Q |f|,
	\]
the Hardy-Littlewood maximal function.
\end{dfn}


We will use this for $f - f_k$. Since $\| f - f_k \|_1 < \ep$, we will obtain $(f-f_k)^*<\ep$ at `most points.' 

\begin{ex}
$f= \chi_{[0,1]}$
	\[
	f^*= \begin{cases}
	1, & x \in (0,1) \\
	\frac{1}{2}, & x=0,1 \\
	\dfrac{1}{2x}, & x>1 \\
	\frac{1}{2(1-x)}, &  x<0
	\end{cases}
	\]
Note $f^* \notin L^1(\R)$.

% Give plot
\end{ex}



% f= 1/sqrt(x) 0<x<1 and 0 otherwise then f^*>f.








%%%%%%%%%%%%%%%%%%%%%%%%




We want to show that $f^*$ is `small' if $\| f \|_1$ is small. For this, we will need a lemma.


\begin{lem}[Simple Covering Lemma (Vitali)]
Suppose $E \subset \R^n$ has finite measure, i.e. $|E|<\infty$, and $E \subset \cup_{Q \in \cK} Q$, where $\cK$ is a collection of cubes. Then there exists a finite disjoint subcollection $\{Q_1,\ldots,Q_N\} \subset \cK$ and $\sum_{j=1}^N |Q_j| \geq 5^{-n} |E|$. 
\end{lem}

\pf Choose a compact set $F \subset E$ with $|F| > (1-\ep) |E|$. For each $Q \in \cK$, choose $Q'$ be a slightly larger open cube containing $Q$, i.e. $|Q'| < (1+\ep) |Q|$. Then $\{ Q' \colon Q \in \cK \}$ is an open cover of $F$ since this collection covers $E$. Choose a finite subcovering $\cK':= \{Q_1',\ldots, Q_M' \}$ of $F$. By possibly relabeling, without loss of generality assume $Q_1'$ be the largest cube in this collection. Again by possibly relabeling, let $Q_2'$ be the largest cube in this collection disjoint from $Q_1'$. Continue this procedure. Since there are finitely many cubes in the collection, this procedure terminates. Say $\{Q_1', \ldots, Q_N' \}$ were the cubes selected. 

We claim $\{ 3Q_1', \ldots, 3Q_N' \}$ cover $F$. Choose $x \in F$. We know that $x \in Q_j'$ for some $Q_j' \in \cK'$. If $Q_j' \in \{Q_1', \ldots, Q_N' \}$, we are done. If $Q_j' \notin \{Q_1', \ldots, Q_N' \}$, there must then exist $\tilde{Q} \in \cK'$ such that $Q' \cap \tilde{Q} \neq \emptyset$, $|\tilde{Q}| \geq |Q'|$, and $\tilde{Q} \in \{Q_1', \ldots, Q_N' \}$. Since $Q' \subset 3\tilde{Q}$, $x$ is covered by $\{Q_1', \ldots, Q_N' \}$. Hence, 
	\[
	\sum |Q_j'| = 3^{-n} \sum |3Q_j'| \geq 3^{-n} |F| > 3^{-n} (1-\ep) |E|.
	\] 
But then $\sum |Q_j| > 3^{-n} \left(\frac{1-\ep}{1+\ep}\right) |E| \geq 5^{-n} |E|$. \qed \\

	
\begin{thm}[Hardy-Littlewood]
	\[
	|\{ f^* > \alpha \}| \leq \dfrac{5^n}{\alpha} \, \|f\|_1
	\]
for all $\alpha>0$. 
\end{thm}

\pf Every $x \in E$ is covered by $Q$ with $\frac{1}{|Q|} \int_Q |f| > \alpha$. By the Simple Covering Lemma, choose disjoint cubes $Q_1,\ldots, Q_N$. Then
	\[
	\sum_j \int_{Q_j} |f| = \int_{\cup Q_j} |f| \leq \| f \|_1.
	\]
But $\sum \int_{Q_j} |f| \geq \alpha \sum |Q_j| \geq \alpha 5^{-n} |E|$. Hence, $|E| \leq \frac{5^n}{\alpha} \, \|f\|$. \qed \\


% Ex: f^* \notin L^1, f= \chi_{[0,1]}, \lambda^*= 1/(2x0$. 


Proof of LDT: Given $\ep>0$, choose $g \in C_c(\R^n)$ such that $\| f - g \|_1< 5^{-n} \ep^2$. Let $A= \{ (f-g)^* > \ep \}$. Then by Hardy-Littlewood, $|A|<\ep$. Let $B= \{ |f-g|>\ep \}$. Then $|B|<\ep$. For $x \notin A \cup B$, 
	\[
	\dfrac{1}{|Q|} \int_Q |f-f(x)| \leq 2\ep + \underbrace{\dfrac{1}{|Q|} \int_Q |g-g(x)|}_{\to 0 \text{ as } Q \searrow x}
	\]
% 1/|Q| \int_Q |f-g| < \ep, use \Delta
Hence, $\limsup_{Q \searrow x} \dfrac{1}{|Q|} \int_Q |f-f(x)| \leq 2 \ep$ for all $x \notin A \cup B$. Then 
	\[
	|\{ x \colon \limsup_{Q \searrow x} \dfrac{1}{|Q|} \int_Q |f-f(x)| \geq 2 \ep\}| \leq 2 \ep.
	\] % (*)
But this set tends to 0 in measure so that we obtain $\lim_{Q \searrow x} \dfrac{1}{|Q|} \int_Q |f-(x)|= 0$ a.e. $x$. Then by this equation, $\frac{1}{|Q|} \int_Q f \to f(x)$. \qed \\


\begin{rem}
A point where (*) holds is called a Lebesgue point of $f$. 
\end{rem}

\begin{ex}
$f(x)= \text{sgn } x$. Then $x=0$ is not a Lebesgue point. 
\end{ex}


If suffices for $f$ to be locally integrable, meaning $\int_{|x|<k} |f|<\infty$ for all $k$. Why? We only need to consider `local sets' in (*); that is, apply LDT to $f \chi_{\{|x|\leq k\}}$ and conclude a.e. point of $\{|x|<k\}$ is a Lebesgue point of $f$. In particular, we can apply to $f= \chi_E$ since its integral is at most 1. LDT says a.e. point of $E$ is a point of density, $|Q \cap E|/|Q| \to 1$ as $Q \searrow x$. 


Any cubes containing $x$ can be allowed, meaning they do not need to be centered at $x$.

% Picture: Cube Q, point x in bottom left corner. Bigger cube Q' centered at x containing Q. 

$|Q'| \leq 2^n |Q|$
$\dfrac{1}{|Q|} \int_Q |f-f(x)| \leq \dfrac{2^n}{|Q'|} \int_{Q'} |f-(x)| \to 0$. 




FTC 1: If $f \in L^1[a,b]$, then $\dfrac{d}{dx} \int_a^x f(t) \;dt= f(x)$ a.e. $x$.  


% reason: use LDT with Q= [x,x+h] or [x-h,x]$, h \to 0. Then 1/h \int_a^x+h - \int_a^x = 1/|[x,x+h] \int x^x+h f




%%%%%%%%%%%%%%%%%%%

% Warm up: Suppose $A \subset \R^n$ is closed, nonempty. Let $B= \{ x \in \R^n \colon \dist(x,A)=1 \}$. Prove $|B|=0$. 

% pf: dist function cont function of x. recall if |E|>0, then a.e. $x$ of $E$ is a point of density, i.e. $|E \cap Q|/|Q| \to 1$ as $Q \searrow x$. Suffices to show $B$ has no points of density. Given $x \in B$, let $y \in A$ be a nearest point of $A$ to $x$. Consider a sector (cone) with vertex $x$ and opening angle $\alpha< \frac{\pi}{2}$. If the cone is sufficiently small, then $\dist(z,y)<1$ for all $z$ in the interior of the cone. It is routine to verify $|z-y|<|x-y|$. Since $\dist(z,A)<1$, this cone is disjoint from $B$. But then $x$ cannot be a point of density: consider a sufficiently small cube about $x$, say $Q$. Then $|Q \cap B|/|Q| \leq 1-\ep$. 

% One can use this to show that any union of closed balls of radius 1 is measurable. 

% Give picture: blob and dotted line surrounding it, same shape, just larger. 




% 7.3
\subsection{Vitali Covering Lemma}


\begin{dfn}[Vitali Covering]
A family of cubes $\cK$ covers a set $E$ in the Vitali sense if every point of $E$ is covered by arbitrarily small cubes $Q \in \cK$. 
\end{dfn}


\begin{lem}[Vitali Covering Lemma] \label{lem:vitali}
Suppose $\cK$ covers $E$ in the Vitali sense, $0<|E|<\infty$. Then for all $\ep>0$, there exist disjoint $Q_j \in \cK$ such that $|E \sm \cup_j Q_j|= 0$ and $\sum |Q_j| < (1+\ep)|E|$. 
\end{lem}

\pf Recall the Simple Covering Lemma: disjoint $Q_j$ $\sum |Q_j| \geq 5^{-n} |E|$. Choose $\ep<1/2 5^{-n}$. Choose open $G \supset E$ such that $|G| < (1+\ep)|E|$. Discard the cubes not contained in $G$, we still have a cover of $E$. The Simple Covering gives disjoint $Q_{0,j}$ such that $\sum_j |Q_{0,j}| \geq 5^{-n} |E|$. Hence, $|G \sm \cup Q_{0,j}| < (1+ 5^{-n}/2) |E| - 5^{-n} |E|= (1-5^{-n}/2)|E|$. Let $r$ be this $r<1$. So $|E \sm \cup Q_{0,j}| \leq r|E|$, which is a finite union of closed cubes. Discard any cubes in $\cK$ that meet any $Q_{0,j}$ remaining cubes cover $E_1= E \sm \cup Q_{0,j}$ (because for all $x \in E_1$, there exists a small cube $Q \in \cK$ that contains $x$ is disjoint from $\cup Q_{0,j}$). Repeat for $E_1$, get cubes $Q_{1,j}$ such that $|E_1 \sm \cup Q_{1,j}| \leq r|E_1|$ etc.. Call $E_2$ the set on left, etc.. $E_1 \supset E_2 \supset E_3 \cdots$. IF $|E_m|=0$, stop. Otherwise, $Z= \cap E_m$ has $|Z|=0$ as $|E_m| \leq r^m |E| \to 0$. \qed \\



% \sum_{m,j} |Q_{m,j}| \leq |G| < (1+\ep)|E| % leq for disjoint

\begin{rem}
If finitely many cubes $Q_j$ are used, we obtain
	\[
	\left| E \sm \bigcup_{j=1}^N Q_j \right| < \ep.
	\]
\end{rem}










\subsection{Differentiation of Monotone Functions}

% WTS every monotone function is differentiable a.e. 


Given $f: (a,b) \to \R$, define four Dini numbers at each $x \in (a,b)$
	\[
	\begin{split}
	\overline{D}_+f(x)&= \limsup_{h \to 0^+} \dfrac{f(x+h) - f(x)}{h} \\
	\overline{D}_-f(x)&= \limsup_{h \to 0^-} \dfrac{f(x+h) - f(x)}{h} \\
	\underline{D}_+f(x)&= \liminf_{h \to 0^+} \dfrac{f(x+h) - f(x)}{h} \\
	\underline{D}_-f(x)&= \liminf_{h \to 0^-} \dfrac{f(x+h) - f(x)}{h}
	\end{split}
	\]
If $f$ is monotone, we want to prove these are equal and finite a.e. $x$. 


\begin{ex}
$f(x)=$ floor $x$.
%Give graph
At $x=1$
	\[
	\begin{split}
	\overline{D}_+&= 0 \\
	\overline{D}_-&= \infty \\
	\underline{D}_+&= 0 \\
	\underline{D}_-&= \infty
	\end{split}
	\]
\end{ex}


If $f$ is increasing, Dini numbers $\geq 0$. 


\begin{ex}
Cantor Lebesgue Staircase Function

For each $m \in \N$, the $[0,1] \sm C$ contains $2^m-1$ intervals of size $ \geq 3^{-m}$. [Why? because $2^{m-1}$ holes of size $3^{-m}$ so those $\geq 3^{-m}$ are $1+2+2^2+\cdots+2^{m-1}= 2^m-1$.] Let $f(x)= \dfrac{k}{2^m}$, $k=1,2,\ldots,2^m-1$ on those intervals. This defines $f: [0,1] \sm C \to \R$. If $|x-y| \leq 3^{-m}$, then $|f(x)-f(y)| \leq 2^{-m}$ (exercise, uniformly contin). hence, $f$ uniformly continuous. So has extension to the closure of its domain. 
\end{ex}





%%%%%%%%

\begin{thm}
If $f: (a,b) \to \R$ is increasing (not necessarily strictly), then $f'$ exists a.e., $f'>0$ a.e., and $f$ is measurable. 
\end{thm}

\pf We postpone the existence of $f'$ and prove the rest first. Observe that 
	\[
	f_k(x):= \dfrac{f(x+1/k) - f(x)}{1/k} \stackrel{\text{a.e.}}{\to} f'(x)
	\]
and each $f_k(x)$ is measurable. However, $f_k$ is defined on $(a,b-1/k)$. To fix this, let $f(x)=f(b-)$ for $x \geq b$. Then $f_k$ is defined on $(a,b)$. But then $f'$ is also measurable. It is clear that $f' \geq 0$ since $f(x+h) - f(x) \geq 0$.

To prove existence, first note that $\overline{D}_{-} \geq \underline{D}_{-}$ and $\overline{D}_+ \geq \underline{D}_+$. We only need show that these inequalities are equalities a.e.. We prove that $\overline{D}_+ \leq \underline{D}_{-}$ a.e.. It suffices to show for all $r>s$, the set $A= \{ x \in (a,b) \colon \overline{D}_+ f(x) > r > s > \underline{D}_{-} f(x) \}$ has $|A|=0$. Now $A$ is measurable as $\overline{D}_+$ and $\underline{D}_{-}$ are measurable (see HW). For all $x \in A$, there exists arbitrarily small $h>0$ such that
	\[
	\dfrac{f(x+h) - f(x)}{-h} < s.
	\]
But then $f(x) - f(x-h) < sh$. The intervals $[x-h,x]$ cover $A$ in the Vitali sense. By the Vitali Covering Lemma, for all $\ep>0$, there exist disjoint $I_j= [x_j-h_j,x_j]$ such that 
	\begin{itemize}
	\item the set $B= A \cap (\cup I_j)$ has $|B|>|A|-\ep$
	\item $\sum |I_j| < (1+\ep)|A|$
	\item $f(x_j) - f(x_j-h_j)< sh_j$
	\end{itemize}
For every $y \in B$, except for possibly at the endpoints of $I_j$, the fact that $\overline{D}_+ f(x)>r$ provides $k>0$ such that $[y, y+k] \subset I_j$ and $f(y+k) - f(y) > rk$. By the Vitali Covering, there exist disjoint $[y_i,y_i+k_i]$ such that $\sum k_i > |B|- \ep$. Hence, $\sum k_i > |A| - 2\ep$. Then
	\[
	\sum (f(y_i + k_i) - f(y_i-)> r (|A|-2\ep),
	\]
but
	\[
	\sum f(y_i+k_i) - f(y_i) \leq \sum f(x_j) - f(x_j-h_j) \leq s (1+\ep) |A|.
	\]
We then obtain $r(|A|-2\ep) \leq s(1+\ep)|A|$, a contradiction unless $|A|=0$, which is what we wanted to show. \qed \\




\begin{cor}
If $f: (a,b) \to \R$ is increasing (not necessarily strictly), then $f$ is integrable and
	\[
	\int_{(a,b)} f' \leq f(b-) - f(a+).
	\]
\end{cor}

\pf We know that $f$ is measurable. Now
	\[
	\int_{(a,b)} f_k = k \left[ \int_{(a,b)} f(x+1/k) \; dx - \int_{(a,b)} f(x) \right] = \dfrac{\int_b^{b+1/k} f(x)}{1/k} - \dfrac{\int_a^{a+1/k} f(x)}{1/k}= f(b-) - \dfrac{\int_a^{a+1/k} f(x)}{1/k}
	\]
Since $f(a+)$ exists, we know that for every $r>f(a+)$, there exists $\delta>0$ such that $f(a+) \leq f \leq r$ on $(a,a+\delta)$. Hence for $k>1/\delta$,
	\[
	f(a+) \leq \dfrac{\int_a^{a+1/k} f}{1/k} \leq r.
	\]
Hence, $\dfrac{\int_a^{a+1/k} f}{1/k} \to f(a+)$. Then 
	\[
	\lim_{k \to \infty} \int_a^b f_k = f(b-) - f(a+)
	\]
By Fatou's Lemma, $\int_a^b f' \leq \lim \int_a^b f_k = f(b-) - f(a+)$. \qed \\


% Picture: line up and to right from value at x=a to open circle at x=b, then solid point at some larger y-value.


% Why \leq? Cantor-Lebesgue f has f' \equiv 0 a.e. but f(1-) - f(0+)=f(1)-f(0)=1 by construction



\begin{cor}
BV functions are differentiable a.e.. 
\end{cor}

\pf We know that if $f$ is BV, then $f= g-h$, where $g,h$ are increasing functions. But we know that $g,h$ are differentiable a.e.. But then $f$ is differentiable a.e.. \qed \\
















