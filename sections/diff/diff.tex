% !TEX root = ../../mat701_notes.tex
\newpage
\section{Differentiation}

% Sec 7.1: Defines ASF, abs continuity, and `indefinite integral' of f \in L^1(\R^n) is F(E)= \int_E f, an absolutely continuous ASF. Just insert review definitions and statement of the theorem. 



Our goal is to `reverse engineer' $f$ knowing $F(E)$. 


\subsection{Lebesgue Differentiation (LDT)}


% Give blob with point x in it. Give descending boxes to the point x, labeled Q's. The blob labeled R^n. The idea is given 'large' set about x, probably do not have info about region near x. But taking 'closer' sets to x should reveal information. Do this with an average.

The idea is to consider $\frac{1}{|Q|} F(Q)$, the average of $f$ over $Q$, 
	\[
	\lim_{Q \searrow x} \dfrac{F(Q)}{|Q|},
	\]
where $Q$ is a cube centered at $x$ and $|Q| \to 0$. 


\begin{thm}[Lebesgue Differentiation Theorem (LDT)]
If $f \in L^1(\R^n)$, then for a.e. $x \in \R^n$, 
	\[
	f(x)=\lim_{Q \searrow x} \dfrac{F(Q)}{|Q|}.
	\]
\end{thm}


\begin{ex}
$f= \chi_{[0,1]}$. 
% $Q= \lfloor x-h, x+h \rfloor$
% At x=0,1 does not exist: \int_{0-h}^{0+h} f= h$, divide by $|Q|=2h$. Get 1/2.
% For x<0 or x>1, clealry 0.
% For 0<x<1, clearly 1. 
% This clearly also shows the importance of continuity.
% More generally, if $f$ is continuous at x, then $F(Q)/|Q| \to f(x)$ as $Q \searrow x$.
\end{ex}


Can we use Lusin's Theorem to help prove this? That is, use: for all $\ep>0$, there exists $E$ such that $f \big|_E$ is continuous and $|E^C|<\ep$. The answer is no as one cannot neglect $(\int_{Q \sm E} f)/|Q|$ as this can be large. 


Can the limit not exist at all?

\begin{ex}
$f= \sum_{n=1}^\infty \chi_{[4^{-n}, 2 \cdot 4^{-n}]}$.

Intervals disjoint. And sum clearly converges (Why?). Now let $x=0$: then 
	\[
	\dfrac{1}{2 \cdot 4^{-k}} \int_{[-4^{-k},4^{-k}]} f = \dfrac{1}{2 \cdot 4^{-k}} \sum_{n=k+1}^\infty 4^{-n}= \dfrac{4^{-k-1}/(1-1/4)}{2 \cdot 4^{-k}}= \dfrac{1}{6}.
	\]
but
	\[
	\dfrac{1}{4 \cdot 4^{-k}} \int_{[-2 \cdot 4^{-k}, 2 \cdot 4^{-k}]}= \dfrac{1}{4 \cdot 4^{-k}} \sum_{n=k}^\infty 4^{-n}= \dfrac{4^{-k}(1-1/4)}{4 \cdot 4^{-k}}= \dfrac{1}{3}. 
	\]
\end{ex}


How to approximate $L^1$ function by continuous functions.


\begin{thm}
Th eset $C_c(\R^n)$ (sometimes called $C_0$) (continuous functions with compact support) is dense in $L^1(\R^n)$. 
\end{thm}

\pf Let $\mathcal{M}= \overline{C_c(\R^n)}$. Also a linear subspace of $L^1$. Recall lienar space of $\{ \chi_I \colon I \text{ interval} \}$ is dense in $L^1$. If we can show $\chi_I \in \mathcal{M}$, we are done. ($\mathcal{M}$ is dense and closed, hence $\mathcal{M}= L^1$). Given $I$ (an interval) let $f_k = [1- k \dist(x,I)]^+$ (positive so once negative call it zero,

% Give picture horizontal line, trapezoid sitting above small interval along the x-axis, line with slope k and -k

% Definitely subset. Continuous on compact, bounded, then can integrate. Moreover, clearly linear subspace which is dense. 

	\[
	\| f_k - \chi_I \| \leq |\{ x \colon 0<\dist(x,I)<1/k \}| \to 0.
	\]

% Notice, did not use I interval more than it was closed set. So any closed set will do. 

But then $\chi_I \in \mathcal{M}$. \qed \\


Let us review the situation. Given $f \in L^1$, we have $f_n \in C_c$ such that $\| f_k - f \|_1 \to 0$ and $\frac{1}{|Q|} \int f_k \to f_k(x)$ as $Q \searrow x$ by continuity. We want to know $\lim_{Q \searrow x} \frac{1}{|Q|} \int_Q f$. From what we know, we have
	\[
	\lim_{Q \searrow x} \dfrac{1}{|Q|} \int_Q f= \lim_{Q \searrow x} \lim_{k \to \infty} \dfrac{1}{|Q|} \int_Q f_k. 
	\]
What we want to be able to do is interchange the two limits, as then using the continuity of $f_k$, we would have 
	\[
	\lim_{Q \searrow x} \lim_{k \to \infty} \dfrac{1}{|Q|} \int_Q f_k\stackrel{?}{=} \lim_{k \to \infty} \lim_{Q \searrow x} \dfrac{1}{|Q|} \int_Q f_k= \lim_{k \to \infty}(x)= f(x).
	\]
a.e. after possibly passing to a subsequence. To make the interchange, need one of them to be uniform wrt the other. Uniformity means $\frac{1}{|Q|} | \int_Q (f-f_k)|$ is small either for large $k$ and all $Q$ (independent of $k$) or for small $Q$ and all $k$ (independent of $Q$). The second is hopeless, as need uniform convergence for this. So we try the first approach. This leads to the idea of a maximal function. 


\begin{dfn}[(Hardy-Littlewood) Maximal Function]
Give $f \in L^1(\R^n)$. Let 
	\[
	f^*(x)= \sup_{Q \text{ cent. at } x} \dfrac{1}{|Q|} \int_Q |f|,
	\]
the Hardy-Littlewood maximal function.
\end{dfn}


We will use this for $f - f_k$. Since $\| f - f_k \|_1 < \ep$, we will obtain $(f-f_k)^*<\ep$ at `most points.' 

\begin{ex}
$f= \chi_{[0,1]}$
	\[
	f^*= \begin{cases}
	1, & x \in (0,1) \\
	\frac{1}{2}, & x=0,1 \\
	\dfrac{1}{2x}, & x>1 \\
	\frac{1}{2(1-x)}, &  x<0
	\end{cases}
	\]
Note $f^* \notin L^1(\R)$.

% Give plot
\end{ex}



% f= 1/sqrt(x) 0<x<1 and 0 otherwise then f^*>f.








%%%%%%%%%%%%%%%%%%%%%%%%




We want to show that $f^*$ is `small' if $\| f \|_1$ is small. For this, we will need a lemma.


\begin{lem}[Simple Covering Lemma (Vitali)]
Suppose $E \subset \R^n$ has finite measure, i.e. $|E|<\infty$, and $E \subset \cup_{Q \in \mathcal{K}} Q$, where $\mathcal{K}$ is a collection of cubes. Then there exists a finite disjoint subcollection $\{Q_1,\ldots,Q_N\} \subset \mathcal{K}$ and $\sum_{j=1}^N |Q_j| \geq 5^{-n} |E|$. 
\end{lem}

\pf Choose a compact set $F \subset E$ with $|F| > (1-\ep) |E|$. For each $Q \in \mathcal{K}$, choose $Q'$ be a slightly larger open cube containing $Q$, i.e. $|Q'| < (1+\ep) |Q|$. Then $\{ Q' \colon Q \in \mathcal{K} \}$ is an open cover of $F$ since this collection covers $E$. Choose a finite subcovering $\mathcal{K}':= \{Q_1',\ldots, Q_M' \}$ of $F$. By possibly relabeling, without loss of generality assume $Q_1'$ be the largest cube in this collection. Again by possibly relabeling, let $Q_2'$ be the largest cube in this collection disjoint from $Q_1'$. Continue this procedure. Since there are finitely many cubes in the collection, this procedure terminates. Say $\{Q_1', \ldots, Q_N' \}$ were the cubes selected. 

We claim $\{ 3Q_1', \ldots, 3Q_N' \}$ cover $F$. Choose $x \in F$. We know that $x \in Q_j'$ for some $Q_j' \in \mathcal{K}'$. If $Q_j' \in \{Q_1', \ldots, Q_N' \}$, we are done. If $Q_j' \notin \{Q_1', \ldots, Q_N' \}$, there must then exist $\tilde{Q} \in \mathcal{K}'$ such that $Q' \cap \tilde{Q} \neq \emptyset$, $|\tilde{Q}| \geq |Q'|$, and $\tilde{Q} \in \{Q_1', \ldots, Q_N' \}$. Since $Q' \subset 3\tilde{Q}$, $x$ is covered by $\{Q_1', \ldots, Q_N' \}$. Hence, 
	\[
	\sum |Q_j'| = 3^{-n} \sum |3Q_j'| \geq 3^{-n} |F| > 3^{-n} (1-\ep) |E|.
	\] 
But then $\sum |Q_j| > 3^{-n} \left(\frac{1-\ep}{1+\ep}\right) |E| \geq 5^{-n} |E|$. \qed \\

	
\begin{thm}[Hardy-Littlewood]
	\[
	|\{ f^* > \alpha \}| \leq \dfrac{5^n}{\alpha} \, \|f\|_1
	\]
for all $\alpha>0$. 
\end{thm}

\pf Every $x \in E$ is covered by $Q$ with $\frac{1}{|Q|} \int_Q |f| > \alpha$. By the Simple Covering Lemma, choose disjoint cubes $Q_1,\ldots, Q_N$. Then
	\[
	\sum_j \int_{Q_j} |f| = \int_{\cup Q_j} |f| \leq \| f \|_1.
	\]
But $\sum \int_{Q_j} |f| \geq \alpha \sum |Q_j| \geq \alpha 5^{-n} |E|$. Hence, $|E| \leq \frac{5^n}{\alpha} \, \|f\|$. \qed \\


% Ex: f^* \notin L^1, f= \chi_{[0,1]}, \lambda^*= 1/(2x0$. 


Proof of LDT: Given $\ep>0$, choose $g \in C_c(\R^n)$ such that $\| f - g \|_1< 5^{-n} \ep^2$. Let $A= \{ (f-g)^* > \ep \}$. Then by Hardy-Littlewood, $|A|<\ep$. Let $B= \{ |f-g|>\ep \}$. Then $|B|<\ep$. For $x \notin A \cup B$, 
	\[
	\dfrac{1}{|Q|} \int_Q |f-f(x)| \leq 2\ep + \underbrace{\dfrac{1}{|Q|} \int_Q |g-g(x)|}_{\to 0 \text{ as } Q \searrow x}
	\]
% 1/|Q| \int_Q |f-g| < \ep, use \Delta
Hence, $\limsup_{Q \searrow x} \dfrac{1}{|Q|} \int_Q |f-f(x)| \leq 2 \ep$ for all $x \notin A \cup B$. Then 
	\[
	|\{ x \colon \limsup_{Q \searrow x} \dfrac{1}{|Q|} \int_Q |f-f(x)| \geq 2 \ep\}| \leq 2 \ep.
	\] % (*)
But this set tends to 0 in measure so that we obtain $\lim_{Q \searrow x} \dfrac{1}{|Q|} \int_Q |f-(x)|= 0$ a.e. $x$. Then by this equation, $\frac{1}{|Q|} \int_Q f \to f(x)$. \qed \\


\begin{rem}
A point where (*) holds is called a Lebesgue point of $f$. 
\end{rem}

\begin{ex}
$f(x)= \text{sgn } x$. Then $x=0$ is not a Lebesgue point. 
\end{ex}


If suffices for $f$ to be locally integrable, meaning $\int_{|x|<k} |f|<\infty$ for all $k$. Why? We only need to consider `local sets' in (*); that is, apply LDT to $f \chi_{\{|x|\leq k\}}$ and conclude a.e. point of $\{|x|<k\}$ is a Lebesgue point of $f$. In particular, we can apply to $f= \chi_E$ since its integral is at most 1. LDT says a.e. point of $E$ is a point of density, $|Q \cap E|/|Q| \to 1$ as $Q \searrow x$. 


Any cubes containing $x$ can be allowed, meaning they do not need to be centered at $x$.

% Picture: Cube Q, point x in bottom left corner. Bigger cube Q' centered at x containing Q. 

$|Q'| \leq 2^n |Q|$
$\dfrac{1}{|Q|} \int_Q |f-f(x)| \leq \dfrac{2^n}{|Q'|} \int_{Q'} |f-(x)| \to 0$. 




FTC 1: If $f \in L^1[a,b]$, then $\dfrac{d}{dx} \int_a^x f(t) \;dt= f(x)$ a.e. $x$.  


% reason: use LDT with Q= [x,x+h] or [x-h,x]$, h \to 0. Then 1/h \int_a^x+h - \int_a^x = 1/|[x,x+h] \int x^x+h f
















