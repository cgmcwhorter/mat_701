% !TEX root = ../../mat701_notes.tex
\newpage
\section{Lebesgue Integral}


% Section 5.1

% QUIZ
% Define $f_l: [0,\infty) \to \R$ by $f_k(x)= e^{-kx}$. We claim $f_k \ma{m} 0$ as $k \to \infty$. 
% True: Yes, find $x_0 = \log(1/\ep)/k$. Can also find areas etc etc. 

Two approaches to $\int_E f$, $f \geq 0$ on $E$.

First, $\int_E f= \sup \{ \int_E g \colon g \text{ simple } g \leq f\}$, $g = \sum_{k=1}^m a_k \chi_{E_k}$, where $E_k$ are disjoint, measurable sets. Here $\int_E g = \sum a_k |E_k|$.

Second, $\int_E = |R(f,E)|$, where $R(f,E)= \{ (x,y) \in \R^{n+1} \colon x \in E, 0 \leq y \leq f(x)\}$. 

First approach is common in many texts, we shall take the second. 

So $\int_E f$ exists if and only if $R(f,E)$ is measurable. 


% plot, curve line, shaded underneath to x-axis, horizontal shaded x-axis part labeled E and shaded region labeled R(f,E).

Some advantages of the second approach are monotonicity and countable additivity over $E$: so $0 \leq f \leq g$ and $\int_E f$, $\int_E g$ exists, then $\int_E f \leq \int_E g$ because $R(f,E) < R(g,E)$. 


Also  if $E= \bigcup_{k \in \N} E_k$, disjoint measurable sets, and $\int_{E_k} f$ exists, then $\int_E f$ exists and $\int_E f= \sum_{k \in \N} \int_{E_k} f$ because $R(f,E)= \bigcup R(f,E_k)$, the union taken over the disjoint, measurable sets. Can show in other definition but takes work due to $\sup$, whereas here we get it immediately. But we have the issue of existence and measurability. 

The following main theorem. 


\begin{thm}
If $fL E \to [0,\infty]$ is measurable, then $\int_E f$ exists (possibility $\infty$). 
\end{thm}



Idea of proof is from 4.1, $f= \lim f_k$, $f_1 \leq f_2 \leq \cdots$ are simple. So we will show $R(f_,E)$ are measurable and take their union. Need Lemma 1, $R(f_k,E)$ is measurable. Lemma 2 $\Gamma(f,E)$, which needs to be defined (simply the graph $\{(x,f(x) \colon x \in E\}$) has measure 0. This is boundary top of graph of shaded area of graph, we care because $\bigcup R(f_k,E)$ need not generally cover $R(f,E)$. It will surely cover $R(f,E) \setminus \Gamma(f,E)$, then $y<f(x)$, so there exists $k$ such that $y<f_k(x)$ so that $(x,y) \in R(f_k,E)$. 




Both lemmas rely on 

\begin{lem}
If $E \subseteq \R^n$ is measurable, then $E_A:= E \times [a,b]$ is measurable in $\R^{n+1}$ with measure $|E_a|= |E| \cdot a$, noting that $0 \cdot \infty=0$ in measure theory. 
\end{lem}

\pf If $E$ is an interval, follows by volume formula. If $E$ is general, then tile it by intervals, half open (so intervals disjoint). 

If $E$ is $G_\delta$ of finite measure, write $E= \bigcap G_k$, $G_1 \supset \cdots$ are open. 
	\[
	E_a = \bigcap_k (G_k \times [0,a]) \text{ so } |E_a|= \lim |G_k \times [0,a]|= a \lim |G_k| = a \cdot |E|.
	\]
What about general set? $|E|<\infty$, $E= H \setminus Z$, $M$ is $G_\delta$, $|M|=|E|$, $|H_a|=|H| \cdot a = |E| \cdot a$. So $|E_a| \leq |E| \cdot a$. 


Also for all $\ep>0$, there exists closed $F \subset E$ (hence $G_\delta$), $|F|>|E|-\ep$. 
	\[
	|F_a|= a |F|> a|E|- a\ep
	\]
So $E_a$ contains $F_a$ closed of measure $>|E_a|_e - a\ep$. Hence $E_a$ is measurable. $|E_a|= a|E|$. 

Still assumes $a<\infty$, otherwise $a=k$ and let $k \to \infty$ \qed? \\





\begin{lem} \label{lem:1}
$R(f_k,E)= \bigcup_j (E_j \times [0,a_j])$, $E_j$ disjoint measurable sets, where $f_k = \sum_j a_j \chi_{E_j}$. 
\end{lem}



Note that this shows that if $f$ takes countably many values, $f=a_k$ on $E_k$, then $\int_E f = \sum a_k |E_k|$. 




\begin{lem} \label{lem:2}
$|\Gamma(f,E)|=0$. Let $g= 2^{-j} floor(2^jf)$, then $g \leq f \leq g+ 2^{-j}$. Then $R(g,E) \subset R(f,E) \subset $. ???? Then $\Gamma(f,E) \subset R(g+2^{-j}, E) \setminus R(g-2^{-j},E)$. This right side has measure at most $2\cdot 2^{-j} |E| \to 0$, hence $|\Gamma(f,E)|=0$. 
\end{lem}






% Missing Lecture
% Friday 09/28


% Quiz $\int_E \chi_F= |E \cap F|$, where $E,F$ are measurable. 
% True: Evalauate over simple functions

Last time: $\int_E f= |R(f,E)|$, $R(f,E)= \{ (x,y) \in \R^{n+1} \colon 0 \leq y \leq f(x), x \in E\}$. $\Gamma(f,E)= \{ (x,f(x)) \in \R^{n+1} \colon x \in E\}$. 

$\int_E f= \sup \{ \int g \colon g \leq f, g \text{ simple}\}$, $g= \sum a_k \chi_{E_k} \to \sum g= \sum a_k |E_k|$, then $R(f,E)= (E \cap F) \times [0,1]$. 

Recall $R(f,E)$ is approximated by $R(f_k,E)$:

$\cup_k R(f_k,E) \subset R(f,E) \subset \left[ \cup_k R(f_k,E) \right] \cup \Gamma(f,E)$.

% Give curve with rectangles fit, Riemann integral 

Lemma: $|\Gamma(f,E)|=0$ for all measurable $f$.

\pf Write $E= \cup E_m$, $|E_m|<\infty$

Suffices $|\Gamma(f,E_m)|=0$ for all $|\ep>0$. Let $\cA_k=\{ x \in E_m \colon \ep k \leq f(x)< \ep(k+1)\}$, $k=0,1,2,\ldots$. Then $\Gamma(f,E_m) \subset \cup_k \cA_k + [\ep_k,\ep(k+1)]$, then $|\Gamma(f,E_m)| \leq \sum_k |\cA_k| \ep = \ep |E_m|$, then $|\Gamma(f,E_m)|=0$. 



\begin{thm}[Monotone Convergence Theorem]
If $f_k \geq 0$ are measurable on $E$ and $f_1 \leq f_2 \leq \cdots$, then $\int_E \lim f_k = \lim \int_E f_k$.
\end{thm}

\pf $f= \lim f_k$, then $R(f,E)= [\cup R(f_k,E)] \cup \Gamma(f,E)$ and $\cup R(f_k,E)$ are nested. So $|R(f,E)|= \lim |R(f_k,E)|$ by continuity of measure. \qed \\


\begin{ex}
$\Q=\{q_1,q_2,\ldots\}$. Let $f_k= \chi_{\{q_1,\ldots,q_k\}}$. Then $f_k \nearrow f= \chi_\Q$. Monotone convergence theorem fails for Riemann integral, so $\int_{\chi_\Q}= \lim \int f_k=0$. 
\end{ex}



\begin{thm}
The two definitions of $\int_E f$ agree:
	\[
	\int_E f= \sup\{ \sum a_k |E_k| \colon g \leq f, g= \sum a_k \chi_{E_k}\}
	\]
\end{thm}

\pf Note $\sum a_k |E_k|= \int_E g$ by measure of product. Since $f \geq g$, we get $\int f \geq \int g$ so $\geq$ holds. Reverse: pick $g_k \nearrow f$ pointwise. Then $\int g_k \to \int f$ so $\sup \int g_k \geq \int f$. \qed \\



\begin{ex}
$f_1 \geq f_2 \geq \cdots$. Counterexample $f_k= \chi_{[k,\infty)}$. $\lim \int f_k= \infty$ but $\int \lim f_k=0$. Similarly $\inf\{ \int g \colon g \geq f, g \text{ simple}\} \neq \int f$ in general. if $g \geq f$ then $g>0$ but being simple $g \geq c>0$ then $\int g= \infty$. 
\end{ex}


Chebshev's Inequality: Suppose $f$ is nonnegative measurable on $E$. For all $\alpha>0$ $|\{ x \in E \colon f(x)>\alpha \}| \leq \frac{1}{\alpha} \int_E f$.

\pf Let $A=\{f>\alpha\}$. Then $f \geq \alpha \chi_A \to \int f \geq \int \alpha \chi_A= \alpha |A| \to |A| \leq \frac{1}{\alpha} \int f$.


\begin{ex}
$f(x)= \dfrac{1}{\sqrt{x}}$ on $[0,1)$. Let $c= \int_{(0,1)} f \to$ Chevyshev $|\{f>\alpha\}| \leq \dfrac{c}{\alpha}$. In fact, $\{f>\alpha\}= (0,1/\alpha^2) \to$ measure is $1/\alpha^2$, if $\alpha \geq 1$. 
\end{ex}



Additivity over functions: $\int_E (f+g)= \int_E f + \int_E g$, $f,g \geq 0$ measurable. 

\pf Use simple fucntions, $f_k \nearrow f$, $g_k \nearrow g$. So it suffices to prove that $\int (f_k+g_k)= \int f_k + \int g_k$. $f_k= \sum a_j \chi_{E_j}$, $g_k = \sum b_k \chi_{F_k}$. Let $H= E_i \cap F_j$ note $\cup E_i= \cup F_i = E$. Then $f_k= \sum_{i,j} a_i \chi_{H_{i,j}}$, $g_k= \sum b_j \chi_{H_{ij}}$. Then $f_k+g_k= \sum (a_i+b_j) \chi_{H_{ij}}$ then $\int (f_k+g_k) = \sum (a_i+b_j) |H_{ij}|= \sum a_i |H_{ij}| + \sum b_j |H_{ij}|$. 

Countable subadditivity:

$\int \sum_{k=1}^\infty f_k = \sum_k \int_E f_k$, $f_k \geq 0$ measurable.

True for $\sum_{k=1}^N$ then $N \to \infty$ using MCT


\begin{ex}
Let $\Q \cap [0,1]= \{q_1,\ldots\}$

$f_k= \dfrac{1}{\sqrt{|x-q_k|}}$ $\int_{[0,1]} f \leq c < \infty$

$f= \sum_{k=1}^\infty \dfrac{1}{k^2} \; \dfrac{1}{\sqrt{|x-q_k|}} \ma{\lim_{x \to q} f(x)=\infty \text{ for all } q \in \Q} \int f \leq c \sum \dfrac{1}{k^2}< \infty$ so $|\{f>\alpha\}| \leq \dfrac{1}{\alpha}\, C'$, what does this say about $\Q$?
\end{ex}




%%%%%%%%%%%%%%%%%%%%%%%%%%%%%%%%%%%%%





% Last Time: $f \geq 0$ measurable: $\int_E f= |R(f,E)|$ or $\int_E f= \sup \{ \sum a_k |E_k| \colon \sum a_k \chi_{E_k} \leq f\}$. 


% Quiz: Suppose $f: [0,\infty) \to [0,\infty)$ is measurable. Claim $\int_{[0,\infty)} e^{-x/k} f(x) \; dx \to \int_{[0,\infty)} f(x) \; dx$ as $k \to \infty$. 

% True: $x/k \searrow 0$, $e^{-x/k} \nearrow 1$, $e^{-x/k} \geq 0$, converges increasing then Monotone Convergence Theorem.






We have not yet proved $\int_E cf = c \int_E f$, where $c \geq 0$ and $f \geq 0$.

\begin{prop}
\[ \int_E cf = c \int_E f \]
\end{prop}

\pf This is true for simple functions: $\int c (\sum a_k \chi_{E_k})= \int \sum c a_k \chi_{E_k} = \sum c a_k |E_k| = c \sum a_k |E_k| = c \int \sum a_k \chi_{E_k}$. Then for general functions use the Monotone Convergence Theorem. For general $f \geq 0$, take simple $f_k \geq 0$, $f_k \nearrow f$, then $c f_k \nearrow cf$, by MCT $\int cf= \lim cf_k = c \lim \int f_k = c \int f$. \qed \\


What if the convergence is not $f_k \nearrow f$? 

\begin{ex}
$\chi_{[k,\infty)} \to 0$ pointwise. But $\int \chi_{[k,\infty)}$ are all infinite and certainly do not converge to $0$. Note that $\chi_{[k,\infty)} \searrow 0$. 
\end{ex}


\begin{ex}
$k \chi_{(0,1/k)}$. But $\int k \chi_{(0,1/k)} =1 \not\to 0$. 

% Give shaded rectangle height k and width 1/k.
\end{ex}


\begin{ex}
$\frac{1}{k} \chi_{[0,k]} \to 0$ uniformly but $\int \frac{1}{k} \chi_{[0,k]}= 1 \to 0$. 

% Give similar image as above. 
\end{ex}


What is the similarity of these three examples? The functions vanish but their integrals do not???? The general principle is that the area under the graph can disappear or `escape' to infinity but does not appear from nowhere.......... A precise statement is Fatou's Lemma. As a reminder,


\begin{dfn}[limsup,liminf]
Given a sequence $\{f_k\}$,
	\[
	\begin{split}
	\limsup_{k \to \infty} f_k = \lim_{n \to \infty} \sup_{k \geq n} f_k
	\liminf_{k \to \infty} f_k = \lim_{n \to \infty} \int_{k \geq n} f_k.
	\end{split}
	\]
\end{dfn}


\begin{ex}
\begin{enumerate}[(i)]
\item $\lim f_k(x)$ exists if and only if $\limsup f_k(x)= \liminf f_k(x)$. 
\item The Piano Key Sequence (press one key going left to right): $\chi_{[0,1]}$, then $\chi_{[0,1/2]}, \chi_{[1/2,1]}$, then $\chi_{[0,1/3]}, \chi_{[1/3,2/3]}, \chi_{[2/3,1]}$. [Go in pascal triangle order]
	\[
	\begin{split}
	\limsup f_k&= \chi_{[0,1]} \\
	\liminf f_k&= 0
	\end{split}
	\]
\end{enumerate}
\end{ex}


\begin{rem}
Relation to sequences to sets $\{E_k\}$: $\limsup E_k= \bigcap_{m=1}^\infty \bigcup_{k \geq m} E_k$, $\liminf E_k= \bigcup_{m=1}^\infty \bigcap_{k \geq m} E_k$. $\limsup \chi_{E_k}= \chi_{\limsup E_k}$ and same for liminf. and sup like union and inf like intersection. 
\end{rem}


\begin{lem}[Fatou]
$f_k \geq 0$ measurable.
\[ \int_E \liminf f_k \leq \liminf \int_E f_k \]
\end{lem}

\pf Let $g_m= \inf_{k \geq m} f_k$. Then $g_m \nearrow \liminf f_k$. By MCT, $\int g_m \to \int \liminf f_k$. But $f_k \geq g_k$ so that $\int f_k \geq \int g_k$, hence $\liminf \int f_k \geq \lim \int g_k$. [ Rem: if $a_k \geq b_k$ for all $k$, then $\liminf a_k \geq \liminf b_k$, $\limsup a_k \geq \limsup b_k$.] \qed \\


\begin{thm}[Lebesgue Dominated Convergence (DCT)]
Suppose $f_k \geq 0$ are measurable and there is $\phi \geq 0$ measurable such that $f_k \geq \phi$ ($\phi$ dominates) and $\int_E \phi< \infty$. If $f_k$ converges to $f$ a.e., then $\int_E f_k \to \int_E f$. 
\end{thm}

\pf Fatou's Lemma gives $\int f \leq \liminf \int f_k$. Also, $\phi - f_k \geq 0$. Then Fatou's Lemma applies again: $\int \phi - f \leq \liminf \int (\phi - f_k)$. Which is $\int f \geq \limsup \int f_k$. Then $\int \phi - \int f \leq \liminf (\int \phi - \int f_k)$ cancel constant $-\int f \leq \liminf (-\int f_k)$ then $\int f \geq \limsup f_k$. Comparing very first and last equations, we have $\int f= \lim_{k \to \infty} \int f_k$. \qed \\





\begin{ex}
$\frac{1}{k} \chi_{[0,k]} \to 0$ uniformly. 

% Give plot: lots of rectangles growing to right, overlay them. plot with $1/x$.

Try to cover with one function get something like $1/x$. This is not good enough for DCT since $\int \frac{!}{x}= \infty$. If $f_k \leq \frac{1}{x^2}$ on $[1,\infty)$, this exmaples does not work. `Best' you can do is $\frac{1}{k^2} \chi_{[0,k]} \to 0= \int \lim f_k$. 
\end{ex}


\begin{ex}
$\lim \int_{[1,\infty)} \dfrac{\sin^{2k} x}{x^2} \; dx$. Now $\dfrac{\sin^{2k} x}{x^2} \to 0$ a.e. and is at most $\frac{1}{x^2}$ in absolute value, so must be 0. Now if $\lim \int_{[0,\infty)} \dfrac{\sin^{2k} x}{x^2} \; dx$, $\dfrac{\sin^{2k} x}{x^2} \leq \dfrac{\sin^2 x}{x^2}$ and $\int_{[0,1]} \dfrac{\sin^2 x}{x^2} \leq \int_{[0,1]} 1 \leq 1$. so $\int_{[0,\infty)} \dfrac{\sin^2 x}{x^2}< \infty$.
\end{ex}











%%%%%%%%%%%%%%%%%%%%%%%%%%%%%%%%%%%%%%%%%




% Not all, however, need such advanced techniques. 

% $\int_0^\infty \dfrac{\sin x}{x} \; dx$
% \int_1^\infty \sin(x^2) \; dx; u=x^2 \; du=2x \; dx

% Want to define integral genreally, 

Suppose that $f: E \to \ov{\R}$ is measurable. Define functions $f^+= \max(f,0)$ and $f^-=\max(-f,0)$. These are both nonnegative functions with $f^+ - f^-=f$ and $f^+ + f^-= |f|$. 


\begin{dfn}[Integrable Function]
Suppose $f: E \to \ov{\R}$ is a measurable function, not necessarily nonnegative. We define $\int_E f$ to be
	\[
	\int_E f:= \int_E f^+ - \int_E f^-,
	\]
provided at least one of the integrals on the right is finite. We say that $f$ is integrable on $E$ if $\int_E f$ is finite, and we write $f \in L^1(E)$, 
\end{dfn}


\begin{rem}
Note that $\int f$ is finite if and only if $\int f^+$ and $\int f^-$ are finite if and only if $\int |f|$ is finite. Therefore, all Lebesgue integrable functions with finite integral are automatically absolutely integrable. This means there are integrals which are (Improper) Riemann integrable which are not Lebesgue integrable. Take Example~\ref{}, $\int_{[0,\infty)} \dfrac{\sin x}{x} \; dx$ does not exist as a Lebesgue integrable, i.e. $\dfrac{\sin x}{x} \notin L^1(0,\infty)$, while it is Riemann integrable on $(0,\infty)$. 
\end{rem}


We next examine a few properties of this integral. Not surprisingly, many of these properties resemble those of their primordial Riemann integral. 

\begin{itemize}
\item Triangle Inequality: $\left| \int_E f \right| \leq \int_E |f|$ as $\left| \int f^+ - \int f^- \right| \leq \int f^+ + \int f^-$.

\item Additivity of Domain: If $\{E_k\}_{k=1}^N$ is a collection of disjoint, measurable sets, then $\int_{\cup E_k} f = \sum_{k=1}^m \int_{E_k} f$ as $\int f^+$, $\inf f^-$ split using the properties of the integral for nonnegative functions. 

However, care is needed even in the countable additivity case. Consider $\int_0^\infty \sin x$, defining the $E_k$ to be as below:

% Give picture. Alternate E_k the positive negative parts of sin x.

We know that $\int_{E_k} f= 0$ for all $k$ but that $\int_{\cup E_k} f$ does not exist. 
\end{itemize}


\begin{lem}
Suppose that $\int_E f$ exists. Let $E= \bigcup_{k=1}^\infty E_k$ be a union of disjoint, measurable sets. Then $\int_{E_k} f$ exists for all $k$, and 
	\[
	\int_E f= \sum_{k=1}^\infty \int_{E_k} f
	\]
\end{lem}

\pf By countable additivity for integrals of nonnegative functions, we know that $\int_E f^+= \sum_k \int_{E_k} f^+$ and $\int_E f^-= \sum_k \int_{E_k} f^-$. Now observe
	\[
	\int_E f = \sum_k \left( \int_{E_k} f^+ - \int_{E_k} f^- \right).
	\]
Since $\int_E f$ exists, at least one of $\int_E f^+$, $\int_E f^-$ is finite. 

% Go through cases and explain why sum exists. 

\qed \\




Additivity over $f$:

Suppose we have $\int_E (f+g)= \int_E f + \int_E g$, assuming that both integrals on the right are finite. Let $h:= f+g$. There are eight possibilities for the signs of $(f,g,h)$. This splits $E$ into eight different parts. On each part, write a between $f,g,h$ as addition of nonnegative functions, e.g. $f \geq 0$, $g<0$, $h<0$, then $f+g=h$ becomes $f+(-h)=(-g)$, all nonnegative here. Then
	\[
	\int f + \int (-h) = \int (-g)
	\]
Note that we are making use of the fact that the functions are finite a.e.. 



Remark: if $f \in L^1(E)$, then $|f|<\infty$ a.e., since $\int |f|<\infty$. 




Convergence: 

There is little to say about convergence since everything follows from convergence of nonnegative functions. 


For example, how does MCT apply? The answer is that it does not really apply. We need an integrable `baseline', $\phi \in L^1(E)$, i.e. a function to play the role of the $x$-axis. 

MCT for nonnegative functions: 

% Picture: Volume of `piles', have small growing arcs along a x-axis.
% Picture: Same idea except have \phi curvy line. Then there are f's with almost same shape growing closer to it. Could also go from below, have these labeled g's. 

If $\phi \leq f_1 \leq f_2 \leq \cdots$ or $\phi \geq f_1 \geq f_2 \geq \cdots$, then $\int f_k \to \int \lim f_k$. 

\pf Use MCT for $f_k - \phi$ or $\phi - f_k$, nonnegative, increasing. Then $\int (f_k - \phi) \to \int (f-\phi)$; hence, $\int f_k \to \int f$. \qed \\



However, there is one useful result here:

\begin{thm}[Dominated Convergence Theorem]
If $f_k \to f$ a.e. on $E$, and are measurable, and there exists $\phi \in L^1(E)$ such that $|f_k| \leq \phi$ for all $k$, then
	\[
	\int_E f_k \longrightarrow \int_E f.
	\]
\end{thm}

% Note these integrals are finite 

\pf Since $0 \leq \underbrace{f_k+\phi}_{\geq 0} \leq 2 \phi$, DCT for nonnegative functions applies. $\int (f_k+\phi) \to \int (f+\phi)$. \qed \\


\begin{ex}
Consider $\sum_{k=1}^\infty B_k \sin(kx)$ on the interval $(0,\pi)$. Assume that $C:=\sum |B_k|<\infty$. What do we need to assume about $B_k$ to obtain
	\[
	\int_{(0,\pi)} f(x) = \sum_k \int_{(0,\pi)} B_k \sin(kx)?
	\] 
Observe that $C$ is a domination function. $f_k(x) = \sum_{j=1}^k B_j \sin(jx)$ satisfies $|f_k| \leq \phi$. Also, $f_k \to f$ pointwise. 

% Similarly, we can use $B_k = \dfrac{1}{2} \int_{(0,\pi)} f(x) \sin(kx) \; dx$, a straightforward computation gives 
\end{ex}












% Last time: $\int_E (f+g) = \int_E f + \int_E g; f,g measurable and \int_E f and \int_E g finite

Shorter proof: let $h= f+g$:
$h^+ - h^-= f^+ - f^- + g^+ - g^-$
$h^+ + f^- + g^-= f^+ + g^+ + h^-$
Integrate, apply additivitivty for nonnegative.
$\int h^+ + \int f^- + \int g^-= \int f^+ + \int g^+ + \int h^-$
all integrals finite then 
$\int h^+ - \int h^- = \int f^+ - \int f^- + \int g^+ - \int g^-$. 



DCT: Works for continuous parameters.

If we have $f_t$ ($t$ real) and $|f_t| \leq \phi$, $\phi \in L^1(E)$ and $f_t$ measurable and $f_t \to f$ a.e. as $t \to t_0$. Then $\lim_{t \to t_0} \int_E f_t= \int f$. 

\pf Sequential characterization of limits: Take any sequence $t_k \to t_0$ and apply DCT to $f_{t_k}$. Get $\int f_{t_k} \to \int f$. \qed \\


\begin{ex}
Suppose $|E|<\infty$, $f \in L^1(E)$, $f>0$ on $E$.
a) $|int_E f^p \to |E|$ as $p \to 0^+$
b) if also $\log f \in l^1(E)$, then $\int_E \dfrac{f^p-1}{p} \to \int_E \log f$ as $p \to 0^+$.


apf) $f^p \to 1$ as $p \to 0^+$. Need dominating function. Consider only $0<p \leq 1$. Then $f^p \leq f$? if $f \geq 1$ but notice $f^p \leq 1$ if $0<f<1$. So $f^p \leq \max(f,1) \leq f+1$. This is simpler than the less mysterious max. So $f+1 \in L^1(E)$ dominating function. Hence, $\int_E f^p \to \int_E 1= |E|$. 

bpf) derivative of $p \mapsto e^{p \log f}$ at $p=0$ which is $\log f$. $\dfrac{f^p-1}{p} \to \log f$. Slope of secant 

% Plot axes e^x but label f^p, given secant from (0,f(0)) to (p,f(p))

Slope inc: $\dfrac{f^p-1}{p} \leq \dfrac{f^1-1}{1}= f-1$

$f \leq 1$: $\left\| \dfrac{f^p-1}{p} \right| \leq |\log f|$.

% Same plot but e^{-x}$.


% Continued in next class:
If $f<0$ is measurable and $f \in L^1(E)$, then
	\[
	\int_E \dfrac{f^p-1}{p} \to \int_E \log f \text{ as } p \to 0^+.
	\]
We need integrability to make this possible. 


MCT applies: 
if $\phi \in L^1$ is a `baseline' and $\phi \geq f_1 \geq f_2 \geq \cdots$ or 
$\phi \leq f_1 \leq f_2 \leq \cdots$, then $\int \lim f_k = \lim \int f_k$. 

For all $p_k \searrow 0$, we want to show $f_k= \dfrac{f^{p_k} - 1}{p_k}$ is a decreasing sequence, $f_1 \geq f_2 \geq f_3 \geq \cdots$. The reason why is because of the shape of these things. 

% Give same $f^p$ graph. Secant line graph. 

Now taking $p_k$ which is smaller, so $p_{k+1}$ is to left so the slope of the secant line is less because it is a convex function. The conclusion is the same for the decreasing version because this is also convex. `Baseline' is then $f-1$. 
\end{ex}









% Monday 10/08/2018
% Section 5.4 - 5.5
\subsection{Relation between the Lebesgue and Riemann-Stieltjes Integrals}


Recall most important HW (5.1\#1): if $f \geq 0$ measurable, then $f \in L^1$ if and only if $\sum_{j= -\infty}^\infty 2^j |\{ f>2^j\}| < \infty$.  Let $\omega_f= |\{ x \in E \colon f(x)>\alpha\}|$. Notice this is just notation for something we have already discussed at length. With this notation, we can say that
	\[
	f \in L^1(E) \Longleftrightarrow \sum_{j= -\infty}^\infty 2^j \omega_{|f|}(2^j)< \infty,
	\]
where $f$ is measurable but not necessarily nonnegative. What is the significance of the `2' here? The proof of the exercise only makes use of the fact that multiplication by 2 preserves the inequality: $g= 2^j \chi_{\{2^j<f<2^{j+1}\}}$ and then $g \leq f \leq 2g$. This is like the Cauchy Condensation Test. However, this works for any real number $\lambda>1$. This allows us to then characterize the above as 
	\[
	\begin{split}
	f \in L^1(E) &\Longleftrightarrow \sum_{j= -\infty}^\infty 2^j \omega_{|f|}(2^j)< \infty \\
	&\Longleftrightarrow \sum_{j= -\infty}^\infty \lambda^j \omega_{|f|}(\lambda^j)< \infty 
	\end{split}
	\]
for any $\lambda>1$. The generalization of this is as follows:

\begin{dfn}[$L^p$]
 $f: E \to \ov{\R}$ belongs to $L^p(E)$ if $\int |f|^p<\infty$, where $0<p<\infty$. 
\end{dfn}


By the discussion above (or \#1 of HW 5.1), we know that $f \in L^p$ if and only if $\sum \lambda^j |\{ |f|^p>\lambda^j \}|<\infty$, i.e. $\sum \lambda^j |\{ |f|> \lambda^{j/p} \}|<\infty$. If we choose $\lambda= 2^p$, then
	\[
	f^p \in L^p \Longleftrightarrow \sum_{j= -\infty}^\infty 2^{pj} |\{ f>2^j \}| < \infty. 
	\]

% Notice larger p, makes life easier/harder for convergence based on power and when positive or negative. and other similar observations. 


\begin{thm}
If $|E|<\infty$ and $f: (a,b]$ is measurable, where $a,b \in \R$, then 
	\[
	\int_E f = \int_a^b \alpha d(-\omega_f(\alpha)).
	\]
\end{thm}

\pf Consider $\mathcal{P}$: $a=x_0<x_1<\cdots<x_n=b$. Let $E_i= \{x_{i-1}< f \leq x_i \}$. Note $E_i$ is a partition of $E$.
	\[
	\sum_i x_{i-1} \chi_{E_i} \leq f \leq \sum_i x_i\chi_{E_i}.
	\]
But then 
	\[
	\sum_{i=1}^n x_{i-1} |E_i| \leq \int_E f \leq \sum_{i=1}n x_i |E_i| (*)
	\]
Here 
	\[
	|E_i|= -(\underbrace{\omega(x_i)}_{f>x_i} - \underbrace{\omega(x_{i-1})}_{f>x_{i-1}}
	\]
But then the left sum is $L(\alpha,P,-\omega)$ and the right one is $U(\alpha,P,-\omega)$. The sup of $L$ and inf of $U$ both go to $\int_a^b \alpha d(-\omega)$. Thus, $\int_E f = \int_a^b f \; d(-\omega)$. \qed \\


% Remind of definition of Riemann Stieltjes integral. Put at beginning? Notice that difference of \alpha function is a measure of (x_{i-1},x_i]. So we see the connection. 







\subsection{Relation between the Lebesgue and Riemann Integral}

We want to compare $\int_{[a,b]} f$ and $R\!\!\int_a^b f$, where $a,b \in \R$ and $f$ is bounded. Following shows we have generalized.


\begin{thm}
If $R\!\!\int_a^b f$ exists, then $\int_{[a,b]} f$ exists and they are equal.
\end{thm}

\pf 
Riemann then Leb: If $R$ integral exists, then there exists sequence of partitions $\{P_k\}$ such that $U(f,P_k)$ and $L(f,P_k)$ converge to $\int_a^b f$. Arrange $P_1 \subset P_2 \subset P_3 \subset \cdots$. Let $g_k, h_k$ be step functions for these partitions. $g_k \leq f \leq h_k$ and $L(f,P_k)= \int g_k$, $U(f,P_k)= \int h_k$. 

% Give plot of small curve with g_k, h_k upper/lower rectangles. Shade colors. 

By MCT, $g_1 \leq g_2 \leq \cdots$ and $h_1 \geq h_2 \geq \cdots$. Then $\int g_k \to \int g$, where $g= \lim g_k$ and $\int h_k \to \int h$, where $h= \lim h_k$. So $\int g = \int h = R\!\!\int_a^b f$ since $U$ and $L$ converge to $\int_a^b f$. But we know that $g \leq f \leq h$. This means that we must have equality a.e.. But this completes the proof. \qed \\






The following characterization of Riemann integrals.

\begin{thm}
Riemann integral exists if and only if $|\mathcal{D}|=0$, where $\mathcal{D}$ is the set of discontinuities of $f$. 
\end{thm}

\pf See text. 





